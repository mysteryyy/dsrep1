\documentclass[11pt]{article}

    \usepackage[breakable]{tcolorbox}
    \usepackage{parskip} % Stop auto-indenting (to mimic markdown behaviour)
    
    \usepackage{iftex}
    \ifPDFTeX
    	\usepackage[T1]{fontenc}
    	\usepackage{mathpazo}
    \else
    	\usepackage{fontspec}
    \fi

    % Basic figure setup, for now with no caption control since it's done
    % automatically by Pandoc (which extracts ![](path) syntax from Markdown).
    \usepackage{graphicx}
    % Maintain compatibility with old templates. Remove in nbconvert 6.0
    \let\Oldincludegraphics\includegraphics
    % Ensure that by default, figures have no caption (until we provide a
    % proper Figure object with a Caption API and a way to capture that
    % in the conversion process - todo).
    \usepackage{caption}
    \DeclareCaptionFormat{nocaption}{}
    \captionsetup{format=nocaption,aboveskip=0pt,belowskip=0pt}

    \usepackage[Export]{adjustbox} % Used to constrain images to a maximum size
    \adjustboxset{max size={0.9\linewidth}{0.9\paperheight}}
    \usepackage{float}
    \floatplacement{figure}{H} % forces figures to be placed at the correct location
    \usepackage{xcolor} % Allow colors to be defined
    \usepackage{enumerate} % Needed for markdown enumerations to work
    \usepackage{geometry} % Used to adjust the document margins
    \usepackage{amsmath} % Equations
    \usepackage{amssymb} % Equations
    \usepackage{textcomp} % defines textquotesingle
    % Hack from http://tex.stackexchange.com/a/47451/13684:
    \AtBeginDocument{%
        \def\PYZsq{\textquotesingle}% Upright quotes in Pygmentized code
    }
    \usepackage{upquote} % Upright quotes for verbatim code
    \usepackage{eurosym} % defines \euro
    \usepackage[mathletters]{ucs} % Extended unicode (utf-8) support
    \usepackage{fancyvrb} % verbatim replacement that allows latex

    % The hyperref package gives us a pdf with properly built
    % internal navigation ('pdf bookmarks' for the table of contents,
    % internal cross-reference links, web links for URLs, etc.)
    \usepackage{hyperref}
    % The default LaTeX title has an obnoxious amount of whitespace. By default,
    % titling removes some of it. It also provides customization options.
    \usepackage{titling}
    \usepackage{longtable} % longtable support required by pandoc >1.10
    \usepackage{booktabs}  % table support for pandoc > 1.12.2
    \usepackage[inline]{enumitem} % IRkernel/repr support (it uses the enumerate* environment)
    \usepackage[normalem]{ulem} % ulem is needed to support strikethroughs (\sout)
                                % normalem makes italics be italics, not underlines
    \usepackage{mathrsfs}
    

    
    % Colors for the hyperref package
    \definecolor{urlcolor}{rgb}{0,.145,.698}
    \definecolor{linkcolor}{rgb}{.71,0.21,0.01}
    \definecolor{citecolor}{rgb}{.12,.54,.11}

    % ANSI colors
    \definecolor{ansi-black}{HTML}{3E424D}
    \definecolor{ansi-black-intense}{HTML}{282C36}
    \definecolor{ansi-red}{HTML}{E75C58}
    \definecolor{ansi-red-intense}{HTML}{B22B31}
    \definecolor{ansi-green}{HTML}{00A250}
    \definecolor{ansi-green-intense}{HTML}{007427}
    \definecolor{ansi-yellow}{HTML}{DDB62B}
    \definecolor{ansi-yellow-intense}{HTML}{B27D12}
    \definecolor{ansi-blue}{HTML}{208FFB}
    \definecolor{ansi-blue-intense}{HTML}{0065CA}
    \definecolor{ansi-magenta}{HTML}{D160C4}
    \definecolor{ansi-magenta-intense}{HTML}{A03196}
    \definecolor{ansi-cyan}{HTML}{60C6C8}
    \definecolor{ansi-cyan-intense}{HTML}{258F8F}
    \definecolor{ansi-white}{HTML}{C5C1B4}
    \definecolor{ansi-white-intense}{HTML}{A1A6B2}
    \definecolor{ansi-default-inverse-fg}{HTML}{FFFFFF}
    \definecolor{ansi-default-inverse-bg}{HTML}{000000}

    % commands and environments needed by pandoc snippets
    % extracted from the output of `pandoc -s`
    \providecommand{\tightlist}{%
      \setlength{\itemsep}{0pt}\setlength{\parskip}{0pt}}
    \DefineVerbatimEnvironment{Highlighting}{Verbatim}{commandchars=\\\{\}}
    % Add ',fontsize=\small' for more characters per line
    \newenvironment{Shaded}{}{}
    \newcommand{\KeywordTok}[1]{\textcolor[rgb]{0.00,0.44,0.13}{\textbf{{#1}}}}
    \newcommand{\DataTypeTok}[1]{\textcolor[rgb]{0.56,0.13,0.00}{{#1}}}
    \newcommand{\DecValTok}[1]{\textcolor[rgb]{0.25,0.63,0.44}{{#1}}}
    \newcommand{\BaseNTok}[1]{\textcolor[rgb]{0.25,0.63,0.44}{{#1}}}
    \newcommand{\FloatTok}[1]{\textcolor[rgb]{0.25,0.63,0.44}{{#1}}}
    \newcommand{\CharTok}[1]{\textcolor[rgb]{0.25,0.44,0.63}{{#1}}}
    \newcommand{\StringTok}[1]{\textcolor[rgb]{0.25,0.44,0.63}{{#1}}}
    \newcommand{\CommentTok}[1]{\textcolor[rgb]{0.38,0.63,0.69}{\textit{{#1}}}}
    \newcommand{\OtherTok}[1]{\textcolor[rgb]{0.00,0.44,0.13}{{#1}}}
    \newcommand{\AlertTok}[1]{\textcolor[rgb]{1.00,0.00,0.00}{\textbf{{#1}}}}
    \newcommand{\FunctionTok}[1]{\textcolor[rgb]{0.02,0.16,0.49}{{#1}}}
    \newcommand{\RegionMarkerTok}[1]{{#1}}
    \newcommand{\ErrorTok}[1]{\textcolor[rgb]{1.00,0.00,0.00}{\textbf{{#1}}}}
    \newcommand{\NormalTok}[1]{{#1}}
    
    % Additional commands for more recent versions of Pandoc
    \newcommand{\ConstantTok}[1]{\textcolor[rgb]{0.53,0.00,0.00}{{#1}}}
    \newcommand{\SpecialCharTok}[1]{\textcolor[rgb]{0.25,0.44,0.63}{{#1}}}
    \newcommand{\VerbatimStringTok}[1]{\textcolor[rgb]{0.25,0.44,0.63}{{#1}}}
    \newcommand{\SpecialStringTok}[1]{\textcolor[rgb]{0.73,0.40,0.53}{{#1}}}
    \newcommand{\ImportTok}[1]{{#1}}
    \newcommand{\DocumentationTok}[1]{\textcolor[rgb]{0.73,0.13,0.13}{\textit{{#1}}}}
    \newcommand{\AnnotationTok}[1]{\textcolor[rgb]{0.38,0.63,0.69}{\textbf{\textit{{#1}}}}}
    \newcommand{\CommentVarTok}[1]{\textcolor[rgb]{0.38,0.63,0.69}{\textbf{\textit{{#1}}}}}
    \newcommand{\VariableTok}[1]{\textcolor[rgb]{0.10,0.09,0.49}{{#1}}}
    \newcommand{\ControlFlowTok}[1]{\textcolor[rgb]{0.00,0.44,0.13}{\textbf{{#1}}}}
    \newcommand{\OperatorTok}[1]{\textcolor[rgb]{0.40,0.40,0.40}{{#1}}}
    \newcommand{\BuiltInTok}[1]{{#1}}
    \newcommand{\ExtensionTok}[1]{{#1}}
    \newcommand{\PreprocessorTok}[1]{\textcolor[rgb]{0.74,0.48,0.00}{{#1}}}
    \newcommand{\AttributeTok}[1]{\textcolor[rgb]{0.49,0.56,0.16}{{#1}}}
    \newcommand{\InformationTok}[1]{\textcolor[rgb]{0.38,0.63,0.69}{\textbf{\textit{{#1}}}}}
    \newcommand{\WarningTok}[1]{\textcolor[rgb]{0.38,0.63,0.69}{\textbf{\textit{{#1}}}}}
    
    
    % Define a nice break command that doesn't care if a line doesn't already
    % exist.
    \def\br{\hspace*{\fill} \\* }
    % Math Jax compatibility definitions
    \def\gt{>}
    \def\lt{<}
    \let\Oldtex\TeX
    \let\Oldlatex\LaTeX
    \renewcommand{\TeX}{\textrm{\Oldtex}}
    \renewcommand{\LaTeX}{\textrm{\Oldlatex}}
    % Document parameters
    % Document title
    \title{dsrep}
    
    
    
    
    
% Pygments definitions
\makeatletter
\def\PY@reset{\let\PY@it=\relax \let\PY@bf=\relax%
    \let\PY@ul=\relax \let\PY@tc=\relax%
    \let\PY@bc=\relax \let\PY@ff=\relax}
\def\PY@tok#1{\csname PY@tok@#1\endcsname}
\def\PY@toks#1+{\ifx\relax#1\empty\else%
    \PY@tok{#1}\expandafter\PY@toks\fi}
\def\PY@do#1{\PY@bc{\PY@tc{\PY@ul{%
    \PY@it{\PY@bf{\PY@ff{#1}}}}}}}
\def\PY#1#2{\PY@reset\PY@toks#1+\relax+\PY@do{#2}}

\@namedef{PY@tok@w}{\def\PY@tc##1{\textcolor[rgb]{0.73,0.73,0.73}{##1}}}
\@namedef{PY@tok@c}{\let\PY@it=\textit\def\PY@tc##1{\textcolor[rgb]{0.25,0.50,0.50}{##1}}}
\@namedef{PY@tok@cp}{\def\PY@tc##1{\textcolor[rgb]{0.74,0.48,0.00}{##1}}}
\@namedef{PY@tok@k}{\let\PY@bf=\textbf\def\PY@tc##1{\textcolor[rgb]{0.00,0.50,0.00}{##1}}}
\@namedef{PY@tok@kp}{\def\PY@tc##1{\textcolor[rgb]{0.00,0.50,0.00}{##1}}}
\@namedef{PY@tok@kt}{\def\PY@tc##1{\textcolor[rgb]{0.69,0.00,0.25}{##1}}}
\@namedef{PY@tok@o}{\def\PY@tc##1{\textcolor[rgb]{0.40,0.40,0.40}{##1}}}
\@namedef{PY@tok@ow}{\let\PY@bf=\textbf\def\PY@tc##1{\textcolor[rgb]{0.67,0.13,1.00}{##1}}}
\@namedef{PY@tok@nb}{\def\PY@tc##1{\textcolor[rgb]{0.00,0.50,0.00}{##1}}}
\@namedef{PY@tok@nf}{\def\PY@tc##1{\textcolor[rgb]{0.00,0.00,1.00}{##1}}}
\@namedef{PY@tok@nc}{\let\PY@bf=\textbf\def\PY@tc##1{\textcolor[rgb]{0.00,0.00,1.00}{##1}}}
\@namedef{PY@tok@nn}{\let\PY@bf=\textbf\def\PY@tc##1{\textcolor[rgb]{0.00,0.00,1.00}{##1}}}
\@namedef{PY@tok@ne}{\let\PY@bf=\textbf\def\PY@tc##1{\textcolor[rgb]{0.82,0.25,0.23}{##1}}}
\@namedef{PY@tok@nv}{\def\PY@tc##1{\textcolor[rgb]{0.10,0.09,0.49}{##1}}}
\@namedef{PY@tok@no}{\def\PY@tc##1{\textcolor[rgb]{0.53,0.00,0.00}{##1}}}
\@namedef{PY@tok@nl}{\def\PY@tc##1{\textcolor[rgb]{0.63,0.63,0.00}{##1}}}
\@namedef{PY@tok@ni}{\let\PY@bf=\textbf\def\PY@tc##1{\textcolor[rgb]{0.60,0.60,0.60}{##1}}}
\@namedef{PY@tok@na}{\def\PY@tc##1{\textcolor[rgb]{0.49,0.56,0.16}{##1}}}
\@namedef{PY@tok@nt}{\let\PY@bf=\textbf\def\PY@tc##1{\textcolor[rgb]{0.00,0.50,0.00}{##1}}}
\@namedef{PY@tok@nd}{\def\PY@tc##1{\textcolor[rgb]{0.67,0.13,1.00}{##1}}}
\@namedef{PY@tok@s}{\def\PY@tc##1{\textcolor[rgb]{0.73,0.13,0.13}{##1}}}
\@namedef{PY@tok@sd}{\let\PY@it=\textit\def\PY@tc##1{\textcolor[rgb]{0.73,0.13,0.13}{##1}}}
\@namedef{PY@tok@si}{\let\PY@bf=\textbf\def\PY@tc##1{\textcolor[rgb]{0.73,0.40,0.53}{##1}}}
\@namedef{PY@tok@se}{\let\PY@bf=\textbf\def\PY@tc##1{\textcolor[rgb]{0.73,0.40,0.13}{##1}}}
\@namedef{PY@tok@sr}{\def\PY@tc##1{\textcolor[rgb]{0.73,0.40,0.53}{##1}}}
\@namedef{PY@tok@ss}{\def\PY@tc##1{\textcolor[rgb]{0.10,0.09,0.49}{##1}}}
\@namedef{PY@tok@sx}{\def\PY@tc##1{\textcolor[rgb]{0.00,0.50,0.00}{##1}}}
\@namedef{PY@tok@m}{\def\PY@tc##1{\textcolor[rgb]{0.40,0.40,0.40}{##1}}}
\@namedef{PY@tok@gh}{\let\PY@bf=\textbf\def\PY@tc##1{\textcolor[rgb]{0.00,0.00,0.50}{##1}}}
\@namedef{PY@tok@gu}{\let\PY@bf=\textbf\def\PY@tc##1{\textcolor[rgb]{0.50,0.00,0.50}{##1}}}
\@namedef{PY@tok@gd}{\def\PY@tc##1{\textcolor[rgb]{0.63,0.00,0.00}{##1}}}
\@namedef{PY@tok@gi}{\def\PY@tc##1{\textcolor[rgb]{0.00,0.63,0.00}{##1}}}
\@namedef{PY@tok@gr}{\def\PY@tc##1{\textcolor[rgb]{1.00,0.00,0.00}{##1}}}
\@namedef{PY@tok@ge}{\let\PY@it=\textit}
\@namedef{PY@tok@gs}{\let\PY@bf=\textbf}
\@namedef{PY@tok@gp}{\let\PY@bf=\textbf\def\PY@tc##1{\textcolor[rgb]{0.00,0.00,0.50}{##1}}}
\@namedef{PY@tok@go}{\def\PY@tc##1{\textcolor[rgb]{0.53,0.53,0.53}{##1}}}
\@namedef{PY@tok@gt}{\def\PY@tc##1{\textcolor[rgb]{0.00,0.27,0.87}{##1}}}
\@namedef{PY@tok@err}{\def\PY@bc##1{{\setlength{\fboxsep}{\string -\fboxrule}\fcolorbox[rgb]{1.00,0.00,0.00}{1,1,1}{\strut ##1}}}}
\@namedef{PY@tok@kc}{\let\PY@bf=\textbf\def\PY@tc##1{\textcolor[rgb]{0.00,0.50,0.00}{##1}}}
\@namedef{PY@tok@kd}{\let\PY@bf=\textbf\def\PY@tc##1{\textcolor[rgb]{0.00,0.50,0.00}{##1}}}
\@namedef{PY@tok@kn}{\let\PY@bf=\textbf\def\PY@tc##1{\textcolor[rgb]{0.00,0.50,0.00}{##1}}}
\@namedef{PY@tok@kr}{\let\PY@bf=\textbf\def\PY@tc##1{\textcolor[rgb]{0.00,0.50,0.00}{##1}}}
\@namedef{PY@tok@bp}{\def\PY@tc##1{\textcolor[rgb]{0.00,0.50,0.00}{##1}}}
\@namedef{PY@tok@fm}{\def\PY@tc##1{\textcolor[rgb]{0.00,0.00,1.00}{##1}}}
\@namedef{PY@tok@vc}{\def\PY@tc##1{\textcolor[rgb]{0.10,0.09,0.49}{##1}}}
\@namedef{PY@tok@vg}{\def\PY@tc##1{\textcolor[rgb]{0.10,0.09,0.49}{##1}}}
\@namedef{PY@tok@vi}{\def\PY@tc##1{\textcolor[rgb]{0.10,0.09,0.49}{##1}}}
\@namedef{PY@tok@vm}{\def\PY@tc##1{\textcolor[rgb]{0.10,0.09,0.49}{##1}}}
\@namedef{PY@tok@sa}{\def\PY@tc##1{\textcolor[rgb]{0.73,0.13,0.13}{##1}}}
\@namedef{PY@tok@sb}{\def\PY@tc##1{\textcolor[rgb]{0.73,0.13,0.13}{##1}}}
\@namedef{PY@tok@sc}{\def\PY@tc##1{\textcolor[rgb]{0.73,0.13,0.13}{##1}}}
\@namedef{PY@tok@dl}{\def\PY@tc##1{\textcolor[rgb]{0.73,0.13,0.13}{##1}}}
\@namedef{PY@tok@s2}{\def\PY@tc##1{\textcolor[rgb]{0.73,0.13,0.13}{##1}}}
\@namedef{PY@tok@sh}{\def\PY@tc##1{\textcolor[rgb]{0.73,0.13,0.13}{##1}}}
\@namedef{PY@tok@s1}{\def\PY@tc##1{\textcolor[rgb]{0.73,0.13,0.13}{##1}}}
\@namedef{PY@tok@mb}{\def\PY@tc##1{\textcolor[rgb]{0.40,0.40,0.40}{##1}}}
\@namedef{PY@tok@mf}{\def\PY@tc##1{\textcolor[rgb]{0.40,0.40,0.40}{##1}}}
\@namedef{PY@tok@mh}{\def\PY@tc##1{\textcolor[rgb]{0.40,0.40,0.40}{##1}}}
\@namedef{PY@tok@mi}{\def\PY@tc##1{\textcolor[rgb]{0.40,0.40,0.40}{##1}}}
\@namedef{PY@tok@il}{\def\PY@tc##1{\textcolor[rgb]{0.40,0.40,0.40}{##1}}}
\@namedef{PY@tok@mo}{\def\PY@tc##1{\textcolor[rgb]{0.40,0.40,0.40}{##1}}}
\@namedef{PY@tok@ch}{\let\PY@it=\textit\def\PY@tc##1{\textcolor[rgb]{0.25,0.50,0.50}{##1}}}
\@namedef{PY@tok@cm}{\let\PY@it=\textit\def\PY@tc##1{\textcolor[rgb]{0.25,0.50,0.50}{##1}}}
\@namedef{PY@tok@cpf}{\let\PY@it=\textit\def\PY@tc##1{\textcolor[rgb]{0.25,0.50,0.50}{##1}}}
\@namedef{PY@tok@c1}{\let\PY@it=\textit\def\PY@tc##1{\textcolor[rgb]{0.25,0.50,0.50}{##1}}}
\@namedef{PY@tok@cs}{\let\PY@it=\textit\def\PY@tc##1{\textcolor[rgb]{0.25,0.50,0.50}{##1}}}

\def\PYZbs{\char`\\}
\def\PYZus{\char`\_}
\def\PYZob{\char`\{}
\def\PYZcb{\char`\}}
\def\PYZca{\char`\^}
\def\PYZam{\char`\&}
\def\PYZlt{\char`\<}
\def\PYZgt{\char`\>}
\def\PYZsh{\char`\#}
\def\PYZpc{\char`\%}
\def\PYZdl{\char`\$}
\def\PYZhy{\char`\-}
\def\PYZsq{\char`\'}
\def\PYZdq{\char`\"}
\def\PYZti{\char`\~}
% for compatibility with earlier versions
\def\PYZat{@}
\def\PYZlb{[}
\def\PYZrb{]}
\makeatother


    % For linebreaks inside Verbatim environment from package fancyvrb. 
    \makeatletter
        \newbox\Wrappedcontinuationbox 
        \newbox\Wrappedvisiblespacebox 
        \newcommand*\Wrappedvisiblespace {\textcolor{red}{\textvisiblespace}} 
        \newcommand*\Wrappedcontinuationsymbol {\textcolor{red}{\llap{\tiny$\m@th\hookrightarrow$}}} 
        \newcommand*\Wrappedcontinuationindent {3ex } 
        \newcommand*\Wrappedafterbreak {\kern\Wrappedcontinuationindent\copy\Wrappedcontinuationbox} 
        % Take advantage of the already applied Pygments mark-up to insert 
        % potential linebreaks for TeX processing. 
        %        {, <, #, %, $, ' and ": go to next line. 
        %        _, }, ^, &, >, - and ~: stay at end of broken line. 
        % Use of \textquotesingle for straight quote. 
        \newcommand*\Wrappedbreaksatspecials {% 
            \def\PYGZus{\discretionary{\char`\_}{\Wrappedafterbreak}{\char`\_}}% 
            \def\PYGZob{\discretionary{}{\Wrappedafterbreak\char`\{}{\char`\{}}% 
            \def\PYGZcb{\discretionary{\char`\}}{\Wrappedafterbreak}{\char`\}}}% 
            \def\PYGZca{\discretionary{\char`\^}{\Wrappedafterbreak}{\char`\^}}% 
            \def\PYGZam{\discretionary{\char`\&}{\Wrappedafterbreak}{\char`\&}}% 
            \def\PYGZlt{\discretionary{}{\Wrappedafterbreak\char`\<}{\char`\<}}% 
            \def\PYGZgt{\discretionary{\char`\>}{\Wrappedafterbreak}{\char`\>}}% 
            \def\PYGZsh{\discretionary{}{\Wrappedafterbreak\char`\#}{\char`\#}}% 
            \def\PYGZpc{\discretionary{}{\Wrappedafterbreak\char`\%}{\char`\%}}% 
            \def\PYGZdl{\discretionary{}{\Wrappedafterbreak\char`\$}{\char`\$}}% 
            \def\PYGZhy{\discretionary{\char`\-}{\Wrappedafterbreak}{\char`\-}}% 
            \def\PYGZsq{\discretionary{}{\Wrappedafterbreak\textquotesingle}{\textquotesingle}}% 
            \def\PYGZdq{\discretionary{}{\Wrappedafterbreak\char`\"}{\char`\"}}% 
            \def\PYGZti{\discretionary{\char`\~}{\Wrappedafterbreak}{\char`\~}}% 
        } 
        % Some characters . , ; ? ! / are not pygmentized. 
        % This macro makes them "active" and they will insert potential linebreaks 
        \newcommand*\Wrappedbreaksatpunct {% 
            \lccode`\~`\.\lowercase{\def~}{\discretionary{\hbox{\char`\.}}{\Wrappedafterbreak}{\hbox{\char`\.}}}% 
            \lccode`\~`\,\lowercase{\def~}{\discretionary{\hbox{\char`\,}}{\Wrappedafterbreak}{\hbox{\char`\,}}}% 
            \lccode`\~`\;\lowercase{\def~}{\discretionary{\hbox{\char`\;}}{\Wrappedafterbreak}{\hbox{\char`\;}}}% 
            \lccode`\~`\:\lowercase{\def~}{\discretionary{\hbox{\char`\:}}{\Wrappedafterbreak}{\hbox{\char`\:}}}% 
            \lccode`\~`\?\lowercase{\def~}{\discretionary{\hbox{\char`\?}}{\Wrappedafterbreak}{\hbox{\char`\?}}}% 
            \lccode`\~`\!\lowercase{\def~}{\discretionary{\hbox{\char`\!}}{\Wrappedafterbreak}{\hbox{\char`\!}}}% 
            \lccode`\~`\/\lowercase{\def~}{\discretionary{\hbox{\char`\/}}{\Wrappedafterbreak}{\hbox{\char`\/}}}% 
            \catcode`\.\active
            \catcode`\,\active 
            \catcode`\;\active
            \catcode`\:\active
            \catcode`\?\active
            \catcode`\!\active
            \catcode`\/\active 
            \lccode`\~`\~ 	
        }
    \makeatother

    \let\OriginalVerbatim=\Verbatim
    \makeatletter
    \renewcommand{\Verbatim}[1][1]{%
        %\parskip\z@skip
        \sbox\Wrappedcontinuationbox {\Wrappedcontinuationsymbol}%
        \sbox\Wrappedvisiblespacebox {\FV@SetupFont\Wrappedvisiblespace}%
        \def\FancyVerbFormatLine ##1{\hsize\linewidth
            \vtop{\raggedright\hyphenpenalty\z@\exhyphenpenalty\z@
                \doublehyphendemerits\z@\finalhyphendemerits\z@
                \strut ##1\strut}%
        }%
        % If the linebreak is at a space, the latter will be displayed as visible
        % space at end of first line, and a continuation symbol starts next line.
        % Stretch/shrink are however usually zero for typewriter font.
        \def\FV@Space {%
            \nobreak\hskip\z@ plus\fontdimen3\font minus\fontdimen4\font
            \discretionary{\copy\Wrappedvisiblespacebox}{\Wrappedafterbreak}
            {\kern\fontdimen2\font}%
        }%
        
        % Allow breaks at special characters using \PYG... macros.
        \Wrappedbreaksatspecials
        % Breaks at punctuation characters . , ; ? ! and / need catcode=\active 	
        \OriginalVerbatim[#1,codes*=\Wrappedbreaksatpunct]%
    }
    \makeatother

    % Exact colors from NB
    \definecolor{incolor}{HTML}{303F9F}
    \definecolor{outcolor}{HTML}{D84315}
    \definecolor{cellborder}{HTML}{CFCFCF}
    \definecolor{cellbackground}{HTML}{F7F7F7}
    
    % prompt
    \makeatletter
    \newcommand{\boxspacing}{\kern\kvtcb@left@rule\kern\kvtcb@boxsep}
    \makeatother
    \newcommand{\prompt}[4]{
        \ttfamily\llap{{\color{#2}[#3]:\hspace{3pt}#4}}\vspace{-\baselineskip}
    }
    

    
    % Prevent overflowing lines due to hard-to-break entities
    \sloppy 
    % Setup hyperref package
    \hypersetup{
      breaklinks=true,  % so long urls are correctly broken across lines
      colorlinks=true,
      urlcolor=urlcolor,
      linkcolor=linkcolor,
      citecolor=citecolor,
      }
    % Slightly bigger margins than the latex defaults
    
    \geometry{verbose,tmargin=1in,bmargin=1in,lmargin=1in,rmargin=1in}
    
    

\begin{document}
    
    \maketitle
    
    

    
    \hypertarget{data-science-research-methods-report-2}{%
\section{Data Science Research Methods
Report-2}\label{data-science-research-methods-report-2}}

    \textbf{Note}: Some processed Dataframes have been stored as pickle
files which will be loaded later on so the rprocessing code need not be
run again and again.

    \hypertarget{extra-libraries-to-install}{%
\section{Extra Libraries to Install}\label{extra-libraries-to-install}}

\begin{enumerate}
\def\labelenumi{\arabic{enumi}.}
\tightlist
\item
  Ipython
\item
  Tabula
\end{enumerate}

    \hypertarget{introduction}{%
\subsection{\texorpdfstring{\textbf{Introduction}}{Introduction }}\label{introduction}}

The PAMAP2 Physical Activity Monitoring dataset (available here)
contains data from 9 participants who participated in 18 various
physical activities (such as walking, cycling, and soccer) while wearing
three inertial measurement units (IMUs) and a heart rate monitor. This
information is saved in separate text files for each subject. The goal
is to build hardware and/or software that can determine the amount and
type of physical activity performed by an individual by using insights
derived from analysing the given dataset.

    \begin{tcolorbox}[breakable, size=fbox, boxrule=1pt, pad at break*=1mm,colback=cellbackground, colframe=cellborder]
\prompt{In}{incolor}{1}{\boxspacing}
\begin{Verbatim}[commandchars=\\\{\}]
\PY{k+kn}{import} \PY{n+nn}{os}
\PY{k+kn}{import} \PY{n+nn}{random}
\PY{k+kn}{from} \PY{n+nn}{collections} \PY{k+kn}{import} \PY{n}{defaultdict}
\end{Verbatim}
\end{tcolorbox}

    \begin{tcolorbox}[breakable, size=fbox, boxrule=1pt, pad at break*=1mm,colback=cellbackground, colframe=cellborder]
\prompt{In}{incolor}{2}{\boxspacing}
\begin{Verbatim}[commandchars=\\\{\}]
\PY{k+kn}{import} \PY{n+nn}{matplotlib}\PY{n+nn}{.}\PY{n+nn}{pyplot} \PY{k}{as} \PY{n+nn}{plt}
\PY{k+kn}{import} \PY{n+nn}{numpy} \PY{k}{as} \PY{n+nn}{np}
\PY{k+kn}{import} \PY{n+nn}{pandas} \PY{k}{as} \PY{n+nn}{pd}
\PY{k+kn}{import} \PY{n+nn}{seaborn} \PY{k}{as} \PY{n+nn}{sns}
\PY{k+kn}{import} \PY{n+nn}{statsmodels}\PY{n+nn}{.}\PY{n+nn}{api} \PY{k}{as} \PY{n+nn}{sm}
\PY{k+kn}{import} \PY{n+nn}{tabula}
\PY{k+kn}{from} \PY{n+nn}{IPython}\PY{n+nn}{.}\PY{n+nn}{display} \PY{k+kn}{import} \PY{n}{display}
\PY{k+kn}{from} \PY{n+nn}{matplotlib} \PY{k+kn}{import} \PY{n}{rcParams}
\PY{k+kn}{from} \PY{n+nn}{numpy}\PY{n+nn}{.}\PY{n+nn}{fft} \PY{k+kn}{import} \PY{n}{rfft}
\PY{k+kn}{from} \PY{n+nn}{scipy}\PY{n+nn}{.}\PY{n+nn}{stats} \PY{k+kn}{import} \PY{n}{ranksums}\PY{p}{,} \PY{n}{ttest\PYZus{}ind}
\PY{k+kn}{from} \PY{n+nn}{sklearn} \PY{k+kn}{import} \PY{n}{cluster}\PY{p}{,} \PY{n}{preprocessing}
\PY{k+kn}{from} \PY{n+nn}{sklearn}\PY{n+nn}{.}\PY{n+nn}{cluster} \PY{k+kn}{import} \PY{n}{KMeans}
\PY{k+kn}{from} \PY{n+nn}{sklearn}\PY{n+nn}{.}\PY{n+nn}{decomposition} \PY{k+kn}{import} \PY{n}{PCA}
\PY{k+kn}{from} \PY{n+nn}{sklearn}\PY{n+nn}{.}\PY{n+nn}{linear\PYZus{}model} \PY{k+kn}{import} \PY{n}{LogisticRegression}
\PY{k+kn}{from} \PY{n+nn}{sklearn}\PY{n+nn}{.}\PY{n+nn}{metrics} \PY{k+kn}{import} \PY{p}{(}\PY{n}{accuracy\PYZus{}score}\PY{p}{,} \PY{n}{classification\PYZus{}report}\PY{p}{,} \PY{n}{f1\PYZus{}score}\PY{p}{,}
                             \PY{n}{log\PYZus{}loss}\PY{p}{,} \PY{n}{v\PYZus{}measure\PYZus{}score}\PY{p}{)}
\end{Verbatim}
\end{tcolorbox}

    \begin{tcolorbox}[breakable, size=fbox, boxrule=1pt, pad at break*=1mm,colback=cellbackground, colframe=cellborder]
\prompt{In}{incolor}{3}{\boxspacing}
\begin{Verbatim}[commandchars=\\\{\}]
\PY{n}{os}\PY{o}{.}\PY{n}{chdir}\PY{p}{(}\PY{l+s+s2}{\PYZdq{}}\PY{l+s+s2}{/home/sahil/Downloads/PAMAP2\PYZus{}Dataset/}\PY{l+s+s2}{\PYZdq{}}\PY{p}{)}  \PY{c+c1}{\PYZsh{} Setting up working directory}
\PY{k+kn}{import} \PY{n+nn}{warnings}
\end{Verbatim}
\end{tcolorbox}

    \begin{tcolorbox}[breakable, size=fbox, boxrule=1pt, pad at break*=1mm,colback=cellbackground, colframe=cellborder]
\prompt{In}{incolor}{46}{\boxspacing}
\begin{Verbatim}[commandchars=\\\{\}]
\PY{n}{pd}\PY{o}{.}\PY{n}{set\PYZus{}option}\PY{p}{(}\PY{l+s+s2}{\PYZdq{}}\PY{l+s+s2}{max\PYZus{}columns}\PY{l+s+s2}{\PYZdq{}}\PY{p}{,} \PY{k+kc}{None}\PY{p}{)} 
\PY{n}{pd}\PY{o}{.}\PY{n}{set\PYZus{}option}\PY{p}{(}\PY{l+s+s2}{\PYZdq{}}\PY{l+s+s2}{max\PYZus{}rows}\PY{l+s+s2}{\PYZdq{}}\PY{p}{,} \PY{k+kc}{None}\PY{p}{)}
\end{Verbatim}
\end{tcolorbox}

    \begin{tcolorbox}[breakable, size=fbox, boxrule=1pt, pad at break*=1mm,colback=cellbackground, colframe=cellborder]
\prompt{In}{incolor}{5}{\boxspacing}
\begin{Verbatim}[commandchars=\\\{\}]
\PY{n}{warnings}\PY{o}{.}\PY{n}{filterwarnings}\PY{p}{(}\PY{l+s+s2}{\PYZdq{}}\PY{l+s+s2}{ignore}\PY{l+s+s2}{\PYZdq{}}\PY{p}{)}
\end{Verbatim}
\end{tcolorbox}

    \hypertarget{data-cleaning}{%
\subsection{\texorpdfstring{Data
Cleaning}{Data Cleaning }}\label{data-cleaning}}

For tidying up the data : - The data of various subjects is loaded and
given relevant column names for various features. - The data for all
subjects are then stacked together to form one table. - `Orientation'
columns are removed because it was mentioned in the data report that it
is invalid in this data collection. - The accelerometer of sensitivity
6g is also removed as it's not completely accurate for all activities
due to its low sensitivity. - Similarly, the rows with Activity ID ``0''
are also removed as it does not relate to any specific activity. - The
missing values are filled up using the forward fill method.In this
method, the blank values are filled with the value occuring just before
it. - Added a new feature, `BMI' or Body Mass Index for the
`subject\_detail' table - Additional feature, `Activity Type' is added
to the data which classifies activities into 3 classes, `Light'
activity,`Moderate' activity and `Intense' activity. 1.
Lying,sitting,ironing and standing are labelled as `light' activities.
2. Vacuum cleaning,descending stairs,normal walking,Nordic walking and
cycling are considered as `Moderate' activities 3. Ascending
stairs,running and rope jumping are labelled as `Intense' activities.
This classification makes it easier to perform hypothesis testing
between pair of attributes. - Subject 109 is not considered for analysis
as it has performed few protocol activities. - Optional activities are
ignored since very few subjects have performed them which makes it
difficult to model.

    Given below are functions to give relevant names to the columns and
create a single table containing data for all subjects

    \begin{tcolorbox}[breakable, size=fbox, boxrule=1pt, pad at break*=1mm,colback=cellbackground, colframe=cellborder]
\prompt{In}{incolor}{91}{\boxspacing}
\begin{Verbatim}[commandchars=\\\{\}]
\PY{k}{def} \PY{n+nf}{gen\PYZus{}activity\PYZus{}names}\PY{p}{(}\PY{p}{)}\PY{p}{:}
    \PY{c+c1}{\PYZsh{} Using this function all the activity names are mapped to their ids}
    \PY{n}{act\PYZus{}name} \PY{o}{=} \PY{p}{\PYZob{}}\PY{p}{\PYZcb{}}
    \PY{n}{act\PYZus{}name}\PY{p}{[}\PY{l+m+mi}{0}\PY{p}{]} \PY{o}{=} \PY{l+s+s2}{\PYZdq{}}\PY{l+s+s2}{transient}\PY{l+s+s2}{\PYZdq{}}
    \PY{n}{act\PYZus{}name}\PY{p}{[}\PY{l+m+mi}{1}\PY{p}{]} \PY{o}{=} \PY{l+s+s2}{\PYZdq{}}\PY{l+s+s2}{lying}\PY{l+s+s2}{\PYZdq{}}
    \PY{n}{act\PYZus{}name}\PY{p}{[}\PY{l+m+mi}{2}\PY{p}{]} \PY{o}{=} \PY{l+s+s2}{\PYZdq{}}\PY{l+s+s2}{sitting}\PY{l+s+s2}{\PYZdq{}}
    \PY{n}{act\PYZus{}name}\PY{p}{[}\PY{l+m+mi}{3}\PY{p}{]} \PY{o}{=} \PY{l+s+s2}{\PYZdq{}}\PY{l+s+s2}{standing}\PY{l+s+s2}{\PYZdq{}}
    \PY{n}{act\PYZus{}name}\PY{p}{[}\PY{l+m+mi}{4}\PY{p}{]} \PY{o}{=} \PY{l+s+s2}{\PYZdq{}}\PY{l+s+s2}{walking}\PY{l+s+s2}{\PYZdq{}}
    \PY{n}{act\PYZus{}name}\PY{p}{[}\PY{l+m+mi}{5}\PY{p}{]} \PY{o}{=} \PY{l+s+s2}{\PYZdq{}}\PY{l+s+s2}{running}\PY{l+s+s2}{\PYZdq{}}
    \PY{n}{act\PYZus{}name}\PY{p}{[}\PY{l+m+mi}{6}\PY{p}{]} \PY{o}{=} \PY{l+s+s2}{\PYZdq{}}\PY{l+s+s2}{cycling}\PY{l+s+s2}{\PYZdq{}}
    \PY{n}{act\PYZus{}name}\PY{p}{[}\PY{l+m+mi}{7}\PY{p}{]} \PY{o}{=} \PY{l+s+s2}{\PYZdq{}}\PY{l+s+s2}{Nordic\PYZus{}walking}\PY{l+s+s2}{\PYZdq{}}
    \PY{n}{act\PYZus{}name}\PY{p}{[}\PY{l+m+mi}{9}\PY{p}{]} \PY{o}{=} \PY{l+s+s2}{\PYZdq{}}\PY{l+s+s2}{watching\PYZus{}TV}\PY{l+s+s2}{\PYZdq{}}
    \PY{n}{act\PYZus{}name}\PY{p}{[}\PY{l+m+mi}{10}\PY{p}{]} \PY{o}{=} \PY{l+s+s2}{\PYZdq{}}\PY{l+s+s2}{computer\PYZus{}work}\PY{l+s+s2}{\PYZdq{}}
    \PY{n}{act\PYZus{}name}\PY{p}{[}\PY{l+m+mi}{11}\PY{p}{]} \PY{o}{=} \PY{l+s+s2}{\PYZdq{}}\PY{l+s+s2}{car driving}\PY{l+s+s2}{\PYZdq{}}
    \PY{n}{act\PYZus{}name}\PY{p}{[}\PY{l+m+mi}{12}\PY{p}{]} \PY{o}{=} \PY{l+s+s2}{\PYZdq{}}\PY{l+s+s2}{ascending\PYZus{}stairs}\PY{l+s+s2}{\PYZdq{}}
    \PY{n}{act\PYZus{}name}\PY{p}{[}\PY{l+m+mi}{13}\PY{p}{]} \PY{o}{=} \PY{l+s+s2}{\PYZdq{}}\PY{l+s+s2}{descending\PYZus{}stairs}\PY{l+s+s2}{\PYZdq{}}
    \PY{n}{act\PYZus{}name}\PY{p}{[}\PY{l+m+mi}{16}\PY{p}{]} \PY{o}{=} \PY{l+s+s2}{\PYZdq{}}\PY{l+s+s2}{vacuum\PYZus{}cleaning}\PY{l+s+s2}{\PYZdq{}}
    \PY{n}{act\PYZus{}name}\PY{p}{[}\PY{l+m+mi}{17}\PY{p}{]} \PY{o}{=} \PY{l+s+s2}{\PYZdq{}}\PY{l+s+s2}{ironing}\PY{l+s+s2}{\PYZdq{}}
    \PY{n}{act\PYZus{}name}\PY{p}{[}\PY{l+m+mi}{18}\PY{p}{]} \PY{o}{=} \PY{l+s+s2}{\PYZdq{}}\PY{l+s+s2}{folding\PYZus{}laundry}\PY{l+s+s2}{\PYZdq{}}
    \PY{n}{act\PYZus{}name}\PY{p}{[}\PY{l+m+mi}{19}\PY{p}{]} \PY{o}{=} \PY{l+s+s2}{\PYZdq{}}\PY{l+s+s2}{house\PYZus{}cleaning}\PY{l+s+s2}{\PYZdq{}}
    \PY{n}{act\PYZus{}name}\PY{p}{[}\PY{l+m+mi}{20}\PY{p}{]} \PY{o}{=} \PY{l+s+s2}{\PYZdq{}}\PY{l+s+s2}{playing\PYZus{}soccer}\PY{l+s+s2}{\PYZdq{}}
    \PY{n}{act\PYZus{}name}\PY{p}{[}\PY{l+m+mi}{24}\PY{p}{]} \PY{o}{=} \PY{l+s+s2}{\PYZdq{}}\PY{l+s+s2}{rope\PYZus{}jumping}\PY{l+s+s2}{\PYZdq{}}
    \PY{k}{return} \PY{n}{act\PYZus{}name}
\end{Verbatim}
\end{tcolorbox}

    \begin{tcolorbox}[breakable, size=fbox, boxrule=1pt, pad at break*=1mm,colback=cellbackground, colframe=cellborder]
\prompt{In}{incolor}{92}{\boxspacing}
\begin{Verbatim}[commandchars=\\\{\}]
\PY{k}{def} \PY{n+nf}{generate\PYZus{}three\PYZus{}IMU}\PY{p}{(}\PY{n}{name}\PY{p}{)}\PY{p}{:}
    \PY{c+c1}{\PYZsh{} Adding coordinate suffix for accelerometer}
    \PY{n}{x} \PY{o}{=} \PY{n}{name} \PY{o}{+} \PY{l+s+s2}{\PYZdq{}}\PY{l+s+s2}{\PYZus{}x}\PY{l+s+s2}{\PYZdq{}}
    \PY{n}{y} \PY{o}{=} \PY{n}{name} \PY{o}{+} \PY{l+s+s2}{\PYZdq{}}\PY{l+s+s2}{\PYZus{}y}\PY{l+s+s2}{\PYZdq{}}
    \PY{n}{z} \PY{o}{=} \PY{n}{name} \PY{o}{+} \PY{l+s+s2}{\PYZdq{}}\PY{l+s+s2}{\PYZus{}z}\PY{l+s+s2}{\PYZdq{}}
    \PY{k}{return} \PY{p}{[}\PY{n}{x}\PY{p}{,} \PY{n}{y}\PY{p}{,} \PY{n}{z}\PY{p}{]}
\end{Verbatim}
\end{tcolorbox}

    \begin{tcolorbox}[breakable, size=fbox, boxrule=1pt, pad at break*=1mm,colback=cellbackground, colframe=cellborder]
\prompt{In}{incolor}{93}{\boxspacing}
\begin{Verbatim}[commandchars=\\\{\}]
\PY{k}{def} \PY{n+nf}{generate\PYZus{}four\PYZus{}IMU}\PY{p}{(}\PY{n}{name}\PY{p}{)}\PY{p}{:}
    \PY{c+c1}{\PYZsh{} Adding coordinates suffixes for orientation}
    \PY{n}{x} \PY{o}{=} \PY{n}{name} \PY{o}{+} \PY{l+s+s2}{\PYZdq{}}\PY{l+s+s2}{\PYZus{}x}\PY{l+s+s2}{\PYZdq{}}
    \PY{n}{y} \PY{o}{=} \PY{n}{name} \PY{o}{+} \PY{l+s+s2}{\PYZdq{}}\PY{l+s+s2}{\PYZus{}y}\PY{l+s+s2}{\PYZdq{}}
    \PY{n}{z} \PY{o}{=} \PY{n}{name} \PY{o}{+} \PY{l+s+s2}{\PYZdq{}}\PY{l+s+s2}{\PYZus{}z}\PY{l+s+s2}{\PYZdq{}}
    \PY{n}{w} \PY{o}{=} \PY{n}{name} \PY{o}{+} \PY{l+s+s2}{\PYZdq{}}\PY{l+s+s2}{\PYZus{}w}\PY{l+s+s2}{\PYZdq{}}
    \PY{k}{return} \PY{p}{[}\PY{n}{x}\PY{p}{,} \PY{n}{y}\PY{p}{,} \PY{n}{z}\PY{p}{,} \PY{n}{w}\PY{p}{]}
\end{Verbatim}
\end{tcolorbox}

    \begin{tcolorbox}[breakable, size=fbox, boxrule=1pt, pad at break*=1mm,colback=cellbackground, colframe=cellborder]
\prompt{In}{incolor}{94}{\boxspacing}
\begin{Verbatim}[commandchars=\\\{\}]
\PY{k}{def} \PY{n+nf}{generate\PYZus{}cols\PYZus{}IMU}\PY{p}{(}\PY{n}{name}\PY{p}{)}\PY{p}{:}
    \PY{c+c1}{\PYZsh{} temperature column names}
    \PY{n}{temp} \PY{o}{=} \PY{n}{name} \PY{o}{+} \PY{l+s+s2}{\PYZdq{}}\PY{l+s+s2}{\PYZus{}temperature}\PY{l+s+s2}{\PYZdq{}}
    \PY{n}{output} \PY{o}{=} \PY{p}{[}\PY{n}{temp}\PY{p}{]}
    \PY{c+c1}{\PYZsh{} acceleration 16g columns names}
    \PY{n}{acceleration16} \PY{o}{=} \PY{n}{name} \PY{o}{+} \PY{l+s+s2}{\PYZdq{}}\PY{l+s+s2}{\PYZus{}3D\PYZus{}acceleration\PYZus{}16}\PY{l+s+s2}{\PYZdq{}}
    \PY{n}{acceleration16} \PY{o}{=} \PY{n}{generate\PYZus{}three\PYZus{}IMU}\PY{p}{(}\PY{n}{acceleration16}\PY{p}{)}
    \PY{n}{output}\PY{o}{.}\PY{n}{extend}\PY{p}{(}\PY{n}{acceleration16}\PY{p}{)}
    \PY{c+c1}{\PYZsh{} acceleration 6g column anmes}
    \PY{n}{acceleration6} \PY{o}{=} \PY{n}{name} \PY{o}{+} \PY{l+s+s2}{\PYZdq{}}\PY{l+s+s2}{\PYZus{}3D\PYZus{}acceleration\PYZus{}6}\PY{l+s+s2}{\PYZdq{}}
    \PY{n}{acceleration6} \PY{o}{=} \PY{n}{generate\PYZus{}three\PYZus{}IMU}\PY{p}{(}\PY{n}{acceleration6}\PY{p}{)}
    \PY{n}{output}\PY{o}{.}\PY{n}{extend}\PY{p}{(}\PY{n}{acceleration6}\PY{p}{)}
    \PY{c+c1}{\PYZsh{} gyroscope column names}
    \PY{n}{gyroscope} \PY{o}{=} \PY{n}{name} \PY{o}{+} \PY{l+s+s2}{\PYZdq{}}\PY{l+s+s2}{\PYZus{}3D\PYZus{}gyroscope}\PY{l+s+s2}{\PYZdq{}}
    \PY{n}{gyroscope} \PY{o}{=} \PY{n}{generate\PYZus{}three\PYZus{}IMU}\PY{p}{(}\PY{n}{gyroscope}\PY{p}{)}
    \PY{n}{output}\PY{o}{.}\PY{n}{extend}\PY{p}{(}\PY{n}{gyroscope}\PY{p}{)}
    \PY{c+c1}{\PYZsh{} magnometer column names}
    \PY{n}{magnometer} \PY{o}{=} \PY{n}{name} \PY{o}{+} \PY{l+s+s2}{\PYZdq{}}\PY{l+s+s2}{\PYZus{}3D\PYZus{}magnetometer}\PY{l+s+s2}{\PYZdq{}}
    \PY{n}{magnometer} \PY{o}{=} \PY{n}{generate\PYZus{}three\PYZus{}IMU}\PY{p}{(}\PY{n}{magnometer}\PY{p}{)}
    \PY{n}{output}\PY{o}{.}\PY{n}{extend}\PY{p}{(}\PY{n}{magnometer}\PY{p}{)}
    \PY{c+c1}{\PYZsh{} oreintation column names}
    \PY{n}{oreintation} \PY{o}{=} \PY{n}{name} \PY{o}{+} \PY{l+s+s2}{\PYZdq{}}\PY{l+s+s2}{\PYZus{}4D\PYZus{}orientation}\PY{l+s+s2}{\PYZdq{}}
    \PY{n}{oreintation} \PY{o}{=} \PY{n}{generate\PYZus{}four\PYZus{}IMU}\PY{p}{(}\PY{n}{oreintation}\PY{p}{)}
    \PY{n}{output}\PY{o}{.}\PY{n}{extend}\PY{p}{(}\PY{n}{oreintation}\PY{p}{)}
    \PY{k}{return} \PY{n}{output}
\end{Verbatim}
\end{tcolorbox}

    \begin{tcolorbox}[breakable, size=fbox, boxrule=1pt, pad at break*=1mm,colback=cellbackground, colframe=cellborder]
\prompt{In}{incolor}{95}{\boxspacing}
\begin{Verbatim}[commandchars=\\\{\}]
\PY{k}{def} \PY{n+nf}{load\PYZus{}IMU}\PY{p}{(}\PY{p}{)}\PY{p}{:}
    \PY{c+c1}{\PYZsh{} Function for generating final column names}
    \PY{n}{output} \PY{o}{=} \PY{p}{[}\PY{l+s+s2}{\PYZdq{}}\PY{l+s+s2}{time\PYZus{}stamp}\PY{l+s+s2}{\PYZdq{}}\PY{p}{,} \PY{l+s+s2}{\PYZdq{}}\PY{l+s+s2}{activity\PYZus{}id}\PY{l+s+s2}{\PYZdq{}}\PY{p}{,} \PY{l+s+s2}{\PYZdq{}}\PY{l+s+s2}{heart\PYZus{}rate}\PY{l+s+s2}{\PYZdq{}}\PY{p}{]}
    \PY{n}{hand} \PY{o}{=} \PY{l+s+s2}{\PYZdq{}}\PY{l+s+s2}{hand}\PY{l+s+s2}{\PYZdq{}}
    \PY{n}{hand} \PY{o}{=} \PY{n}{generate\PYZus{}cols\PYZus{}IMU}\PY{p}{(}\PY{n}{hand}\PY{p}{)}
    \PY{n}{output}\PY{o}{.}\PY{n}{extend}\PY{p}{(}\PY{n}{hand}\PY{p}{)}
    \PY{n}{chest} \PY{o}{=} \PY{l+s+s2}{\PYZdq{}}\PY{l+s+s2}{chest}\PY{l+s+s2}{\PYZdq{}}
    \PY{n}{chest} \PY{o}{=} \PY{n}{generate\PYZus{}cols\PYZus{}IMU}\PY{p}{(}\PY{n}{chest}\PY{p}{)}
    \PY{n}{output}\PY{o}{.}\PY{n}{extend}\PY{p}{(}\PY{n}{chest}\PY{p}{)}
    \PY{n}{ankle} \PY{o}{=} \PY{l+s+s2}{\PYZdq{}}\PY{l+s+s2}{ankle}\PY{l+s+s2}{\PYZdq{}}
    \PY{n}{ankle} \PY{o}{=} \PY{n}{generate\PYZus{}cols\PYZus{}IMU}\PY{p}{(}\PY{n}{ankle}\PY{p}{)}
    \PY{n}{output}\PY{o}{.}\PY{n}{extend}\PY{p}{(}\PY{n}{ankle}\PY{p}{)}
    \PY{k}{return} \PY{n}{output}
\end{Verbatim}
\end{tcolorbox}

    \begin{tcolorbox}[breakable, size=fbox, boxrule=1pt, pad at break*=1mm,colback=cellbackground, colframe=cellborder]
\prompt{In}{incolor}{ }{\boxspacing}
\begin{Verbatim}[commandchars=\\\{\}]
\PY{k}{def} \PY{n+nf}{load\PYZus{}subjects}\PY{p}{(}
    \PY{n}{root1}\PY{o}{=}\PY{l+s+s2}{\PYZdq{}}\PY{l+s+s2}{/home/sahil/Downloads/PAMAP2\PYZus{}Dataset/Protocol/subject}\PY{l+s+s2}{\PYZdq{}}\PY{p}{,}
    \PY{n}{root2}\PY{o}{=}\PY{l+s+s2}{\PYZdq{}}\PY{l+s+s2}{/home/sahil/Downloads/PAMAP2\PYZus{}Dataset/Optional/subject}\PY{l+s+s2}{\PYZdq{}}\PY{p}{,}
\PY{p}{)}\PY{p}{:}  
    \PY{c+c1}{\PYZsh{} This function loads data from subject files and names the columns}
    \PY{n}{cols} \PY{o}{=} \PY{n}{load\PYZus{}IMU}\PY{p}{(}\PY{p}{)}
    \PY{n}{output} \PY{o}{=} \PY{n}{pd}\PY{o}{.}\PY{n}{DataFrame}\PY{p}{(}\PY{p}{)}
    \PY{k}{for} \PY{n}{i} \PY{o+ow}{in} \PY{n+nb}{range}\PY{p}{(}\PY{l+m+mi}{101}\PY{p}{,} \PY{l+m+mi}{110}\PY{p}{)}\PY{p}{:}
        \PY{n}{path1} \PY{o}{=} \PY{n}{root1} \PY{o}{+} \PY{n+nb}{str}\PY{p}{(}\PY{n}{i}\PY{p}{)} \PY{o}{+} \PY{l+s+s2}{\PYZdq{}}\PY{l+s+s2}{.dat}\PY{l+s+s2}{\PYZdq{}}
        \PY{n}{subject} \PY{o}{=} \PY{n}{pd}\PY{o}{.}\PY{n}{DataFrame}\PY{p}{(}\PY{p}{)}
        \PY{n}{subject\PYZus{}prot} \PY{o}{=} \PY{n}{pd}\PY{o}{.}\PY{n}{read\PYZus{}table}\PY{p}{(}\PY{n}{path1}\PY{p}{,} \PY{n}{header}\PY{o}{=}\PY{k+kc}{None}\PY{p}{,} \PY{n}{sep}\PY{o}{=}\PY{l+s+s2}{\PYZdq{}}\PY{l+s+s2}{\PYZbs{}}\PY{l+s+s2}{s+}\PY{l+s+s2}{\PYZdq{}}\PY{p}{)}  \PY{c+c1}{\PYZsh{} subject data from}
        \PY{c+c1}{\PYZsh{} protocol activities}
        \PY{n}{subject} \PY{o}{=} \PY{n}{subject}\PY{o}{.}\PY{n}{append}\PY{p}{(}\PY{n}{subject\PYZus{}prot}\PY{p}{)}
        \PY{n}{subject}\PY{o}{.}\PY{n}{columns} \PY{o}{=} \PY{n}{cols}
        \PY{n}{subject} \PY{o}{=} \PY{n}{subject}\PY{o}{.}\PY{n}{sort\PYZus{}values}\PY{p}{(}
            \PY{n}{by}\PY{o}{=}\PY{l+s+s2}{\PYZdq{}}\PY{l+s+s2}{time\PYZus{}stamp}\PY{l+s+s2}{\PYZdq{}}
        \PY{p}{)}  \PY{c+c1}{\PYZsh{} Arranging all measurements according to}
        \PY{c+c1}{\PYZsh{} time}
        \PY{n}{subject}\PY{p}{[}\PY{l+s+s2}{\PYZdq{}}\PY{l+s+s2}{id}\PY{l+s+s2}{\PYZdq{}}\PY{p}{]} \PY{o}{=} \PY{n}{i}
        \PY{n}{output} \PY{o}{=} \PY{n}{output}\PY{o}{.}\PY{n}{append}\PY{p}{(}\PY{n}{subject}\PY{p}{,} \PY{n}{ignore\PYZus{}index}\PY{o}{=}\PY{k+kc}{True}\PY{p}{)}
    \PY{k}{return} \PY{n}{output}
\end{Verbatim}
\end{tcolorbox}

    Feel free to skip the execution of this cell as the output is saved and
reloaded in the cells below it.This is done because the eprocessing
takes a lot of time and so it was found more appropriate to save the
output so re-running the cell is not required

    \begin{tcolorbox}[breakable, size=fbox, boxrule=1pt, pad at break*=1mm,colback=cellbackground, colframe=cellborder]
\prompt{In}{incolor}{ }{\boxspacing}
\begin{Verbatim}[commandchars=\\\{\}]
\PY{n}{data} \PY{o}{=} \PY{n}{load\PYZus{}subjects}\PY{p}{(}\PY{p}{)}  \PY{c+c1}{\PYZsh{} Add your own location for the data here to replicate the code}
\PY{c+c1}{\PYZsh{} for eg data = load\PYZus{}subjects(\PYZsq{}filepath\PYZsq{})}
\PY{n}{data} \PY{o}{=} \PY{n}{data}\PY{o}{.}\PY{n}{drop}\PY{p}{(}
    \PY{n}{data}\PY{p}{[}\PY{n}{data}\PY{p}{[}\PY{l+s+s2}{\PYZdq{}}\PY{l+s+s2}{activity\PYZus{}id}\PY{l+s+s2}{\PYZdq{}}\PY{p}{]} \PY{o}{==} \PY{l+m+mi}{0}\PY{p}{]}\PY{o}{.}\PY{n}{index}
\PY{p}{)}  \PY{c+c1}{\PYZsh{} Removing rows with activity id of 0}
\PY{n}{act} \PY{o}{=} \PY{n}{gen\PYZus{}activity\PYZus{}names}\PY{p}{(}\PY{p}{)}
\PY{n}{data}\PY{p}{[}\PY{l+s+s2}{\PYZdq{}}\PY{l+s+s2}{activity\PYZus{}name}\PY{l+s+s2}{\PYZdq{}}\PY{p}{]} \PY{o}{=} \PY{n}{data}\PY{o}{.}\PY{n}{activity\PYZus{}id}\PY{o}{.}\PY{n}{apply}\PY{p}{(}\PY{k}{lambda} \PY{n}{x}\PY{p}{:} \PY{n}{act}\PY{p}{[}\PY{n}{x}\PY{p}{]}\PY{p}{)}
\PY{n}{data} \PY{o}{=} \PY{n}{data}\PY{o}{.}\PY{n}{drop}\PY{p}{(}
    \PY{p}{[}\PY{n}{i} \PY{k}{for} \PY{n}{i} \PY{o+ow}{in} \PY{n}{data}\PY{o}{.}\PY{n}{columns} \PY{k}{if} \PY{l+s+s2}{\PYZdq{}}\PY{l+s+s2}{orientation}\PY{l+s+s2}{\PYZdq{}} \PY{o+ow}{in} \PY{n}{i}\PY{p}{]}\PY{p}{,} \PY{n}{axis}\PY{o}{=}\PY{l+m+mi}{1}
\PY{p}{)}  \PY{c+c1}{\PYZsh{} Dropping Orientation  columns}
\PY{n}{cols\PYZus{}6g} \PY{o}{=} \PY{p}{[}\PY{n}{i} \PY{k}{for} \PY{n}{i} \PY{o+ow}{in} \PY{n}{data}\PY{o}{.}\PY{n}{columns} \PY{k}{if} \PY{l+s+s2}{\PYZdq{}}\PY{l+s+s2}{\PYZus{}6\PYZus{}}\PY{l+s+s2}{\PYZdq{}} \PY{o+ow}{in} \PY{n}{i}\PY{p}{]}  \PY{c+c1}{\PYZsh{} 6g acceleration data columns}
\PY{n}{data} \PY{o}{=} \PY{n}{data}\PY{o}{.}\PY{n}{drop}\PY{p}{(}\PY{n}{cols\PYZus{}6g}\PY{p}{,} \PY{n}{axis}\PY{o}{=}\PY{l+m+mi}{1}\PY{p}{)}  \PY{c+c1}{\PYZsh{} dropping 6g acceleration columns}
\PY{n}{display}\PY{p}{(}\PY{n}{data}\PY{o}{.}\PY{n}{head}\PY{p}{(}\PY{p}{)}\PY{p}{)}
\PY{c+c1}{\PYZsh{} Saving transformed data in pickle format becuse it has the fastest read time compared}
\PY{c+c1}{\PYZsh{} to all other formats}
\PY{n}{data}\PY{o}{.}\PY{n}{to\PYZus{}pickle}\PY{p}{(}\PY{l+s+s2}{\PYZdq{}}\PY{l+s+s2}{activity\PYZus{}data.pkl}\PY{l+s+s2}{\PYZdq{}}\PY{p}{)}  \PY{c+c1}{\PYZsh{} Saving transformed data for future use}
\end{Verbatim}
\end{tcolorbox}

    \begin{tcolorbox}[breakable, size=fbox, boxrule=1pt, pad at break*=1mm,colback=cellbackground, colframe=cellborder]
\prompt{In}{incolor}{97}{\boxspacing}
\begin{Verbatim}[commandchars=\\\{\}]
\PY{k}{def} \PY{n+nf}{train\PYZus{}test\PYZus{}split\PYZus{}by\PYZus{}subjects}\PY{p}{(}\PY{n}{data}\PY{p}{)}\PY{p}{:}  \PY{c+c1}{\PYZsh{} splitting by subjects}
    \PY{n}{subjects} \PY{o}{=} \PY{p}{[}
        \PY{n}{i} \PY{k}{for} \PY{n}{i} \PY{o+ow}{in} \PY{n+nb}{range}\PY{p}{(}\PY{l+m+mi}{101}\PY{p}{,} \PY{l+m+mi}{109}\PY{p}{)}
    \PY{p}{]}  \PY{c+c1}{\PYZsh{} Eliminate subject 109  due to less activities}
    \PY{n}{train\PYZus{}subjects} \PY{o}{=} \PY{p}{[}\PY{l+m+mi}{101}\PY{p}{,} \PY{l+m+mi}{103}\PY{p}{,} \PY{l+m+mi}{104}\PY{p}{,} \PY{l+m+mi}{105}\PY{p}{]}
    \PY{n}{test\PYZus{}subjects} \PY{o}{=} \PY{p}{[}\PY{n}{i} \PY{k}{for} \PY{n}{i} \PY{o+ow}{in} \PY{n}{subjects} \PY{k}{if} \PY{n}{i} \PY{o+ow}{not} \PY{o+ow}{in} \PY{n}{train\PYZus{}subjects}\PY{p}{]}
    \PY{n}{train} \PY{o}{=} \PY{n}{data}\PY{p}{[}\PY{n}{data}\PY{o}{.}\PY{n}{id}\PY{o}{.}\PY{n}{isin}\PY{p}{(}\PY{n}{train\PYZus{}subjects}\PY{p}{)}\PY{p}{]}  \PY{c+c1}{\PYZsh{} Generating training data}
    \PY{n}{test} \PY{o}{=} \PY{n}{data}\PY{p}{[}\PY{n}{data}\PY{o}{.}\PY{n}{id}\PY{o}{.}\PY{n}{isin}\PY{p}{(}\PY{n}{test\PYZus{}subjects}\PY{p}{)}\PY{p}{]}  \PY{c+c1}{\PYZsh{} generating testing data}
    \PY{k}{return} \PY{n}{train}\PY{p}{,} \PY{n}{test}
\end{Verbatim}
\end{tcolorbox}

    \begin{tcolorbox}[breakable, size=fbox, boxrule=1pt, pad at break*=1mm,colback=cellbackground, colframe=cellborder]
\prompt{In}{incolor}{98}{\boxspacing}
\begin{Verbatim}[commandchars=\\\{\}]
\PY{k}{def} \PY{n+nf}{split\PYZus{}by\PYZus{}activities}\PY{p}{(}\PY{n}{data}\PY{p}{)}\PY{p}{:}
    \PY{n}{light} \PY{o}{=} \PY{p}{[}\PY{l+s+s2}{\PYZdq{}}\PY{l+s+s2}{lying}\PY{l+s+s2}{\PYZdq{}}\PY{p}{,} \PY{l+s+s2}{\PYZdq{}}\PY{l+s+s2}{sitting}\PY{l+s+s2}{\PYZdq{}}\PY{p}{,} \PY{l+s+s2}{\PYZdq{}}\PY{l+s+s2}{standing}\PY{l+s+s2}{\PYZdq{}}\PY{p}{,} \PY{l+s+s2}{\PYZdq{}}\PY{l+s+s2}{ironing}\PY{l+s+s2}{\PYZdq{}}\PY{p}{]}
    \PY{n}{moderate} \PY{o}{=} \PY{p}{[}
        \PY{l+s+s2}{\PYZdq{}}\PY{l+s+s2}{vacuum\PYZus{}cleaning}\PY{l+s+s2}{\PYZdq{}}\PY{p}{,}
        \PY{l+s+s2}{\PYZdq{}}\PY{l+s+s2}{descending\PYZus{}stairs}\PY{l+s+s2}{\PYZdq{}}\PY{p}{,}
        \PY{l+s+s2}{\PYZdq{}}\PY{l+s+s2}{normal\PYZus{}walking}\PY{l+s+s2}{\PYZdq{}}\PY{p}{,}
        \PY{l+s+s2}{\PYZdq{}}\PY{l+s+s2}{nordic\PYZus{}walking}\PY{l+s+s2}{\PYZdq{}}\PY{p}{,}
        \PY{l+s+s2}{\PYZdq{}}\PY{l+s+s2}{cycling}\PY{l+s+s2}{\PYZdq{}}\PY{p}{,}
    \PY{p}{]}
    \PY{n}{intense} \PY{o}{=} \PY{p}{[}\PY{l+s+s2}{\PYZdq{}}\PY{l+s+s2}{ascending\PYZus{}stairs}\PY{l+s+s2}{\PYZdq{}}\PY{p}{,} \PY{l+s+s2}{\PYZdq{}}\PY{l+s+s2}{running}\PY{l+s+s2}{\PYZdq{}}\PY{p}{,} \PY{l+s+s2}{\PYZdq{}}\PY{l+s+s2}{rope\PYZus{}jumping}\PY{l+s+s2}{\PYZdq{}}\PY{p}{]}
    \PY{k}{def} \PY{n+nf}{split}\PY{p}{(}\PY{n}{activity}\PY{p}{)}\PY{p}{:}  \PY{c+c1}{\PYZsh{}  method for returning activity labels for activities}
        \PY{k}{if} \PY{n}{activity} \PY{o+ow}{in} \PY{n}{light}\PY{p}{:}
            \PY{k}{return} \PY{l+s+s2}{\PYZdq{}}\PY{l+s+s2}{light}\PY{l+s+s2}{\PYZdq{}}
        \PY{k}{elif} \PY{n}{activity} \PY{o+ow}{in} \PY{n}{moderate}\PY{p}{:}
            \PY{k}{return} \PY{l+s+s2}{\PYZdq{}}\PY{l+s+s2}{moderate}\PY{l+s+s2}{\PYZdq{}}
        \PY{k}{else}\PY{p}{:}
            \PY{k}{return} \PY{l+s+s2}{\PYZdq{}}\PY{l+s+s2}{intense}\PY{l+s+s2}{\PYZdq{}}
    \PY{n}{data}\PY{p}{[}\PY{l+s+s2}{\PYZdq{}}\PY{l+s+s2}{activity\PYZus{}type}\PY{l+s+s2}{\PYZdq{}}\PY{p}{]} \PY{o}{=} \PY{n}{data}\PY{o}{.}\PY{n}{activity\PYZus{}name}\PY{o}{.}\PY{n}{apply}\PY{p}{(}\PY{k}{lambda} \PY{n}{x}\PY{p}{:} \PY{n}{split}\PY{p}{(}\PY{n}{x}\PY{p}{)}\PY{p}{)}
    \PY{k}{return} \PY{n}{data}
\end{Verbatim}
\end{tcolorbox}

    Loading data and doing the train-test split for EDA and Hypothesis
testing.

    \begin{tcolorbox}[breakable, size=fbox, boxrule=1pt, pad at break*=1mm,colback=cellbackground, colframe=cellborder]
\prompt{In}{incolor}{99}{\boxspacing}
\begin{Verbatim}[commandchars=\\\{\}]
\PY{n}{data} \PY{o}{=} \PY{n}{pd}\PY{o}{.}\PY{n}{read\PYZus{}pickle}\PY{p}{(}\PY{l+s+s2}{\PYZdq{}}\PY{l+s+s2}{activity\PYZus{}data.pkl}\PY{l+s+s2}{\PYZdq{}}\PY{p}{)}
\PY{n}{data} \PY{o}{=} \PY{n}{split\PYZus{}by\PYZus{}activities}\PY{p}{(}\PY{n}{data}\PY{p}{)}
\PY{n}{train}\PY{p}{,} \PY{n}{test} \PY{o}{=} \PY{n}{train\PYZus{}test\PYZus{}split\PYZus{}by\PYZus{}subjects}\PY{p}{(}
    \PY{n}{data}
\PY{p}{)}  \PY{c+c1}{\PYZsh{} train and test data for EDA and hypothesis testing respectively.}
\PY{n}{subj\PYZus{}det} \PY{o}{=} \PY{n}{tabula}\PY{o}{.}\PY{n}{read\PYZus{}pdf}\PY{p}{(}
    \PY{l+s+s2}{\PYZdq{}}\PY{l+s+s2}{subjectInformation.pdf}\PY{l+s+s2}{\PYZdq{}}\PY{p}{,} \PY{n}{pages}\PY{o}{=}\PY{l+m+mi}{1}
\PY{p}{)}  \PY{c+c1}{\PYZsh{} loading subject detail table from pdf file.}
\PY{c+c1}{\PYZsh{} Eliminating unnecessary columns and fixing the column alignment of the table.}
\PY{n}{sd} \PY{o}{=} \PY{n}{subj\PYZus{}det}\PY{p}{[}\PY{l+m+mi}{0}\PY{p}{]}
\PY{n}{new\PYZus{}cols} \PY{o}{=} \PY{n+nb}{list}\PY{p}{(}\PY{n}{sd}\PY{o}{.}\PY{n}{columns}\PY{p}{)}\PY{p}{[}\PY{l+m+mi}{1}\PY{p}{:}\PY{l+m+mi}{9}\PY{p}{]}
\PY{n}{sd} \PY{o}{=} \PY{n}{sd}\PY{p}{[}\PY{n}{sd}\PY{o}{.}\PY{n}{columns}\PY{p}{[}\PY{l+m+mi}{0}\PY{p}{:}\PY{l+m+mi}{8}\PY{p}{]}\PY{p}{]}
\PY{n}{sd}\PY{o}{.}\PY{n}{columns} \PY{o}{=} \PY{n}{new\PYZus{}cols}
\PY{n}{subj\PYZus{}det} \PY{o}{=} \PY{n}{sd}
\end{Verbatim}
\end{tcolorbox}

    Create clean data for use in modelling

    \begin{tcolorbox}[breakable, size=fbox, boxrule=1pt, pad at break*=1mm,colback=cellbackground, colframe=cellborder]
\prompt{In}{incolor}{100}{\boxspacing}
\begin{Verbatim}[commandchars=\\\{\}]
\PY{n}{eliminate} \PY{o}{=} \PY{p}{[}
    \PY{l+s+s2}{\PYZdq{}}\PY{l+s+s2}{activity\PYZus{}id}\PY{l+s+s2}{\PYZdq{}}\PY{p}{,}
    \PY{l+s+s2}{\PYZdq{}}\PY{l+s+s2}{activity\PYZus{}name}\PY{l+s+s2}{\PYZdq{}}\PY{p}{,}
    \PY{l+s+s2}{\PYZdq{}}\PY{l+s+s2}{time\PYZus{}stamp}\PY{l+s+s2}{\PYZdq{}}\PY{p}{,}
    \PY{l+s+s2}{\PYZdq{}}\PY{l+s+s2}{id}\PY{l+s+s2}{\PYZdq{}}\PY{p}{,}
\PY{p}{]}  \PY{c+c1}{\PYZsh{} Columns not meant to be cleaned}
\PY{n}{features} \PY{o}{=} \PY{p}{[}\PY{n}{i} \PY{k}{for} \PY{n}{i} \PY{o+ow}{in} \PY{n}{data}\PY{o}{.}\PY{n}{columns} \PY{k}{if} \PY{n}{i} \PY{o+ow}{not} \PY{o+ow}{in} \PY{n}{eliminate}\PY{p}{]}
\PY{n}{clean\PYZus{}data} \PY{o}{=} \PY{n}{data}
\PY{n}{clean\PYZus{}data}\PY{p}{[}\PY{n}{features}\PY{p}{]} \PY{o}{=} \PY{n}{clean\PYZus{}data}\PY{p}{[}\PY{n}{features}\PY{p}{]}\PY{o}{.}\PY{n}{ffill}\PY{p}{(}\PY{p}{)} \PY{c+c1}{\PYZsh{} Code for forward fill}
\PY{n}{display}\PY{p}{(}\PY{n}{clean\PYZus{}data}\PY{o}{.}\PY{n}{head}\PY{p}{(}\PY{p}{)}\PY{p}{)}
\end{Verbatim}
\end{tcolorbox}

    
    \begin{verbatim}
      time_stamp  activity_id  heart_rate  hand_temperature  \
2928       37.66            1         NaN            30.375   
2929       37.67            1         NaN            30.375   
2930       37.68            1         NaN            30.375   
2931       37.69            1         NaN            30.375   
2932       37.70            1       100.0            30.375   

      hand_3D_acceleration_16_x  hand_3D_acceleration_16_y  \
2928                    2.21530                    8.27915   
2929                    2.29196                    7.67288   
2930                    2.29090                    7.14240   
2931                    2.21800                    7.14365   
2932                    2.30106                    7.25857   

      hand_3D_acceleration_16_z  hand_3D_gyroscope_x  hand_3D_gyroscope_y  \
2928                    5.58753            -0.004750             0.037579   
2929                    5.74467            -0.171710             0.025479   
2930                    5.82342            -0.238241             0.011214   
2931                    5.89930            -0.192912             0.019053   
2932                    6.09259            -0.069961            -0.018328   

      hand_3D_gyroscope_z  ...  ankle_3D_acceleration_16_z  \
2928            -0.011145  ...                    0.095156   
2929            -0.009538  ...                   -0.020804   
2930             0.000831  ...                   -0.059173   
2931             0.013374  ...                    0.094385   
2932             0.004582  ...                    0.095775   

      ankle_3D_gyroscope_x  ankle_3D_gyroscope_y  ankle_3D_gyroscope_z  \
2928              0.002908             -0.027714              0.001752   
2929              0.020882              0.000945              0.006007   
2930             -0.035392             -0.052422             -0.004882   
2931             -0.032514             -0.018844              0.026950   
2932              0.001351             -0.048878             -0.006328   

      ankle_3D_magnetometer_x  ankle_3D_magnetometer_y  \
2928                 -61.1081                 -36.8636   
2929                 -60.8916                 -36.3197   
2930                 -60.3407                 -35.7842   
2931                 -60.7646                 -37.1028   
2932                 -60.2040                 -37.1225   

      ankle_3D_magnetometer_z   id  activity_name  activity_type  
2928                 -58.3696  101          lying          light  
2929                 -58.3656  101          lying          light  
2930                 -58.6119  101          lying          light  
2931                 -57.8799  101          lying          light  
2932                 -57.8847  101          lying          light  

[5 rows x 36 columns]
    \end{verbatim}

    
    After using the Forward Fill method, the first four values of heart rate
are still missing. So the first four rows are dropped

    \begin{tcolorbox}[breakable, size=fbox, boxrule=1pt, pad at break*=1mm,colback=cellbackground, colframe=cellborder]
\prompt{In}{incolor}{101}{\boxspacing}
\begin{Verbatim}[commandchars=\\\{\}]
\PY{n}{clean\PYZus{}data} \PY{o}{=} \PY{n}{clean\PYZus{}data}\PY{o}{.}\PY{n}{dropna}\PY{p}{(}\PY{p}{)}
\PY{n}{display}\PY{p}{(}\PY{n}{clean\PYZus{}data}\PY{o}{.}\PY{n}{head}\PY{p}{(}\PY{p}{)}\PY{p}{)}
\end{Verbatim}
\end{tcolorbox}

    
    \begin{verbatim}
      time_stamp  activity_id  heart_rate  hand_temperature  \
2932       37.70            1       100.0            30.375   
2933       37.71            1       100.0            30.375   
2934       37.72            1       100.0            30.375   
2935       37.73            1       100.0            30.375   
2936       37.74            1       100.0            30.375   

      hand_3D_acceleration_16_x  hand_3D_acceleration_16_y  \
2932                    2.30106                    7.25857   
2933                    2.07165                    7.25965   
2934                    2.41148                    7.59780   
2935                    2.32815                    7.63431   
2936                    2.25096                    7.78598   

      hand_3D_acceleration_16_z  hand_3D_gyroscope_x  hand_3D_gyroscope_y  \
2932                    6.09259            -0.069961            -0.018328   
2933                    6.01218             0.063895             0.007175   
2934                    5.93915             0.190837             0.003116   
2935                    5.70686             0.200328            -0.009266   
2936                    5.62821             0.204098            -0.068256   

      hand_3D_gyroscope_z  ...  ankle_3D_acceleration_16_z  \
2932             0.004582  ...                    0.095775   
2933             0.024701  ...                   -0.098161   
2934             0.038762  ...                   -0.098862   
2935             0.068567  ...                   -0.136998   
2936             0.050000  ...                    0.133911   

      ankle_3D_gyroscope_x  ankle_3D_gyroscope_y  ankle_3D_gyroscope_z  \
2932              0.001351             -0.048878             -0.006328   
2933              0.003793             -0.026906              0.004125   
2934              0.036814             -0.032277             -0.006866   
2935             -0.010352             -0.016621              0.006548   
2936              0.039346              0.020393             -0.011880   

      ankle_3D_magnetometer_x  ankle_3D_magnetometer_y  \
2932                 -60.2040                 -37.1225   
2933                 -61.3257                 -36.9744   
2934                 -61.5520                 -36.9632   
2935                 -61.5738                 -36.1724   
2936                 -61.7741                 -37.1744   

      ankle_3D_magnetometer_z   id  activity_name  activity_type  
2932                 -57.8847  101          lying          light  
2933                 -57.7501  101          lying          light  
2934                 -57.9957  101          lying          light  
2935                 -59.3487  101          lying          light  
2936                 -58.1199  101          lying          light  

[5 rows x 36 columns]
    \end{verbatim}

    
    Finally, save the clean data for future use in model prediction

    \begin{tcolorbox}[breakable, size=fbox, boxrule=1pt, pad at break*=1mm,colback=cellbackground, colframe=cellborder]
\prompt{In}{incolor}{102}{\boxspacing}
\begin{Verbatim}[commandchars=\\\{\}]
\PY{n}{clean\PYZus{}data}\PY{o}{.}\PY{n}{to\PYZus{}pickle}\PY{p}{(}\PY{l+s+s2}{\PYZdq{}}\PY{l+s+s2}{clean\PYZus{}act\PYZus{}data.pkl}\PY{l+s+s2}{\PYZdq{}}\PY{p}{)}
\end{Verbatim}
\end{tcolorbox}

    \hypertarget{exploratory-data-analysis}{%
\subsection{\texorpdfstring{Exploratory Data
Analysis}{Exploratory Data Analysis }}\label{exploratory-data-analysis}}

After labelling the data appropriately, 4 subjects are selected for
training set.Subjects 101, 103, 104, 105 are selected for training set
adn rest for training set. 4 subjects for testing set such that the
training and testing set have approximately equal size. In the training
set, we perform Exploratory Data Analysis and come up with potential
hypotheses. We then test those hypotheses on the testing set. 50\% of
data is used for training in this case(Exploratory data analysis) and
the rest for testing.

    Calculating BMI of the subjects

    \begin{tcolorbox}[breakable, size=fbox, boxrule=1pt, pad at break*=1mm,colback=cellbackground, colframe=cellborder]
\prompt{In}{incolor}{103}{\boxspacing}
\begin{Verbatim}[commandchars=\\\{\}]
\PY{n}{height\PYZus{}in\PYZus{}metres} \PY{o}{=} \PY{n}{subj\PYZus{}det}\PY{p}{[}\PY{l+s+s2}{\PYZdq{}}\PY{l+s+s2}{Height (cm)}\PY{l+s+s2}{\PYZdq{}}\PY{p}{]} \PY{o}{/} \PY{l+m+mi}{100} \PY{c+c1}{\PYZsh{} Calculating Height in metres}
\PY{n}{weight\PYZus{}in\PYZus{}kg} \PY{o}{=} \PY{n}{subj\PYZus{}det}\PY{p}{[}\PY{l+s+s2}{\PYZdq{}}\PY{l+s+s2}{Weight (kg)}\PY{l+s+s2}{\PYZdq{}}\PY{p}{]}
\PY{n}{subj\PYZus{}det}\PY{p}{[}\PY{l+s+s2}{\PYZdq{}}\PY{l+s+s2}{BMI}\PY{l+s+s2}{\PYZdq{}}\PY{p}{]} \PY{o}{=} \PY{n}{weight\PYZus{}in\PYZus{}kg} \PY{o}{/} \PY{p}{(}\PY{n}{height\PYZus{}in\PYZus{}metres}\PY{p}{)} \PY{o}{*}\PY{o}{*} \PY{l+m+mi}{2} 
\end{Verbatim}
\end{tcolorbox}

    \hypertarget{data-visualizations}{%
\subsubsection{Data Visualizations}\label{data-visualizations}}

    \begin{itemize}
\tightlist
\item
  Bar chart for frequency of activities.
\end{itemize}

    \begin{tcolorbox}[breakable, size=fbox, boxrule=1pt, pad at break*=1mm,colback=cellbackground, colframe=cellborder]
\prompt{In}{incolor}{105}{\boxspacing}
\begin{Verbatim}[commandchars=\\\{\}]
\PY{n}{rcParams}\PY{p}{[}\PY{l+s+s2}{\PYZdq{}}\PY{l+s+s2}{figure.figsize}\PY{l+s+s2}{\PYZdq{}}\PY{p}{]} \PY{o}{=} \PY{l+m+mi}{40}\PY{p}{,} \PY{l+m+mi}{25} \PY{c+c1}{\PYZsh{} setting the figure dimensions}
\PY{n}{rcParams}\PY{p}{[}\PY{l+s+s2}{\PYZdq{}}\PY{l+s+s2}{font.size}\PY{l+s+s2}{\PYZdq{}}\PY{p}{]} \PY{o}{=}  \PY{l+m+mi}{25} \PY{c+c1}{\PYZsh{} Setting font size}

\PY{n}{ax} \PY{o}{=} \PY{n}{sns}\PY{o}{.}\PY{n}{countplot}\PY{p}{(}\PY{n}{x}\PY{o}{=}\PY{l+s+s2}{\PYZdq{}}\PY{l+s+s2}{activity\PYZus{}name}\PY{l+s+s2}{\PYZdq{}}\PY{p}{,} \PY{n}{data}\PY{o}{=}\PY{n}{train}\PY{p}{)}
\PY{n}{ax}\PY{o}{.}\PY{n}{set\PYZus{}xticklabels}\PY{p}{(}\PY{n}{ax}\PY{o}{.}\PY{n}{get\PYZus{}xticklabels}\PY{p}{(}\PY{p}{)}\PY{p}{,} \PY{n}{rotation}\PY{o}{=}\PY{l+m+mi}{45}\PY{p}{)}  \PY{c+c1}{\PYZsh{} Rotating Text}
\PY{n}{plt}\PY{o}{.}\PY{n}{show}\PY{p}{(}\PY{p}{)}
\end{Verbatim}
\end{tcolorbox}

    \begin{center}
    \adjustimage{max size={0.9\linewidth}{0.9\paperheight}}{trial.pdf_files/trial.pdf_34_0.png}
    \end{center}
    { \hspace*{\fill} \\}
    
    \begin{itemize}
\tightlist
\item
  3D scatter plot of chest acceleration coordinates for lying It is
  expected that vertical chest acceleration will be more while lying due
  to the movements involved and an attempt is made to check this
  visually over here.
\end{itemize}

    \begin{tcolorbox}[breakable, size=fbox, boxrule=1pt, pad at break*=1mm,colback=cellbackground, colframe=cellborder]
\prompt{In}{incolor}{107}{\boxspacing}
\begin{Verbatim}[commandchars=\\\{\}]
\PY{n}{plt}\PY{o}{.}\PY{n}{clf}\PY{p}{(}\PY{p}{)}
\PY{n}{rcParams}\PY{p}{[}\PY{l+s+s2}{\PYZdq{}}\PY{l+s+s2}{font.size}\PY{l+s+s2}{\PYZdq{}}\PY{p}{]} \PY{o}{=}  \PY{l+m+mi}{15}

\PY{n}{train\PYZus{}running} \PY{o}{=} \PY{n}{train}\PY{p}{[}\PY{n}{train}\PY{o}{.}\PY{n}{activity\PYZus{}name} \PY{o}{==} \PY{l+s+s2}{\PYZdq{}}\PY{l+s+s2}{lying}\PY{l+s+s2}{\PYZdq{}}\PY{p}{]} \PY{c+c1}{\PYZsh{} Extracting rows with activity labelled as lying}
\PY{n}{fig} \PY{o}{=} \PY{n}{plt}\PY{o}{.}\PY{n}{figure}\PY{p}{(}\PY{p}{)}
\PY{n}{ax} \PY{o}{=} \PY{n}{fig}\PY{o}{.}\PY{n}{add\PYZus{}subplot}\PY{p}{(}\PY{n}{projection}\PY{o}{=}\PY{l+s+s2}{\PYZdq{}}\PY{l+s+s2}{3d}\PY{l+s+s2}{\PYZdq{}}\PY{p}{)}
\PY{n}{x} \PY{o}{=} \PY{n}{train\PYZus{}running}\PY{p}{[}\PY{l+s+s2}{\PYZdq{}}\PY{l+s+s2}{chest\PYZus{}3D\PYZus{}acceleration\PYZus{}16\PYZus{}x}\PY{l+s+s2}{\PYZdq{}}\PY{p}{]}
\PY{n}{y} \PY{o}{=} \PY{n}{train\PYZus{}running}\PY{p}{[}\PY{l+s+s2}{\PYZdq{}}\PY{l+s+s2}{chest\PYZus{}3D\PYZus{}acceleration\PYZus{}16\PYZus{}y}\PY{l+s+s2}{\PYZdq{}}\PY{p}{]}
\PY{n}{z} \PY{o}{=} \PY{n}{train\PYZus{}running}\PY{p}{[}\PY{l+s+s2}{\PYZdq{}}\PY{l+s+s2}{chest\PYZus{}3D\PYZus{}acceleration\PYZus{}16\PYZus{}z}\PY{l+s+s2}{\PYZdq{}}\PY{p}{]}
\PY{n}{ax}\PY{o}{.}\PY{n}{scatter}\PY{p}{(}\PY{n}{x}\PY{p}{,} \PY{n}{y}\PY{p}{,} \PY{n}{z}\PY{p}{)}
\PY{n}{ax}\PY{o}{.}\PY{n}{set\PYZus{}xlabel}\PY{p}{(}\PY{l+s+s2}{\PYZdq{}}\PY{l+s+s2}{X Axis}\PY{l+s+s2}{\PYZdq{}}\PY{p}{)}
\PY{n}{ax}\PY{o}{.}\PY{n}{set\PYZus{}ylabel}\PY{p}{(}\PY{l+s+s2}{\PYZdq{}}\PY{l+s+s2}{Y Axis}\PY{l+s+s2}{\PYZdq{}}\PY{p}{)}
\PY{n}{ax}\PY{o}{.}\PY{n}{set\PYZus{}zlabel}\PY{p}{(}\PY{l+s+s2}{\PYZdq{}}\PY{l+s+s2}{Z Axis}\PY{l+s+s2}{\PYZdq{}}\PY{p}{)}
\PY{n}{plt}\PY{o}{.}\PY{n}{show}\PY{p}{(}\PY{p}{)}
\end{Verbatim}
\end{tcolorbox}

    
    \begin{verbatim}
<Figure size 4000x2500 with 0 Axes>
    \end{verbatim}

    
    \begin{center}
    \adjustimage{max size={0.9\linewidth}{0.9\paperheight}}{trial.pdf_files/trial.pdf_36_1.png}
    \end{center}
    { \hspace*{\fill} \\}
    
    As we see, there seems to be more variance along the z axis(vertical
direction) than the x and y axis.

    \begin{itemize}
\tightlist
\item
  3D scatter plot of chest acceleration coordinates for running Since
  running involves mostly horizontal movements for the chest, we expect
  most of chest acceleration data to lie on the horizontal x amd y axis.
\end{itemize}

    \begin{tcolorbox}[breakable, size=fbox, boxrule=1pt, pad at break*=1mm,colback=cellbackground, colframe=cellborder]
\prompt{In}{incolor}{108}{\boxspacing}
\begin{Verbatim}[commandchars=\\\{\}]
\PY{n}{plt}\PY{o}{.}\PY{n}{clf}\PY{p}{(}\PY{p}{)}
\PY{n}{train\PYZus{}running} \PY{o}{=} \PY{n}{train}\PY{p}{[}\PY{n}{train}\PY{o}{.}\PY{n}{activity\PYZus{}name} \PY{o}{==} \PY{l+s+s2}{\PYZdq{}}\PY{l+s+s2}{running}\PY{l+s+s2}{\PYZdq{}}\PY{p}{]} \PY{c+c1}{\PYZsh{} Extracting rows  with activity labeleed as running}
\PY{n}{fig} \PY{o}{=} \PY{n}{plt}\PY{o}{.}\PY{n}{figure}\PY{p}{(}\PY{p}{)}
\PY{n}{ax} \PY{o}{=} \PY{n}{fig}\PY{o}{.}\PY{n}{add\PYZus{}subplot}\PY{p}{(}\PY{n}{projection}\PY{o}{=}\PY{l+s+s2}{\PYZdq{}}\PY{l+s+s2}{3d}\PY{l+s+s2}{\PYZdq{}}\PY{p}{)}
\PY{n}{x} \PY{o}{=} \PY{n}{train\PYZus{}running}\PY{p}{[}\PY{l+s+s2}{\PYZdq{}}\PY{l+s+s2}{chest\PYZus{}3D\PYZus{}acceleration\PYZus{}16\PYZus{}x}\PY{l+s+s2}{\PYZdq{}}\PY{p}{]}
\PY{n}{y} \PY{o}{=} \PY{n}{train\PYZus{}running}\PY{p}{[}\PY{l+s+s2}{\PYZdq{}}\PY{l+s+s2}{chest\PYZus{}3D\PYZus{}acceleration\PYZus{}16\PYZus{}y}\PY{l+s+s2}{\PYZdq{}}\PY{p}{]}
\PY{n}{z} \PY{o}{=} \PY{n}{train\PYZus{}running}\PY{p}{[}\PY{l+s+s2}{\PYZdq{}}\PY{l+s+s2}{chest\PYZus{}3D\PYZus{}acceleration\PYZus{}16\PYZus{}z}\PY{l+s+s2}{\PYZdq{}}\PY{p}{]}
\PY{n}{ax}\PY{o}{.}\PY{n}{scatter}\PY{p}{(}\PY{n}{x}\PY{p}{,} \PY{n}{y}\PY{p}{,} \PY{n}{z}\PY{p}{)}
\PY{n}{ax}\PY{o}{.}\PY{n}{set\PYZus{}xlabel}\PY{p}{(}\PY{l+s+s2}{\PYZdq{}}\PY{l+s+s2}{X Axis}\PY{l+s+s2}{\PYZdq{}}\PY{p}{)}
\PY{n}{ax}\PY{o}{.}\PY{n}{set\PYZus{}ylabel}\PY{p}{(}\PY{l+s+s2}{\PYZdq{}}\PY{l+s+s2}{Y Axis}\PY{l+s+s2}{\PYZdq{}}\PY{p}{)}
\PY{n}{ax}\PY{o}{.}\PY{n}{set\PYZus{}zlabel}\PY{p}{(}\PY{l+s+s2}{\PYZdq{}}\PY{l+s+s2}{Z Axis}\PY{l+s+s2}{\PYZdq{}}\PY{p}{)}
\PY{n}{plt}\PY{o}{.}\PY{n}{show}\PY{p}{(}\PY{p}{)}
\end{Verbatim}
\end{tcolorbox}

    
    \begin{verbatim}
<Figure size 4000x2500 with 0 Axes>
    \end{verbatim}

    
    \begin{center}
    \adjustimage{max size={0.9\linewidth}{0.9\paperheight}}{trial.pdf_files/trial.pdf_39_1.png}
    \end{center}
    { \hspace*{\fill} \\}
    
    As we expected, for running, most of the points lie along the x and y
axis.

    \begin{itemize}
\tightlist
\item
  Time series plot of z axis chest acceleration
\end{itemize}

    Subject 101 is considered for this time series plot as it does all
protocol activities

    \begin{tcolorbox}[breakable, size=fbox, boxrule=1pt, pad at break*=1mm,colback=cellbackground, colframe=cellborder]
\prompt{In}{incolor}{110}{\boxspacing}
\begin{Verbatim}[commandchars=\\\{\}]
\PY{n}{plt}\PY{o}{.}\PY{n}{clf}\PY{p}{(}\PY{p}{)}
\PY{n}{rcParams}\PY{p}{[}\PY{l+s+s2}{\PYZdq{}}\PY{l+s+s2}{font.size}\PY{l+s+s2}{\PYZdq{}}\PY{p}{]} \PY{o}{=}  \PY{l+m+mi}{25}

\PY{n}{random}\PY{o}{.}\PY{n}{seed}\PY{p}{(}\PY{l+m+mi}{4}\PY{p}{)}
\PY{n}{train1} \PY{o}{=} \PY{n}{train}\PY{p}{[}\PY{n}{train}\PY{o}{.}\PY{n}{id} \PY{o}{==} \PY{l+m+mi}{101}\PY{p}{]}
\PY{n}{sns}\PY{o}{.}\PY{n}{lineplot}\PY{p}{(}
    \PY{n}{x}\PY{o}{=}\PY{l+s+s2}{\PYZdq{}}\PY{l+s+s2}{time\PYZus{}stamp}\PY{l+s+s2}{\PYZdq{}}\PY{p}{,} \PY{n}{y}\PY{o}{=}\PY{l+s+s2}{\PYZdq{}}\PY{l+s+s2}{chest\PYZus{}3D\PYZus{}acceleration\PYZus{}16\PYZus{}z}\PY{l+s+s2}{\PYZdq{}}\PY{p}{,} \PY{n}{hue}\PY{o}{=}\PY{l+s+s2}{\PYZdq{}}\PY{l+s+s2}{activity\PYZus{}name}\PY{l+s+s2}{\PYZdq{}}\PY{p}{,} \PY{n}{data}\PY{o}{=}\PY{n}{train1}
\PY{p}{)} \PY{c+c1}{\PYZsh{} Generating timeplot and grouping it with activity name}
\PY{n}{plt}\PY{o}{.}\PY{n}{show}\PY{p}{(}\PY{p}{)}
\end{Verbatim}
\end{tcolorbox}

    \begin{center}
    \adjustimage{max size={0.9\linewidth}{0.9\paperheight}}{trial.pdf_files/trial.pdf_43_0.png}
    \end{center}
    { \hspace*{\fill} \\}
    
    It look like the vertical chest acceleration during lying is higher
compared to other activities.Also there seems to be a lot of variance
for this feature while the subject is running.

    \begin{itemize}
\tightlist
\item
  Time series plot of x axis chest acceleration
\end{itemize}

    \begin{tcolorbox}[breakable, size=fbox, boxrule=1pt, pad at break*=1mm,colback=cellbackground, colframe=cellborder]
\prompt{In}{incolor}{112}{\boxspacing}
\begin{Verbatim}[commandchars=\\\{\}]
\PY{n}{plt}\PY{o}{.}\PY{n}{clf}\PY{p}{(}\PY{p}{)}
\PY{n}{random}\PY{o}{.}\PY{n}{seed}\PY{p}{(}\PY{l+m+mi}{4}\PY{p}{)}
\PY{n}{train1} \PY{o}{=} \PY{n}{train}\PY{p}{[}\PY{n}{train}\PY{o}{.}\PY{n}{id} \PY{o}{==} \PY{l+m+mi}{101}\PY{p}{]}
\PY{n}{sns}\PY{o}{.}\PY{n}{lineplot}\PY{p}{(}
    \PY{n}{x}\PY{o}{=}\PY{l+s+s2}{\PYZdq{}}\PY{l+s+s2}{time\PYZus{}stamp}\PY{l+s+s2}{\PYZdq{}}\PY{p}{,} \PY{n}{y}\PY{o}{=}\PY{l+s+s2}{\PYZdq{}}\PY{l+s+s2}{chest\PYZus{}3D\PYZus{}acceleration\PYZus{}16\PYZus{}x}\PY{l+s+s2}{\PYZdq{}}\PY{p}{,} \PY{n}{hue}\PY{o}{=}\PY{l+s+s2}{\PYZdq{}}\PY{l+s+s2}{activity\PYZus{}name}\PY{l+s+s2}{\PYZdq{}}\PY{p}{,} \PY{n}{data}\PY{o}{=}\PY{n}{train1}
\PY{p}{)}
\PY{n}{plt}\PY{o}{.}\PY{n}{show}\PY{p}{(}\PY{p}{)}
\end{Verbatim}
\end{tcolorbox}

    \begin{center}
    \adjustimage{max size={0.9\linewidth}{0.9\paperheight}}{trial.pdf_files/trial.pdf_46_0.png}
    \end{center}
    { \hspace*{\fill} \\}
    
    As expected the variance is higher for activities that require
horizontal movement through space

    \begin{itemize}
\tightlist
\item
  Boxplot of heart rate grouped by activity type.
\end{itemize}

    \begin{tcolorbox}[breakable, size=fbox, boxrule=1pt, pad at break*=1mm,colback=cellbackground, colframe=cellborder]
\prompt{In}{incolor}{120}{\boxspacing}
\begin{Verbatim}[commandchars=\\\{\}]
\PY{n}{rcParams}\PY{p}{[}\PY{l+s+s2}{\PYZdq{}}\PY{l+s+s2}{figure.figsize}\PY{l+s+s2}{\PYZdq{}}\PY{p}{]} \PY{o}{=} \PY{l+m+mi}{10}\PY{p}{,} \PY{l+m+mi}{10}
\PY{n}{rcParams}\PY{p}{[}\PY{l+s+s2}{\PYZdq{}}\PY{l+s+s2}{font.size}\PY{l+s+s2}{\PYZdq{}}\PY{p}{]} \PY{o}{=}  \PY{l+m+mi}{10}

\PY{n}{ax} \PY{o}{=} \PY{n}{sns}\PY{o}{.}\PY{n}{boxplot}\PY{p}{(}\PY{n}{x}\PY{o}{=}\PY{l+s+s2}{\PYZdq{}}\PY{l+s+s2}{activity\PYZus{}type}\PY{l+s+s2}{\PYZdq{}}\PY{p}{,} \PY{n}{y}\PY{o}{=}\PY{l+s+s2}{\PYZdq{}}\PY{l+s+s2}{heart\PYZus{}rate}\PY{l+s+s2}{\PYZdq{}}\PY{p}{,} \PY{n}{data}\PY{o}{=}\PY{n}{train}\PY{p}{)}
\PY{n}{ax}\PY{o}{.}\PY{n}{set\PYZus{}xticklabels}\PY{p}{(}\PY{n}{ax}\PY{o}{.}\PY{n}{get\PYZus{}xticklabels}\PY{p}{(}\PY{p}{)}\PY{p}{,} \PY{n}{rotation}\PY{o}{=}\PY{l+m+mi}{0}\PY{p}{)}  \PY{c+c1}{\PYZsh{} Rotating Text}
\PY{n}{plt}\PY{o}{.}\PY{n}{show}\PY{p}{(}\PY{p}{)}
\end{Verbatim}
\end{tcolorbox}

    \begin{center}
    \adjustimage{max size={0.9\linewidth}{0.9\paperheight}}{trial.pdf_files/trial.pdf_49_0.png}
    \end{center}
    { \hspace*{\fill} \\}
    
    \begin{enumerate}
\def\labelenumi{\arabic{enumi}.}
\tightlist
\item
  It is observed that moderate and intense activities have higher heart
  rate than light activities as expected.
\item
  There doesn't seem to be much seperation between heart rate of
  moderate and intesne activity.
\end{enumerate}

    \begin{itemize}
\tightlist
\item
  Boxplot of heart rate grouped by activity.
\end{itemize}

    \begin{tcolorbox}[breakable, size=fbox, boxrule=1pt, pad at break*=1mm,colback=cellbackground, colframe=cellborder]
\prompt{In}{incolor}{114}{\boxspacing}
\begin{Verbatim}[commandchars=\\\{\}]
\PY{n}{rcParams}\PY{p}{[}\PY{l+s+s2}{\PYZdq{}}\PY{l+s+s2}{figure.figsize}\PY{l+s+s2}{\PYZdq{}}\PY{p}{]} \PY{o}{=} \PY{l+m+mi}{40}\PY{p}{,} \PY{l+m+mi}{25}
\PY{n}{ax} \PY{o}{=} \PY{n}{sns}\PY{o}{.}\PY{n}{boxplot}\PY{p}{(}\PY{n}{x}\PY{o}{=}\PY{l+s+s2}{\PYZdq{}}\PY{l+s+s2}{activity\PYZus{}name}\PY{l+s+s2}{\PYZdq{}}\PY{p}{,} \PY{n}{y}\PY{o}{=}\PY{l+s+s2}{\PYZdq{}}\PY{l+s+s2}{heart\PYZus{}rate}\PY{l+s+s2}{\PYZdq{}}\PY{p}{,} \PY{n}{data}\PY{o}{=}\PY{n}{train}\PY{p}{)}
\PY{n}{ax}\PY{o}{.}\PY{n}{set\PYZus{}xticklabels}\PY{p}{(}\PY{n}{ax}\PY{o}{.}\PY{n}{get\PYZus{}xticklabels}\PY{p}{(}\PY{p}{)}\PY{p}{,} \PY{n}{rotation}\PY{o}{=}\PY{l+m+mi}{45}\PY{p}{)}  \PY{c+c1}{\PYZsh{} Rotating Text}
\PY{n}{plt}\PY{o}{.}\PY{n}{show}\PY{p}{(}\PY{p}{)}
\end{Verbatim}
\end{tcolorbox}

    \begin{center}
    \adjustimage{max size={0.9\linewidth}{0.9\paperheight}}{trial.pdf_files/trial.pdf_52_0.png}
    \end{center}
    { \hspace*{\fill} \\}
    
    \begin{enumerate}
\def\labelenumi{\arabic{enumi}.}
\tightlist
\item
  Most of the activities have a skewed distribution for heart rate.
\item
  `Nordic\_walking',`running' and `cycling' have a lot of outliers on
  the lower side.
\item
  Activities like `lying',`sitting' and `standing' have a lot of
  outliers on the upper side.
\end{enumerate}

    \begin{itemize}
\tightlist
\item
  Boxplot of hand temperature grouped by activity type.
\end{itemize}

    \begin{tcolorbox}[breakable, size=fbox, boxrule=1pt, pad at break*=1mm,colback=cellbackground, colframe=cellborder]
\prompt{In}{incolor}{119}{\boxspacing}
\begin{Verbatim}[commandchars=\\\{\}]
\PY{n}{rcParams}\PY{p}{[}\PY{l+s+s2}{\PYZdq{}}\PY{l+s+s2}{figure.figsize}\PY{l+s+s2}{\PYZdq{}}\PY{p}{]} \PY{o}{=} \PY{l+m+mi}{10}\PY{p}{,} \PY{l+m+mi}{10}
\PY{n}{rcParams}\PY{p}{[}\PY{l+s+s2}{\PYZdq{}}\PY{l+s+s2}{font.size}\PY{l+s+s2}{\PYZdq{}}\PY{p}{]} \PY{o}{=}  \PY{l+m+mi}{10}

\PY{n}{ax} \PY{o}{=} \PY{n}{sns}\PY{o}{.}\PY{n}{boxplot}\PY{p}{(}\PY{n}{x}\PY{o}{=}\PY{l+s+s2}{\PYZdq{}}\PY{l+s+s2}{activity\PYZus{}type}\PY{l+s+s2}{\PYZdq{}}\PY{p}{,} \PY{n}{y}\PY{o}{=}\PY{l+s+s2}{\PYZdq{}}\PY{l+s+s2}{hand\PYZus{}temperature}\PY{l+s+s2}{\PYZdq{}}\PY{p}{,} \PY{n}{data}\PY{o}{=}\PY{n}{train}\PY{p}{)}
\PY{n}{ax}\PY{o}{.}\PY{n}{set\PYZus{}xticklabels}\PY{p}{(}\PY{n}{ax}\PY{o}{.}\PY{n}{get\PYZus{}xticklabels}\PY{p}{(}\PY{p}{)}\PY{p}{,} \PY{n}{rotation}\PY{o}{=}\PY{l+m+mi}{0}\PY{p}{)}
\PY{n}{plt}\PY{o}{.}\PY{n}{show}\PY{p}{(}\PY{p}{)}
\end{Verbatim}
\end{tcolorbox}

    \begin{center}
    \adjustimage{max size={0.9\linewidth}{0.9\paperheight}}{trial.pdf_files/trial.pdf_55_0.png}
    \end{center}
    { \hspace*{\fill} \\}
    
    \begin{enumerate}
\def\labelenumi{\arabic{enumi}.}
\tightlist
\item
  Hand temperature of moderate activitie have a lot of outliers on the
  lower side.
\item
  There doesn't seem to be much difference in temperatures between
  activities.
\end{enumerate}

    \begin{itemize}
\tightlist
\item
  Boxplot of hand temperature grouped by activity.
\end{itemize}

    \begin{tcolorbox}[breakable, size=fbox, boxrule=1pt, pad at break*=1mm,colback=cellbackground, colframe=cellborder]
\prompt{In}{incolor}{125}{\boxspacing}
\begin{Verbatim}[commandchars=\\\{\}]
\PY{n}{rcParams}\PY{p}{[}\PY{l+s+s2}{\PYZdq{}}\PY{l+s+s2}{figure.figsize}\PY{l+s+s2}{\PYZdq{}}\PY{p}{]} \PY{o}{=} \PY{l+m+mi}{40}\PY{p}{,} \PY{l+m+mi}{25}
\PY{n}{rcParams}\PY{p}{[}\PY{l+s+s2}{\PYZdq{}}\PY{l+s+s2}{font.size}\PY{l+s+s2}{\PYZdq{}}\PY{p}{]} \PY{o}{=}  \PY{l+m+mi}{22}

\PY{n}{ax} \PY{o}{=} \PY{n}{sns}\PY{o}{.}\PY{n}{boxplot}\PY{p}{(}\PY{n}{x}\PY{o}{=}\PY{l+s+s2}{\PYZdq{}}\PY{l+s+s2}{activity\PYZus{}name}\PY{l+s+s2}{\PYZdq{}}\PY{p}{,} \PY{n}{y}\PY{o}{=}\PY{l+s+s2}{\PYZdq{}}\PY{l+s+s2}{hand\PYZus{}temperature}\PY{l+s+s2}{\PYZdq{}}\PY{p}{,} \PY{n}{data}\PY{o}{=}\PY{n}{train}\PY{p}{)}
\PY{n}{ax}\PY{o}{.}\PY{n}{set\PYZus{}xticklabels}\PY{p}{(}\PY{n}{ax}\PY{o}{.}\PY{n}{get\PYZus{}xticklabels}\PY{p}{(}\PY{p}{)}\PY{p}{,} \PY{n}{rotation}\PY{o}{=}\PY{l+m+mi}{45}\PY{p}{)}  \PY{c+c1}{\PYZsh{} Rotating Text}
\PY{n}{plt}\PY{o}{.}\PY{n}{show}\PY{p}{(}\PY{p}{)}
\end{Verbatim}
\end{tcolorbox}

    \begin{center}
    \adjustimage{max size={0.9\linewidth}{0.9\paperheight}}{trial.pdf_files/trial.pdf_58_0.png}
    \end{center}
    { \hspace*{\fill} \\}
    
    \begin{itemize}
\tightlist
\item
  Boxplot of ankle temperature grouped by activity\_type
\end{itemize}

    \begin{tcolorbox}[breakable, size=fbox, boxrule=1pt, pad at break*=1mm,colback=cellbackground, colframe=cellborder]
\prompt{In}{incolor}{126}{\boxspacing}
\begin{Verbatim}[commandchars=\\\{\}]
\PY{n}{rcParams}\PY{p}{[}\PY{l+s+s2}{\PYZdq{}}\PY{l+s+s2}{figure.figsize}\PY{l+s+s2}{\PYZdq{}}\PY{p}{]} \PY{o}{=} \PY{l+m+mi}{15}\PY{p}{,} \PY{l+m+mi}{10}
\PY{n}{rcParams}\PY{p}{[}\PY{l+s+s2}{\PYZdq{}}\PY{l+s+s2}{font.size}\PY{l+s+s2}{\PYZdq{}}\PY{p}{]} \PY{o}{=}  \PY{l+m+mi}{20}

\PY{n}{ax} \PY{o}{=} \PY{n}{sns}\PY{o}{.}\PY{n}{boxplot}\PY{p}{(}\PY{n}{x}\PY{o}{=}\PY{l+s+s2}{\PYZdq{}}\PY{l+s+s2}{activity\PYZus{}type}\PY{l+s+s2}{\PYZdq{}}\PY{p}{,} \PY{n}{y}\PY{o}{=}\PY{l+s+s2}{\PYZdq{}}\PY{l+s+s2}{ankle\PYZus{}temperature}\PY{l+s+s2}{\PYZdq{}}\PY{p}{,} \PY{n}{data}\PY{o}{=}\PY{n}{train}\PY{p}{)}
\PY{n}{ax}\PY{o}{.}\PY{n}{set\PYZus{}xticklabels}\PY{p}{(}\PY{n}{ax}\PY{o}{.}\PY{n}{get\PYZus{}xticklabels}\PY{p}{(}\PY{p}{)}\PY{p}{,} \PY{n}{rotation}\PY{o}{=}\PY{l+m+mi}{0}\PY{p}{)}
\PY{n}{plt}\PY{o}{.}\PY{n}{show}\PY{p}{(}\PY{p}{)}
\end{Verbatim}
\end{tcolorbox}

    \begin{center}
    \adjustimage{max size={0.9\linewidth}{0.9\paperheight}}{trial.pdf_files/trial.pdf_60_0.png}
    \end{center}
    { \hspace*{\fill} \\}
    
    \begin{enumerate}
\def\labelenumi{\arabic{enumi}.}
\tightlist
\item
  Ankle temperature of light and moderate activitie have outliers on the
  lower side.
\item
  There doesn't seem to be much difference in temperatures between
  activities.
\end{enumerate}

    \begin{itemize}
\tightlist
\item
  Boxplot of ankle temperature grouped by activity
\end{itemize}

    \begin{tcolorbox}[breakable, size=fbox, boxrule=1pt, pad at break*=1mm,colback=cellbackground, colframe=cellborder]
\prompt{In}{incolor}{128}{\boxspacing}
\begin{Verbatim}[commandchars=\\\{\}]
\PY{n}{rcParams}\PY{p}{[}\PY{l+s+s2}{\PYZdq{}}\PY{l+s+s2}{figure.figsize}\PY{l+s+s2}{\PYZdq{}}\PY{p}{]} \PY{o}{=} \PY{l+m+mi}{40}\PY{p}{,} \PY{l+m+mi}{25}
\PY{n}{rcParams}\PY{p}{[}\PY{l+s+s2}{\PYZdq{}}\PY{l+s+s2}{font.size}\PY{l+s+s2}{\PYZdq{}}\PY{p}{]} \PY{o}{=}  \PY{l+m+mi}{25}

\PY{n}{ax} \PY{o}{=} \PY{n}{sns}\PY{o}{.}\PY{n}{boxplot}\PY{p}{(}\PY{n}{x}\PY{o}{=}\PY{l+s+s2}{\PYZdq{}}\PY{l+s+s2}{activity\PYZus{}name}\PY{l+s+s2}{\PYZdq{}}\PY{p}{,} \PY{n}{y}\PY{o}{=}\PY{l+s+s2}{\PYZdq{}}\PY{l+s+s2}{ankle\PYZus{}temperature}\PY{l+s+s2}{\PYZdq{}}\PY{p}{,} \PY{n}{data}\PY{o}{=}\PY{n}{train}\PY{p}{)}
\PY{n}{ax}\PY{o}{.}\PY{n}{set\PYZus{}xticklabels}\PY{p}{(}\PY{n}{ax}\PY{o}{.}\PY{n}{get\PYZus{}xticklabels}\PY{p}{(}\PY{p}{)}\PY{p}{,} \PY{n}{rotation}\PY{o}{=}\PY{l+m+mi}{45}\PY{p}{)}  \PY{c+c1}{\PYZsh{} Rotating Text}
\PY{n}{plt}\PY{o}{.}\PY{n}{show}\PY{p}{(}\PY{p}{)}
\end{Verbatim}
\end{tcolorbox}

    \begin{center}
    \adjustimage{max size={0.9\linewidth}{0.9\paperheight}}{trial.pdf_files/trial.pdf_63_0.png}
    \end{center}
    { \hspace*{\fill} \\}
    
    \begin{enumerate}
\def\labelenumi{\arabic{enumi}.}
\tightlist
\item
  Outliers are mostly present in `vacuum\_cleaning' on the lower side.
\end{enumerate}

    \begin{itemize}
\tightlist
\item
  Boxplot of chest temperature grouped by activity\_type
\end{itemize}

    \begin{tcolorbox}[breakable, size=fbox, boxrule=1pt, pad at break*=1mm,colback=cellbackground, colframe=cellborder]
\prompt{In}{incolor}{130}{\boxspacing}
\begin{Verbatim}[commandchars=\\\{\}]
\PY{n}{rcParams}\PY{p}{[}\PY{l+s+s2}{\PYZdq{}}\PY{l+s+s2}{figure.figsize}\PY{l+s+s2}{\PYZdq{}}\PY{p}{]} \PY{o}{=} \PY{l+m+mi}{15}\PY{p}{,} \PY{l+m+mi}{10}
\PY{n}{rcParams}\PY{p}{[}\PY{l+s+s2}{\PYZdq{}}\PY{l+s+s2}{font.size}\PY{l+s+s2}{\PYZdq{}}\PY{p}{]} \PY{o}{=}  \PY{l+m+mi}{15}

\PY{n}{ax} \PY{o}{=} \PY{n}{sns}\PY{o}{.}\PY{n}{boxplot}\PY{p}{(}\PY{n}{x}\PY{o}{=}\PY{l+s+s2}{\PYZdq{}}\PY{l+s+s2}{activity\PYZus{}type}\PY{l+s+s2}{\PYZdq{}}\PY{p}{,} \PY{n}{y}\PY{o}{=}\PY{l+s+s2}{\PYZdq{}}\PY{l+s+s2}{chest\PYZus{}temperature}\PY{l+s+s2}{\PYZdq{}}\PY{p}{,} \PY{n}{data}\PY{o}{=}\PY{n}{train}\PY{p}{)}
\PY{n}{ax}\PY{o}{.}\PY{n}{set\PYZus{}xticklabels}\PY{p}{(}\PY{n}{ax}\PY{o}{.}\PY{n}{get\PYZus{}xticklabels}\PY{p}{(}\PY{p}{)}\PY{p}{,} \PY{n}{rotation}\PY{o}{=}\PY{l+m+mi}{0}\PY{p}{)}
\PY{n}{plt}\PY{o}{.}\PY{n}{show}\PY{p}{(}\PY{p}{)}
\end{Verbatim}
\end{tcolorbox}

    \begin{center}
    \adjustimage{max size={0.9\linewidth}{0.9\paperheight}}{trial.pdf_files/trial.pdf_66_0.png}
    \end{center}
    { \hspace*{\fill} \\}
    
    \begin{enumerate}
\def\labelenumi{\arabic{enumi}.}
\tightlist
\item
  For chest temperatures, only the `intense' activity type has one
  outlier.
\item
  For this feature as well, there doesn't seem to be much difference
  between temperatures.
\end{enumerate}

    \begin{itemize}
\tightlist
\item
  Boxplot of chest temperature grouped by activity.
\end{itemize}

    \begin{tcolorbox}[breakable, size=fbox, boxrule=1pt, pad at break*=1mm,colback=cellbackground, colframe=cellborder]
\prompt{In}{incolor}{134}{\boxspacing}
\begin{Verbatim}[commandchars=\\\{\}]
\PY{n}{rcParams}\PY{p}{[}\PY{l+s+s2}{\PYZdq{}}\PY{l+s+s2}{figure.figsize}\PY{l+s+s2}{\PYZdq{}}\PY{p}{]} \PY{o}{=} \PY{l+m+mi}{40}\PY{p}{,} \PY{l+m+mi}{25}
\PY{n}{rcParams}\PY{p}{[}\PY{l+s+s2}{\PYZdq{}}\PY{l+s+s2}{font.size}\PY{l+s+s2}{\PYZdq{}}\PY{p}{]} \PY{o}{=}  \PY{l+m+mi}{25}

\PY{n}{ax} \PY{o}{=} \PY{n}{sns}\PY{o}{.}\PY{n}{boxplot}\PY{p}{(}\PY{n}{x}\PY{o}{=}\PY{l+s+s2}{\PYZdq{}}\PY{l+s+s2}{activity\PYZus{}name}\PY{l+s+s2}{\PYZdq{}}\PY{p}{,} \PY{n}{y}\PY{o}{=}\PY{l+s+s2}{\PYZdq{}}\PY{l+s+s2}{chest\PYZus{}temperature}\PY{l+s+s2}{\PYZdq{}}\PY{p}{,} \PY{n}{data}\PY{o}{=}\PY{n}{train}\PY{p}{)}
\PY{n}{ax}\PY{o}{.}\PY{n}{set\PYZus{}xticklabels}\PY{p}{(}\PY{n}{ax}\PY{o}{.}\PY{n}{get\PYZus{}xticklabels}\PY{p}{(}\PY{p}{)}\PY{p}{,} \PY{n}{rotation}\PY{o}{=}\PY{l+m+mi}{45}\PY{p}{)}  \PY{c+c1}{\PYZsh{} Rotating Text}
\PY{n}{plt}\PY{o}{.}\PY{n}{show}\PY{p}{(}\PY{p}{)}
\end{Verbatim}
\end{tcolorbox}

    \begin{center}
    \adjustimage{max size={0.9\linewidth}{0.9\paperheight}}{trial.pdf_files/trial.pdf_69_0.png}
    \end{center}
    { \hspace*{\fill} \\}
    
    \begin{enumerate}
\def\labelenumi{\arabic{enumi}.}
\tightlist
\item
  Most of the activities seem to have a skewed distribution for chest
  temperature.
\end{enumerate}

    \begin{itemize}
\tightlist
\item
  Correlation map for relevant features
\end{itemize}

    \begin{tcolorbox}[breakable, size=fbox, boxrule=1pt, pad at break*=1mm,colback=cellbackground, colframe=cellborder]
\prompt{In}{incolor}{135}{\boxspacing}
\begin{Verbatim}[commandchars=\\\{\}]
\PY{n}{discard} \PY{o}{=} \PY{p}{[}
    \PY{l+s+s2}{\PYZdq{}}\PY{l+s+s2}{activity\PYZus{}id}\PY{l+s+s2}{\PYZdq{}}\PY{p}{,}
    \PY{l+s+s2}{\PYZdq{}}\PY{l+s+s2}{activity}\PY{l+s+s2}{\PYZdq{}}\PY{p}{,}
    \PY{l+s+s2}{\PYZdq{}}\PY{l+s+s2}{time\PYZus{}stamp}\PY{l+s+s2}{\PYZdq{}}\PY{p}{,}
    \PY{l+s+s2}{\PYZdq{}}\PY{l+s+s2}{id}\PY{l+s+s2}{\PYZdq{}}\PY{p}{,}
\PY{p}{]}  \PY{c+c1}{\PYZsh{} Columns to exclude from correlation map and descriptive statistics}
\PY{n}{train\PYZus{}trimmed} \PY{o}{=} \PY{n}{train}\PY{p}{[}\PY{n+nb}{set}\PY{p}{(}\PY{n}{train}\PY{o}{.}\PY{n}{columns}\PY{p}{)}\PY{o}{.}\PY{n}{difference}\PY{p}{(}\PY{n+nb}{set}\PY{p}{(}\PY{n}{discard}\PY{p}{)}\PY{p}{)}\PY{p}{]}
\end{Verbatim}
\end{tcolorbox}

    \begin{tcolorbox}[breakable, size=fbox, boxrule=1pt, pad at break*=1mm,colback=cellbackground, colframe=cellborder]
\prompt{In}{incolor}{137}{\boxspacing}
\begin{Verbatim}[commandchars=\\\{\}]
\PY{n}{rcParams}\PY{p}{[}\PY{l+s+s2}{\PYZdq{}}\PY{l+s+s2}{figure.figsize}\PY{l+s+s2}{\PYZdq{}}\PY{p}{]} \PY{o}{=} \PY{l+m+mi}{20}\PY{p}{,} \PY{l+m+mi}{20} \PY{c+c1}{\PYZsh{} Setting figure dimension}
\PY{n}{sns}\PY{o}{.}\PY{n}{heatmap}\PY{p}{(}\PY{n}{train\PYZus{}trimmed}\PY{o}{.}\PY{n}{corr}\PY{p}{(}\PY{p}{)}\PY{p}{,} \PY{n}{cmap}\PY{o}{=}\PY{l+s+s2}{\PYZdq{}}\PY{l+s+s2}{BrBG}\PY{l+s+s2}{\PYZdq{}}\PY{p}{)} \PY{c+c1}{\PYZsh{} Setting colorsheme and giving correlation matrix as input}
\PY{n}{plt}\PY{o}{.}\PY{n}{show}\PY{p}{(}\PY{p}{)}
\end{Verbatim}
\end{tcolorbox}

    \begin{center}
    \adjustimage{max size={0.9\linewidth}{0.9\paperheight}}{trial.pdf_files/trial.pdf_73_0.png}
    \end{center}
    { \hspace*{\fill} \\}
    
    There seems to be a lot of significant correlations between many
features

    \hypertarget{descriptive-statistics}{%
\subsubsection{\texorpdfstring{Descriptive
Statistics}{Descriptive Statistics }}\label{descriptive-statistics}}

Subject Details

    \begin{tcolorbox}[breakable, size=fbox, boxrule=1pt, pad at break*=1mm,colback=cellbackground, colframe=cellborder]
\prompt{In}{incolor}{138}{\boxspacing}
\begin{Verbatim}[commandchars=\\\{\}]
\PY{n}{display}\PY{p}{(}\PY{n}{subj\PYZus{}det}\PY{p}{)}
\end{Verbatim}
\end{tcolorbox}

    
    \begin{verbatim}
   Subject ID     Sex  Age (years)  Height (cm)  Weight (kg)  \
0         101    Male           27          182           83   
1         102  Female           25          169           78   
2         103    Male           31          187           92   
3         104    Male           24          194           95   
4         105    Male           26          180           73   
5         106    Male           26          183           69   
6         107    Male           23          173           86   
7         108    Male           32          179           87   
8         109    Male           31          168           65   

   Resting HR (bpm)  Max HR (bpm) Dominant hand        BMI  
0                75           193         right  25.057360  
1                74           195         right  27.309968  
2                68           189         right  26.309017  
3                58           196         right  25.241790  
4                70           194         right  22.530864  
5                60           194         right  20.603780  
6                60           197         right  28.734672  
7                66           188          left  27.152711  
8                54           189         right  23.030045  
    \end{verbatim}

    
    Mean of heart rate and temperatures for each activity

    \begin{tcolorbox}[breakable, size=fbox, boxrule=1pt, pad at break*=1mm,colback=cellbackground, colframe=cellborder]
\prompt{In}{incolor}{139}{\boxspacing}
\begin{Verbatim}[commandchars=\\\{\}]
\PY{n}{display}\PY{p}{(}
    \PY{n}{train}\PY{o}{.}\PY{n}{groupby}\PY{p}{(}\PY{n}{by}\PY{o}{=}\PY{l+s+s2}{\PYZdq{}}\PY{l+s+s2}{activity\PYZus{}name}\PY{l+s+s2}{\PYZdq{}}\PY{p}{)}\PY{p}{[}
        \PY{p}{[}\PY{l+s+s2}{\PYZdq{}}\PY{l+s+s2}{heart\PYZus{}rate}\PY{l+s+s2}{\PYZdq{}}\PY{p}{,} \PY{l+s+s2}{\PYZdq{}}\PY{l+s+s2}{chest\PYZus{}temperature}\PY{l+s+s2}{\PYZdq{}}\PY{p}{,} \PY{l+s+s2}{\PYZdq{}}\PY{l+s+s2}{hand\PYZus{}temperature}\PY{l+s+s2}{\PYZdq{}}\PY{p}{,} \PY{l+s+s2}{\PYZdq{}}\PY{l+s+s2}{ankle\PYZus{}temperature}\PY{l+s+s2}{\PYZdq{}}\PY{p}{]}
    \PY{p}{]}\PY{o}{.}\PY{n}{mean}\PY{p}{(}\PY{p}{)}
\PY{p}{)} 
\end{Verbatim}
\end{tcolorbox}

    
    \begin{verbatim}
                   heart_rate  chest_temperature  hand_temperature  \
activity_name                                                        
Nordic_walking     128.934574          36.640204         32.124192   
ascending_stairs   132.404398          36.586821         33.381660   
cycling            127.117356          35.894279         31.493273   
descending_stairs  130.733971          36.701044         33.195439   
ironing             94.586717          36.162343         33.845229   
lying               78.609097          34.449084         32.508522   
rope_jumping       165.084261          34.773034         31.562491   
running            158.613734          35.262980         32.372712   
sitting             82.242313          35.238913         33.025149   
standing            95.112994          35.590371         33.445441   
vacuum_cleaning    107.774620          36.530023         34.006465   
walking            115.147733          36.964260         32.232594   

                   ankle_temperature  
activity_name                         
Nordic_walking             33.778505  
ascending_stairs           34.146043  
cycling                    33.617997  
descending_stairs          34.147815  
ironing                    34.103435  
lying                      32.690584  
rope_jumping               33.499627  
running                    33.646088  
sitting                    33.370173  
standing                   33.742018  
vacuum_cleaning            34.358195  
walking                    34.152572  
    \end{verbatim}

    
    Descriptive info of relevant features

    \begin{tcolorbox}[breakable, size=fbox, boxrule=1pt, pad at break*=1mm,colback=cellbackground, colframe=cellborder]
\prompt{In}{incolor}{140}{\boxspacing}
\begin{Verbatim}[commandchars=\\\{\}]
\PY{n}{display}\PY{p}{(}\PY{n}{train\PYZus{}trimmed}\PY{o}{.}\PY{n}{describe}\PY{p}{(}\PY{p}{)}\PY{p}{)}
\end{Verbatim}
\end{tcolorbox}

    
    \begin{verbatim}
       chest_temperature  hand_3D_magnetometer_x    heart_rate  \
count      927296.000000           922733.000000  84798.000000   
mean           35.961841               20.935601    109.912769   
std             1.398668               25.717188     26.029426   
min            32.187500             -103.941000     57.000000   
25%            35.062500                2.499940     89.000000   
50%            36.000000               22.012300    109.000000   
75%            37.000000               40.078900    127.000000   
max            38.562500              133.830000    183.000000   

       ankle_temperature  chest_3D_acceleration_16_x  hand_3D_gyroscope_x  \
count      924261.000000               927296.000000        922733.000000   
mean           33.776342                    0.240625             0.012205   
std             0.761948                    1.736878             1.251699   
min            30.750000                  -39.203400           -27.804400   
25%            33.375000                   -0.580049            -0.387488   
50%            34.000000                    0.355426            -0.005547   
75%            34.312500                    1.035780             0.338614   
max            35.000000                   27.522300            26.415800   

       chest_3D_magnetometer_z  chest_3D_gyroscope_y  hand_3D_magnetometer_z  \
count            927296.000000         927296.000000           922733.000000   
mean                  3.616971              0.009578              -25.306925   
std                  23.695410              0.549806               22.172517   
min                 -66.684700             -4.672250             -164.937000   
25%                 -11.103325             -0.135082              -40.472500   
50%                   1.054870             -0.000901              -24.964300   
75%                  20.523525              0.162768              -10.695900   
max                  96.358500              4.540310              101.758000   

       ankle_3D_gyroscope_y  ...  chest_3D_acceleration_16_y  \
count         924261.000000  ...               927296.000000   
mean              -0.029152  ...                    8.137126   
std                0.571440  ...                    4.867817   
min               -7.807450  ...                  -25.955900   
25%               -0.130252  ...                    5.901193   
50%               -0.006023  ...                    9.265210   
75%                0.085158  ...                    9.768020   
max                6.410380  ...                  107.825000   

       hand_3D_gyroscope_z  chest_3D_magnetometer_x  chest_3D_gyroscope_z  \
count        922733.000000            927296.000000         927296.000000   
mean             -0.000027                 4.634805             -0.025351   
std               1.550689                18.799139              0.288043   
min             -14.264700               -70.062700             -2.642760   
25%              -0.354068                -6.381310             -0.127878   
50%              -0.004496                 3.193960             -0.016610   
75%               0.393650                14.399500              0.077227   
max              14.338400                80.473900              2.716240   

       ankle_3D_gyroscope_x  ankle_3D_magnetometer_y  ankle_3D_magnetometer_z  \
count         924261.000000            924261.000000            924261.000000   
mean               0.007019                -0.243321                16.765524   
std                1.022017                23.717646                21.612666   
min              -13.385600              -137.908000              -102.716000   
25%               -0.201214               -15.123500                 2.088300   
50%                0.003077                -0.349413                20.468600   
75%                0.095482                17.762500                30.407100   
max               13.142500                94.247800               122.521000   

       chest_3D_acceleration_16_z  ankle_3D_magnetometer_x  \
count               927296.000000            924261.000000   
mean                    -1.164960               -34.292921   
std                      4.721856                21.381194   
min                    -53.401900              -172.865000   
25%                     -3.827573               -45.705600   
50%                     -1.233810               -35.768600   
75%                      0.835319               -17.365300   
max                     17.878100                91.551600   

       hand_3D_acceleration_16_z  
count              922733.000000  
mean                    3.819557  
std                     3.675067  
min                   -38.907800  
25%                     1.721670  
50%                     3.721100  
75%                     6.407160  
max                    76.639600  

[8 rows x 31 columns]
    \end{verbatim}

    
    Variance of each axis of acceleration grouped by activities It is
expected that variance along x,y and z axis for acceleration will be
different for different activities.An attempt is made to investigate
this.

    \begin{tcolorbox}[breakable, size=fbox, boxrule=1pt, pad at break*=1mm,colback=cellbackground, colframe=cellborder]
\prompt{In}{incolor}{141}{\boxspacing}
\begin{Verbatim}[commandchars=\\\{\}]
\PY{n}{coordinates} \PY{o}{=} \PY{p}{[}\PY{n}{i} \PY{k}{for} \PY{n}{i} \PY{o+ow}{in} \PY{n}{train}\PY{o}{.}\PY{n}{columns} \PY{k}{if} \PY{l+s+s2}{\PYZdq{}}\PY{l+s+s2}{acceleration}\PY{l+s+s2}{\PYZdq{}} \PY{o+ow}{in} \PY{n}{i}\PY{p}{]}
\PY{n}{display}\PY{p}{(}\PY{n}{train}\PY{o}{.}\PY{n}{groupby}\PY{p}{(}\PY{n}{by}\PY{o}{=}\PY{l+s+s2}{\PYZdq{}}\PY{l+s+s2}{activity\PYZus{}name}\PY{l+s+s2}{\PYZdq{}}\PY{p}{)}\PY{p}{[}\PY{n}{coordinates}\PY{p}{]}\PY{o}{.}\PY{n}{var}\PY{p}{(}\PY{p}{)}\PY{p}{)}
\end{Verbatim}
\end{tcolorbox}

    
    \begin{verbatim}
                   hand_3D_acceleration_16_x  hand_3D_acceleration_16_y  \
activity_name                                                             
Nordic_walking                     25.629540                  58.398422   
ascending_stairs                   23.543054                   7.566316   
cycling                            17.791446                  15.649830   
descending_stairs                  24.243241                  11.550501   
ironing                            10.310193                  10.363109   
lying                              15.855200                  13.097997   
rope_jumping                       75.582150                  58.570550   
running                           110.632062                 212.062379   
sitting                            10.279172                  13.464679   
standing                           20.283887                   6.405919   
vacuum_cleaning                    18.818634                  15.030792   
walking                            13.350637                   6.510350   

                   hand_3D_acceleration_16_z  chest_3D_acceleration_16_x  \
activity_name                                                              
Nordic_walking                     10.603472                    2.575901   
ascending_stairs                    5.824274                    2.765901   
cycling                             9.939227                    0.715780   
descending_stairs                  12.489337                    2.984501   
ironing                            13.358504                    2.033213   
lying                              13.166171                    4.153232   
rope_jumping                       38.432095                    4.649561   
running                            20.369313                    8.078898   
sitting                            10.087709                    0.247896   
standing                            3.793687                    0.679866   
vacuum_cleaning                    11.900489                    5.039040   
walking                             6.264966                    2.527115   

                   chest_3D_acceleration_16_y  chest_3D_acceleration_16_z  \
activity_name                                                               
Nordic_walking                      15.242082                    6.148342   
ascending_stairs                    10.775441                    5.624425   
cycling                              3.711593                    4.622890   
descending_stairs                   21.431527                    4.440417   
ironing                              0.736907                    4.374898   
lying                                6.062105                   12.668459   
rope_jumping                       247.259665                   29.357327   
running                            102.004097                    9.118545   
sitting                              0.804636                    6.960005   
standing                             0.068863                    1.491701   
vacuum_cleaning                     14.608914                   10.007767   
walking                             10.131506                    4.359562   

                   ankle_3D_acceleration_16_x  ankle_3D_acceleration_16_y  \
activity_name                                                               
Nordic_walking                      33.484057                   92.289911   
ascending_stairs                    42.033634                   55.288195   
cycling                             17.858721                   12.164733   
descending_stairs                   47.242317                   70.014202   
ironing                              0.174704                    1.796946   
lying                                7.192011                   11.658936   
rope_jumping                       118.280760                  234.159169   
running                            161.038054                  228.653523   
sitting                              6.838823                    7.252916   
standing                             0.081663                    2.684077   
vacuum_cleaning                      1.981244                    7.303831   
walking                             34.600482                   90.401580   

                   ankle_3D_acceleration_16_z  
activity_name                                  
Nordic_walking                      15.376649  
ascending_stairs                    28.013199  
cycling                              3.018095  
descending_stairs                   25.787439  
ironing                              1.696242  
lying                               12.639666  
rope_jumping                        50.880373  
running                             47.994581  
sitting                              8.381130  
standing                             1.072036  
vacuum_cleaning                      4.504396  
walking                             17.312541  
    \end{verbatim}

    
    As we notice the variance is quite different along different axes for
different activities

    \hypertarget{hypothesis-testing}{%
\subsection{Hypothesis Testing}\label{hypothesis-testing}}

    Based on the exploratory data analysis carried out, the following
hypotheses are tested on the test set: - Heart rate of moderate
activities are greater than heart rate of light activities. - Heart rate
of intense activities are greater than heart rate of light activities. -
Chest acceleration along z axis is greater while lying compared to z
axis chest acceleration of other activities.

    Based on the EDA we performed, it does not seem that the data is
normally distributed. It is for this reason that Wilcoxon rank sum test
was used to test the above hypothesis instead of the usual t-test which
assumes that the samples follow a normal distribution. We test the above
hypothesis using the confidence level of 5\%.

    \hypertarget{hypothesis-1}{%
\subsubsection{\texorpdfstring{Hypothesis
1}{Hypothesis 1 }}\label{hypothesis-1}}

\(H_0\)(Null) : The heart rate during moderate activities are the same
or lower than that of light activities. \(H_1\)(Alternate) : The heart
rate during moderate activities are likely to be higher during lying
compared to light activities.

    \begin{tcolorbox}[breakable, size=fbox, boxrule=1pt, pad at break*=1mm,colback=cellbackground, colframe=cellborder]
\prompt{In}{incolor}{145}{\boxspacing}
\begin{Verbatim}[commandchars=\\\{\}]
\PY{n}{test1} \PY{o}{=} \PY{n}{test}\PY{p}{[}
    \PY{n}{test}\PY{o}{.}\PY{n}{activity\PYZus{}type} \PY{o}{==} \PY{l+s+s2}{\PYZdq{}}\PY{l+s+s2}{moderate}\PY{l+s+s2}{\PYZdq{}}
\PY{p}{]}\PY{o}{.}\PY{n}{heart\PYZus{}rate}\PY{o}{.}\PY{n}{dropna}\PY{p}{(}\PY{p}{)}  \PY{c+c1}{\PYZsh{} Heart rate of moderate activities with nan values dropped}
\PY{n}{test2} \PY{o}{=} \PY{n}{test}\PY{p}{[}
    \PY{n}{test}\PY{o}{.}\PY{n}{activity\PYZus{}type} \PY{o}{==} \PY{l+s+s2}{\PYZdq{}}\PY{l+s+s2}{light}\PY{l+s+s2}{\PYZdq{}}
\PY{p}{]}\PY{o}{.}\PY{n}{heart\PYZus{}rate}\PY{o}{.}\PY{n}{dropna}\PY{p}{(}\PY{p}{)}  \PY{c+c1}{\PYZsh{} Heart rate of light activities with nan values dropped}
\PY{n+nb}{print}\PY{p}{(}\PY{n}{ranksums}\PY{p}{(}\PY{n}{test1}\PY{p}{,} \PY{n}{test2}\PY{p}{,} \PY{n}{alternative}\PY{o}{=}\PY{l+s+s2}{\PYZdq{}}\PY{l+s+s2}{greater}\PY{l+s+s2}{\PYZdq{}}\PY{p}{)}\PY{p}{)}
\end{Verbatim}
\end{tcolorbox}

    \begin{Verbatim}[commandchars=\\\{\}]
RanksumsResult(statistic=188.93129841668087, pvalue=0.0)
    \end{Verbatim}

    Since we get a p value of 0 we have to reject the null hypothesis and
accept the alternate hypotheses that the moderate intensity activities
have higher heart rate than light intensity activities

    \hypertarget{hypothesis-2}{%
\subsubsection{\texorpdfstring{Hypothesis
2}{Hypothesis 2 }}\label{hypothesis-2}}

\(H_0\)(Null) : The heart rate during intense activities are the same or
lower than that of light activities. \(H_1\)(Alternate) : The heart rate
during intense activities are likely to be higher during than during
lower activities.

    \begin{tcolorbox}[breakable, size=fbox, boxrule=1pt, pad at break*=1mm,colback=cellbackground, colframe=cellborder]
\prompt{In}{incolor}{146}{\boxspacing}
\begin{Verbatim}[commandchars=\\\{\}]
\PY{n}{test1} \PY{o}{=} \PY{n}{test}\PY{p}{[}
    \PY{n}{test}\PY{o}{.}\PY{n}{activity\PYZus{}type} \PY{o}{==} \PY{l+s+s2}{\PYZdq{}}\PY{l+s+s2}{intense}\PY{l+s+s2}{\PYZdq{}}
\PY{p}{]}\PY{o}{.}\PY{n}{heart\PYZus{}rate}\PY{o}{.}\PY{n}{dropna}\PY{p}{(}\PY{p}{)}  \PY{c+c1}{\PYZsh{} Heart rate of moderate activities with nan values dropped}
\PY{n}{test2} \PY{o}{=} \PY{n}{test}\PY{p}{[}
    \PY{n}{test}\PY{o}{.}\PY{n}{activity\PYZus{}type} \PY{o}{==} \PY{l+s+s2}{\PYZdq{}}\PY{l+s+s2}{light}\PY{l+s+s2}{\PYZdq{}}
\PY{p}{]}\PY{o}{.}\PY{n}{heart\PYZus{}rate}\PY{o}{.}\PY{n}{dropna}\PY{p}{(}\PY{p}{)}  \PY{c+c1}{\PYZsh{} Heart rate of light activities with nan values dropped}
\PY{n+nb}{print}\PY{p}{(}\PY{n}{ranksums}\PY{p}{(}\PY{n}{test1}\PY{p}{,} \PY{n}{test2}\PY{p}{,} \PY{n}{alternative}\PY{o}{=}\PY{l+s+s2}{\PYZdq{}}\PY{l+s+s2}{greater}\PY{l+s+s2}{\PYZdq{}}\PY{p}{)}\PY{p}{)}
\end{Verbatim}
\end{tcolorbox}

    \begin{Verbatim}[commandchars=\\\{\}]
RanksumsResult(statistic=225.13542455896652, pvalue=0.0)
    \end{Verbatim}

    Since we get a p-value of 0 which is lower than 0.05 we reject the null
hypothesis and accept the alternate hypothesis. It implies that intense
activities have higher heart rate than light activities.

    \hypertarget{hypothesis-3}{%
\subsubsection{\texorpdfstring{Hypothesis
3}{Hypothesis 3 }}\label{hypothesis-3}}

\(H_0\)(Null) : The z axis chest acceleration during lying is lower or
same as acceleration of all other activities. \(H_1\)(Alternate) :The z
axis chest acceleration during lying is higher than the acceleration of
all other activities. A linear regression is carried out to test this
hypothesis. \(Y_{t}\) is the chest acceleration along z axis at time t
and \(X_{t}\) is the dummy variable for the activity of lying such that
\[ X_{t} = \begin{cases} 1, & \text{if subject is lying at time t} \\ 0,  & \text{if subject is performing any other activity apart from lying at time t} \end{cases} \]
The regression equation is \(Y_{t} = \beta X_{t} + \alpha\) where
\(\alpha\) is the intercept. It is tested if \(\beta>0\) A \(\beta\)
value of greater than 0 proves that the vertical chest acceleration is
more during lying when compared to other activities. Our hypothesis can
be restated as \$ H\_0: \beta \leq 0 \$ \$ H\_1: \beta > 0

    \begin{tcolorbox}[breakable, size=fbox, boxrule=1pt, pad at break*=1mm,colback=cellbackground, colframe=cellborder]
\prompt{In}{incolor}{189}{\boxspacing}
\begin{Verbatim}[commandchars=\\\{\}]
\PY{n}{test\PYZus{}feat} \PY{o}{=} \PY{l+s+s2}{\PYZdq{}}\PY{l+s+s2}{chest\PYZus{}3D\PYZus{}acceleration\PYZus{}16\PYZus{}z}\PY{l+s+s2}{\PYZdq{}}
\PY{n}{act} \PY{o}{=} \PY{l+s+s2}{\PYZdq{}}\PY{l+s+s2}{lying}\PY{l+s+s2}{\PYZdq{}}
\PY{n}{act\PYZus{}name} \PY{o}{=} \PY{l+s+sa}{f}\PY{l+s+s2}{\PYZdq{}}\PY{l+s+si}{\PYZob{}}\PY{n}{act}\PY{l+s+si}{\PYZcb{}}\PY{l+s+s2}{\PYZus{}or\PYZus{}not}\PY{l+s+s2}{\PYZdq{}}
\PY{n}{test}\PY{p}{[}\PY{n}{act\PYZus{}name}\PY{p}{]} \PY{o}{=} \PY{n}{test}\PY{o}{.}\PY{n}{activity\PYZus{}name}\PY{o}{.}\PY{n}{apply}\PY{p}{(}\PY{k}{lambda} \PY{n}{x}\PY{p}{:} \PY{l+m+mi}{1} \PY{k}{if} \PY{n}{x} \PY{o}{==} \PY{n}{act} \PY{k}{else} \PY{l+m+mi}{0}\PY{p}{)} \PY{c+c1}{\PYZsh{} Generate dummy variable for lying}
\PY{n}{test1} \PY{o}{=} \PY{n}{test}\PY{p}{[}\PY{p}{[}\PY{n}{act\PYZus{}name}\PY{p}{,} \PY{n}{test\PYZus{}feat}\PY{p}{]}\PY{p}{]}\PY{o}{.}\PY{n}{dropna}\PY{p}{(}\PY{p}{)} \PY{c+c1}{\PYZsh{} Drop nan values for regression}
\PY{n}{y} \PY{o}{=} \PY{n}{test1}\PY{p}{[}\PY{n}{test\PYZus{}feat}\PY{p}{]}
\PY{n}{x} \PY{o}{=} \PY{n}{test1}\PY{p}{[}\PY{n}{act\PYZus{}name}\PY{p}{]}
\PY{n}{x} \PY{o}{=} \PY{n}{sm}\PY{o}{.}\PY{n}{add\PYZus{}constant}\PY{p}{(}\PY{n}{x}\PY{p}{)}
\PY{n}{model} \PY{o}{=} \PY{n}{sm}\PY{o}{.}\PY{n}{OLS}\PY{p}{(}\PY{n}{y}\PY{p}{,} \PY{n}{x}\PY{p}{)}
\PY{n}{res} \PY{o}{=} \PY{n}{model}\PY{o}{.}\PY{n}{fit}\PY{p}{(}\PY{p}{)}
\PY{n}{display}\PY{p}{(}\PY{n}{res}\PY{o}{.}\PY{n}{summary}\PY{p}{(}\PY{p}{)}\PY{p}{)}
\end{Verbatim}
\end{tcolorbox}

    
    \begin{verbatim}
<class 'statsmodels.iolib.summary.Summary'>
"""
                                OLS Regression Results                                
======================================================================================
Dep. Variable:     chest_3D_acceleration_16_z   R-squared:                       0.475
Model:                                    OLS   Adj. R-squared:                  0.475
Method:                         Least Squares   F-statistic:                 9.120e+05
Date:                        Thu, 06 Jan 2022   Prob (F-statistic):               0.00
Time:                                16:32:55   Log-Likelihood:            -2.6807e+06
No. Observations:                     1006765   AIC:                         5.361e+06
Df Residuals:                         1006763   BIC:                         5.361e+06
Df Model:                                   1                                         
Covariance Type:                    nonrobust                                         
================================================================================
                   coef    std err          t      P>|t|      [0.025      0.975]
--------------------------------------------------------------------------------
const           -2.7776      0.004   -764.045      0.000      -2.785      -2.771
lying_or_not    11.2204      0.012    955.000      0.000      11.197      11.243
==============================================================================
Omnibus:                   263236.597   Durbin-Watson:                   0.108
Prob(Omnibus):                  0.000   Jarque-Bera (JB):          1577756.632
Skew:                          -1.125   Prob(JB):                         0.00
Kurtosis:                       8.705   Cond. No.                         3.43
==============================================================================

Notes:
[1] Standard Errors assume that the covariance matrix of the errors is correctly specified.
"""
    \end{verbatim}

    
    We get a t-statistic of 955 and a p-value of 0. It implies that we can
safely reject the null hypothesis considering the 5\% confidence
interval and accept the alternate hypothesis that vertical chest
acceleration is indeed more during lying when compared with other
activities as there is some significant evidence for this. Since we get
a p-value of 0 which is lower than 0.05 we reject the null hypothesis
and accept the alternate hypothesis.

    \hypertarget{model-prediction}{%
\subsection{\texorpdfstring{Model
Prediction}{Model Prediction }}\label{model-prediction}}

\hypertarget{data-split}{%
\subsubsection{Data Split}\label{data-split}}

For modelling, the data is split into train ,validation and test
set.Train set is is used to calibrate the model, validation set is used
to test various models and select the best one and the selected model is
finally tested in the test set. The splitting is done by subjects, such
that the train set has subjects (101, 103, 104, 105) the validation set
has subjects (102, 106) and the test set has subjects (107, 108). These
subjects are selected such that the validation and test set has
approximately the same size and that length of training size is
approximately equal to the combined length of testing and validation
set. A model that is trained on one set of subjects, is able to
generalize over other subjects, is highely likely to have achieved good
generalization.

\hypertarget{computing-features}{%
\subsubsection{Computing Features}\label{computing-features}}

Model prediction is performed with the aim of determining the type of
activity being performed.

For this task, it was decided that it is better to make predictions
every 1 second instead of every 0.01 second as the activities are not
likely to change that fast.To accomplish this, many additional features
were computed using a sliding window approach.

A sliding window length of 256 rows or 2.56 seconds was used with a step
size of 100 or 1 second.A window length of 256 was selected since it is
a power of 2, which makes it easier to compute the Fast Fourier
Transform, an algorithm used in the computation of spectral
centroid.Features like mean, median, variance and spectral centroid were
then computed for each window and for each feature.For example, if the
first sliding window was taken from time t to (t+2.56), then the second
sliding window would be taken from time (t+1) to (t+3.56).The original
features were retained to compare their usefulness with the newly
computed features.

These features that are computed over the rolling window are essentially
much more noise resistant, because they rely on taking into account
multiple samples instead of just relying on one highly error-prone
sample.Mean, median and variance of a feature computed over the sliding
window capture essential information about the distribution of the
feature in the period covered by the sliding window.

In addition to the above three features, a frequency domain feature
called spectral centroid was computed. A frequency-domain feature was
considered because it was expected that different activities would have
different rhythms for certain features which could be determined by
figuring out, for instance, their most dominant frequency when that
activity is being performed. Spectral Centroid is computed as,
\[\frac{\sum_{n=0}^{N-1} f(n) x(n)}{\sum_{n=0}^{N-1} x(n)}\] where
\(f(n)\) is the frequency value, while \(x(n)\) is the absolute value of
the Fourier coefficient of that frequency. In this analysis, the
spectral centroid gives a value between 0 and 1. The actual frequency
can be found out by multiplying this value by 256.

A high value of spectral centroid implies the dominance of high
frequency signals in the data and vice versa.

    \begin{tcolorbox}[breakable, size=fbox, boxrule=1pt, pad at break*=1mm,colback=cellbackground, colframe=cellborder]
\prompt{In}{incolor}{69}{\boxspacing}
\begin{Verbatim}[commandchars=\\\{\}]
\PY{c+c1}{\PYZsh{} Function for copying pandas table into clipboard in markdown format}
\PY{n}{copy} \PY{o}{=} \PY{k}{lambda} \PY{n}{x}\PY{p}{:}\PY{n}{pd}\PY{o}{.}\PY{n}{io}\PY{o}{.}\PY{n}{clipboards}\PY{o}{.}\PY{n}{to\PYZus{}clipboard}\PY{p}{(}\PY{n}{x}\PY{o}{.}\PY{n}{to\PYZus{}markdown}\PY{p}{(}\PY{p}{)}\PY{p}{,}\PY{n}{excel}\PY{o}{=}\PY{k+kc}{False}\PY{p}{)} 
\end{Verbatim}
\end{tcolorbox}

    \begin{tcolorbox}[breakable, size=fbox, boxrule=1pt, pad at break*=1mm,colback=cellbackground, colframe=cellborder]
\prompt{In}{incolor}{6}{\boxspacing}
\begin{Verbatim}[commandchars=\\\{\}]
\PY{n}{clean\PYZus{}data} \PY{o}{=} \PY{n}{pd}\PY{o}{.}\PY{n}{read\PYZus{}pickle}\PY{p}{(}\PY{l+s+s2}{\PYZdq{}}\PY{l+s+s2}{clean\PYZus{}act\PYZus{}data.pkl}\PY{l+s+s2}{\PYZdq{}}\PY{p}{)}
\PY{n}{discard} \PY{o}{=} \PY{p}{[}
    \PY{l+s+s2}{\PYZdq{}}\PY{l+s+s2}{activity\PYZus{}id}\PY{l+s+s2}{\PYZdq{}}\PY{p}{,}
    \PY{l+s+s2}{\PYZdq{}}\PY{l+s+s2}{activity}\PY{l+s+s2}{\PYZdq{}}\PY{p}{,}
    \PY{l+s+s2}{\PYZdq{}}\PY{l+s+s2}{activity\PYZus{}name}\PY{l+s+s2}{\PYZdq{}}\PY{p}{,}
    \PY{l+s+s2}{\PYZdq{}}\PY{l+s+s2}{time\PYZus{}stamp}\PY{l+s+s2}{\PYZdq{}}\PY{p}{,}
    \PY{l+s+s2}{\PYZdq{}}\PY{l+s+s2}{id}\PY{l+s+s2}{\PYZdq{}}\PY{p}{,}
    \PY{l+s+s2}{\PYZdq{}}\PY{l+s+s2}{activity\PYZus{}type}\PY{l+s+s2}{\PYZdq{}}\PY{p}{,}
\PY{p}{]}  \PY{c+c1}{\PYZsh{} Columns to exclude from descriptive stat}
\end{Verbatim}
\end{tcolorbox}

    \begin{tcolorbox}[breakable, size=fbox, boxrule=1pt, pad at break*=1mm,colback=cellbackground, colframe=cellborder]
\prompt{In}{incolor}{7}{\boxspacing}
\begin{Verbatim}[commandchars=\\\{\}]
\PY{k}{def} \PY{n+nf}{spectral\PYZus{}centroid}\PY{p}{(}\PY{n}{signal}\PY{p}{)}\PY{p}{:}
    \PY{n}{spectrum} \PY{o}{=} \PY{n}{np}\PY{o}{.}\PY{n}{abs}\PY{p}{(}\PY{n}{np}\PY{o}{.}\PY{n}{fft}\PY{o}{.}\PY{n}{rfft}\PY{p}{(}\PY{n}{signal}\PY{p}{)}\PY{p}{)} \PY{c+c1}{\PYZsh{} Computing absolute value of fourier coefficient}
    \PY{n}{normalized\PYZus{}spectrum} \PY{o}{=} \PY{n}{spectrum} \PY{o}{/} \PY{n}{np}\PY{o}{.}\PY{n}{sum}\PY{p}{(}
        \PY{n}{spectrum}
    \PY{p}{)}  \PY{c+c1}{\PYZsh{} similar to  a probability mass function}
    \PY{n}{normalized\PYZus{}frequencies} \PY{o}{=} \PY{n}{np}\PY{o}{.}\PY{n}{linspace}\PY{p}{(}\PY{l+m+mi}{0}\PY{p}{,} \PY{l+m+mi}{1}\PY{p}{,} \PY{n+nb}{len}\PY{p}{(}\PY{n}{spectrum}\PY{p}{)}\PY{p}{)}
    \PY{n}{spectral\PYZus{}centroid} \PY{o}{=} \PY{n}{np}\PY{o}{.}\PY{n}{sum}\PY{p}{(}\PY{n}{normalized\PYZus{}frequencies} \PY{o}{*} \PY{n}{normalized\PYZus{}spectrum}\PY{p}{)}
    \PY{k}{return} \PY{n}{spectral\PYZus{}centroid}
\end{Verbatim}
\end{tcolorbox}

    \begin{tcolorbox}[breakable, size=fbox, boxrule=1pt, pad at break*=1mm,colback=cellbackground, colframe=cellborder]
\prompt{In}{incolor}{8}{\boxspacing}
\begin{Verbatim}[commandchars=\\\{\}]
\PY{k}{def} \PY{n+nf}{sliding\PYZus{}window\PYZus{}feats}\PY{p}{(}\PY{n}{data}\PY{p}{,} \PY{n}{feats}\PY{p}{,} \PY{n}{win\PYZus{}len}\PY{p}{,} \PY{n}{step}\PY{p}{)}\PY{p}{:}
    \PY{n}{final} \PY{o}{=} \PY{p}{[}\PY{p}{]}
    \PY{n}{i} \PY{o}{=} \PY{l+m+mi}{0}
    \PY{k}{for} \PY{n}{i} \PY{o+ow}{in} \PY{n+nb}{range}\PY{p}{(}\PY{l+m+mi}{0}\PY{p}{,} \PY{n+nb}{len}\PY{p}{(}\PY{n}{data}\PY{p}{)}\PY{p}{,} \PY{l+m+mi}{100}\PY{p}{)}\PY{p}{:}
        \PY{k}{if} \PY{p}{(}\PY{n}{i} \PY{o}{+} \PY{l+m+mi}{256}\PY{p}{)} \PY{o}{\PYZgt{}} \PY{n+nb}{len}\PY{p}{(}\PY{n}{data}\PY{p}{)}\PY{p}{:}
            \PY{k}{break}
        \PY{n}{temp} \PY{o}{=} \PY{n}{data}\PY{o}{.}\PY{n}{iloc}\PY{p}{[}\PY{n}{i} \PY{p}{:} \PY{n}{i} \PY{o}{+} \PY{l+m+mi}{256}\PY{p}{]}
        \PY{n}{temp1} \PY{o}{=} \PY{n}{pd}\PY{o}{.}\PY{n}{DataFrame}\PY{p}{(}\PY{p}{)}
        \PY{k}{for} \PY{n}{feat} \PY{o+ow}{in} \PY{n}{feats}\PY{p}{:}
            \PY{c+c1}{\PYZsh{} Computing sliding window features}
            \PY{n}{temp1}\PY{p}{[}\PY{l+s+sa}{f}\PY{l+s+s2}{\PYZdq{}}\PY{l+s+si}{\PYZob{}}\PY{n}{feat}\PY{l+s+si}{\PYZcb{}}\PY{l+s+s2}{\PYZus{}roll\PYZus{}mean}\PY{l+s+s2}{\PYZdq{}}\PY{p}{]} \PY{o}{=} \PY{p}{[}\PY{n}{temp}\PY{p}{[}\PY{n}{feat}\PY{p}{]}\PY{o}{.}\PY{n}{mean}\PY{p}{(}\PY{p}{)}\PY{p}{]}
            \PY{n}{temp1}\PY{p}{[}\PY{l+s+sa}{f}\PY{l+s+s2}{\PYZdq{}}\PY{l+s+si}{\PYZob{}}\PY{n}{feat}\PY{l+s+si}{\PYZcb{}}\PY{l+s+s2}{\PYZus{}roll\PYZus{}median}\PY{l+s+s2}{\PYZdq{}}\PY{p}{]} \PY{o}{=} \PY{p}{[}\PY{n}{temp}\PY{p}{[}\PY{n}{feat}\PY{p}{]}\PY{o}{.}\PY{n}{median}\PY{p}{(}\PY{p}{)}\PY{p}{]}
            \PY{n}{temp1}\PY{p}{[}\PY{l+s+sa}{f}\PY{l+s+s2}{\PYZdq{}}\PY{l+s+si}{\PYZob{}}\PY{n}{feat}\PY{l+s+si}{\PYZcb{}}\PY{l+s+s2}{\PYZus{}roll\PYZus{}var}\PY{l+s+s2}{\PYZdq{}}\PY{p}{]} \PY{o}{=} \PY{p}{[}\PY{n}{temp}\PY{p}{[}\PY{n}{feat}\PY{p}{]}\PY{o}{.}\PY{n}{var}\PY{p}{(}\PY{p}{)}\PY{p}{]}
            \PY{n}{temp1}\PY{p}{[}\PY{l+s+sa}{f}\PY{l+s+s2}{\PYZdq{}}\PY{l+s+si}{\PYZob{}}\PY{n}{feat}\PY{l+s+si}{\PYZcb{}}\PY{l+s+s2}{\PYZus{}spectral\PYZus{}centroid}\PY{l+s+s2}{\PYZdq{}}\PY{p}{]} \PY{o}{=} \PY{p}{[}\PY{n}{spectral\PYZus{}centroid}\PY{p}{(}\PY{n}{temp}\PY{p}{[}\PY{n}{feat}\PY{p}{]}\PY{p}{)}\PY{p}{]}
        \PY{n}{temp1}\PY{p}{[}\PY{l+s+s2}{\PYZdq{}}\PY{l+s+s2}{time\PYZus{}stamp}\PY{l+s+s2}{\PYZdq{}}\PY{p}{]} \PY{o}{=} \PY{p}{[}\PY{n+nb}{list}\PY{p}{(}\PY{n}{temp}\PY{o}{.}\PY{n}{time\PYZus{}stamp}\PY{o}{.}\PY{n}{values}\PY{p}{)}\PY{p}{[}\PY{o}{\PYZhy{}}\PY{l+m+mi}{1}\PY{p}{]}\PY{p}{]}
        \PY{n}{temp1}\PY{p}{[}\PY{n}{feats}\PY{p}{]} \PY{o}{=} \PY{p}{[}\PY{n}{temp}\PY{p}{[}\PY{n}{feats}\PY{p}{]}\PY{o}{.}\PY{n}{iloc}\PY{p}{[}\PY{o}{\PYZhy{}}\PY{l+m+mi}{1}\PY{p}{]}\PY{p}{]}
        \PY{n}{temp1}\PY{p}{[}\PY{l+s+s2}{\PYZdq{}}\PY{l+s+s2}{activity\PYZus{}name}\PY{l+s+s2}{\PYZdq{}}\PY{p}{]} \PY{o}{=} \PY{p}{[}\PY{n}{temp}\PY{p}{[}\PY{l+s+s2}{\PYZdq{}}\PY{l+s+s2}{activity\PYZus{}name}\PY{l+s+s2}{\PYZdq{}}\PY{p}{]}\PY{o}{.}\PY{n}{iloc}\PY{p}{[}\PY{o}{\PYZhy{}}\PY{l+m+mi}{1}\PY{p}{]}\PY{p}{]}
        \PY{n}{temp1}\PY{p}{[}\PY{l+s+s2}{\PYZdq{}}\PY{l+s+s2}{activity\PYZus{}type}\PY{l+s+s2}{\PYZdq{}}\PY{p}{]} \PY{o}{=} \PY{p}{[}\PY{n}{temp}\PY{p}{[}\PY{l+s+s2}{\PYZdq{}}\PY{l+s+s2}{activity\PYZus{}type}\PY{l+s+s2}{\PYZdq{}}\PY{p}{]}\PY{o}{.}\PY{n}{iloc}\PY{p}{[}\PY{o}{\PYZhy{}}\PY{l+m+mi}{1}\PY{p}{]}\PY{p}{]}
        \PY{n}{final}\PY{o}{.}\PY{n}{append}\PY{p}{(}\PY{n}{temp1}\PY{p}{)}
    \PY{n}{final\PYZus{}data} \PY{o}{=} \PY{n}{pd}\PY{o}{.}\PY{n}{concat}\PY{p}{(}\PY{n}{final}\PY{p}{)}
    \PY{k}{return} \PY{n}{final\PYZus{}data}
\end{Verbatim}
\end{tcolorbox}

    \begin{tcolorbox}[breakable, size=fbox, boxrule=1pt, pad at break*=1mm,colback=cellbackground, colframe=cellborder]
\prompt{In}{incolor}{9}{\boxspacing}
\begin{Verbatim}[commandchars=\\\{\}]
\PY{k}{class} \PY{n+nc}{modelling}\PY{p}{:}
    \PY{k}{def} \PY{n+nf+fm}{\PYZus{}\PYZus{}init\PYZus{}\PYZus{}}\PY{p}{(}
        \PY{n+nb+bp}{self}\PY{p}{,}
        \PY{n}{clean\PYZus{}data}\PY{p}{,}
        \PY{n}{features}\PY{p}{,}
        \PY{n}{train\PYZus{}subjects}\PY{o}{=}\PY{p}{[}\PY{l+m+mi}{101}\PY{p}{,} \PY{l+m+mi}{103}\PY{p}{,} \PY{l+m+mi}{104}\PY{p}{,} \PY{l+m+mi}{105}\PY{p}{]}\PY{p}{,}
        \PY{n}{val\PYZus{}subjects}\PY{o}{=}\PY{p}{[}\PY{l+m+mi}{102}\PY{p}{,} \PY{l+m+mi}{106}\PY{p}{]}\PY{p}{,}
        \PY{n}{test\PYZus{}subjects}\PY{o}{=}\PY{p}{[}\PY{l+m+mi}{107}\PY{p}{,} \PY{l+m+mi}{108}\PY{p}{]}\PY{p}{,}
    \PY{p}{)}\PY{p}{:}
        \PY{c+c1}{\PYZsh{} Initializing variables}
        \PY{n+nb+bp}{self}\PY{o}{.}\PY{n}{clean\PYZus{}data} \PY{o}{=} \PY{n}{clean\PYZus{}data}
        \PY{n+nb+bp}{self}\PY{o}{.}\PY{n}{train\PYZus{}subjects} \PY{o}{=} \PY{n}{train\PYZus{}subjects}
        \PY{n+nb+bp}{self}\PY{o}{.}\PY{n}{val\PYZus{}subjects} \PY{o}{=} \PY{n}{val\PYZus{}subjects}
        \PY{n+nb+bp}{self}\PY{o}{.}\PY{n}{test\PYZus{}subjects} \PY{o}{=} \PY{n}{test\PYZus{}subjects}
        \PY{n+nb+bp}{self}\PY{o}{.}\PY{n}{features} \PY{o}{=} \PY{n}{features}
    \PY{k}{def} \PY{n+nf}{split\PYZus{}input\PYZus{}data}\PY{p}{(}\PY{n+nb+bp}{self}\PY{p}{)}\PY{p}{:}
        \PY{c+c1}{\PYZsh{} Splitting input features into train,test and val}
        \PY{n}{train} \PY{o}{=} \PY{n+nb+bp}{self}\PY{o}{.}\PY{n}{clean\PYZus{}data}\PY{p}{[}\PY{n+nb+bp}{self}\PY{o}{.}\PY{n}{clean\PYZus{}data}\PY{o}{.}\PY{n}{id}\PY{o}{.}\PY{n}{isin}\PY{p}{(}\PY{n+nb+bp}{self}\PY{o}{.}\PY{n}{train\PYZus{}subjects}\PY{p}{)}\PY{p}{]}
        \PY{n}{val} \PY{o}{=} \PY{n+nb+bp}{self}\PY{o}{.}\PY{n}{clean\PYZus{}data}\PY{p}{[}\PY{n+nb+bp}{self}\PY{o}{.}\PY{n}{clean\PYZus{}data}\PY{o}{.}\PY{n}{id}\PY{o}{.}\PY{n}{isin}\PY{p}{(}\PY{n+nb+bp}{self}\PY{o}{.}\PY{n}{val\PYZus{}subjects}\PY{p}{)}\PY{p}{]}
        \PY{n}{test} \PY{o}{=} \PY{n+nb+bp}{self}\PY{o}{.}\PY{n}{clean\PYZus{}data}\PY{p}{[}\PY{n+nb+bp}{self}\PY{o}{.}\PY{n}{clean\PYZus{}data}\PY{o}{.}\PY{n}{id}\PY{o}{.}\PY{n}{isin}\PY{p}{(}\PY{n+nb+bp}{self}\PY{o}{.}\PY{n}{test\PYZus{}subjects}\PY{p}{)}\PY{p}{]}
        \PY{n}{x\PYZus{}train} \PY{o}{=} \PY{n}{train}\PY{p}{[}\PY{n+nb+bp}{self}\PY{o}{.}\PY{n}{features}\PY{p}{]}
        \PY{n}{x\PYZus{}val} \PY{o}{=} \PY{n}{val}\PY{p}{[}\PY{n+nb+bp}{self}\PY{o}{.}\PY{n}{features}\PY{p}{]}
        \PY{n}{x\PYZus{}test} \PY{o}{=} \PY{n}{test}\PY{p}{[}\PY{n+nb+bp}{self}\PY{o}{.}\PY{n}{features}\PY{p}{]}
        \PY{k}{return} \PY{n}{train}\PY{p}{,} \PY{n}{val}\PY{p}{,} \PY{n}{test}\PY{p}{,} \PY{n}{x\PYZus{}train}\PY{p}{,} \PY{n}{x\PYZus{}val}\PY{p}{,} \PY{n}{x\PYZus{}test}
    \PY{k}{def} \PY{n+nf}{split\PYZus{}one\PYZus{}act}\PY{p}{(}\PY{n+nb+bp}{self}\PY{p}{,} \PY{n}{activity}\PY{p}{)}\PY{p}{:}
        \PY{c+c1}{\PYZsh{} Function to give input ouput matrix for one activity}
        \PY{n}{train}\PY{p}{,} \PY{n}{val}\PY{p}{,} \PY{n}{test}\PY{p}{,} \PY{n}{x\PYZus{}train}\PY{p}{,} \PY{n}{x\PYZus{}val}\PY{p}{,} \PY{n}{x\PYZus{}test} \PY{o}{=} \PY{n+nb+bp}{self}\PY{o}{.}\PY{n}{split\PYZus{}input\PYZus{}data}\PY{p}{(}\PY{p}{)}
        \PY{n}{one\PYZus{}hot\PYZus{}label} \PY{o}{=} \PY{k}{lambda} \PY{n}{x}\PY{p}{:} \PY{l+m+mi}{1} \PY{k}{if} \PY{n}{x} \PY{o}{==} \PY{n}{activity} \PY{k}{else} \PY{l+m+mi}{0} \PY{c+c1}{\PYZsh{} generate dummy variable for one activity}
        \PY{n}{y\PYZus{}train} \PY{o}{=} \PY{n}{train}\PY{o}{.}\PY{n}{activity\PYZus{}name}\PY{o}{.}\PY{n}{apply}\PY{p}{(}\PY{k}{lambda} \PY{n}{x}\PY{p}{:} \PY{n}{one\PYZus{}hot\PYZus{}label}\PY{p}{(}\PY{n}{x}\PY{p}{)}\PY{p}{)}
        \PY{n}{y\PYZus{}val} \PY{o}{=} \PY{n}{val}\PY{o}{.}\PY{n}{activity\PYZus{}name}\PY{o}{.}\PY{n}{apply}\PY{p}{(}\PY{k}{lambda} \PY{n}{x}\PY{p}{:} \PY{n}{one\PYZus{}hot\PYZus{}label}\PY{p}{(}\PY{n}{x}\PY{p}{)}\PY{p}{)}
        \PY{n}{y\PYZus{}test} \PY{o}{=} \PY{n}{test}\PY{o}{.}\PY{n}{activity\PYZus{}name}\PY{o}{.}\PY{n}{apply}\PY{p}{(}\PY{k}{lambda} \PY{n}{x}\PY{p}{:} \PY{n}{one\PYZus{}hot\PYZus{}label}\PY{p}{(}\PY{n}{x}\PY{p}{)}\PY{p}{)}
        \PY{k}{return} \PY{n}{x\PYZus{}train}\PY{p}{,} \PY{n}{x\PYZus{}val}\PY{p}{,} \PY{n}{x\PYZus{}test}\PY{p}{,} \PY{n}{y\PYZus{}train}\PY{p}{,} \PY{n}{y\PYZus{}val}\PY{p}{,} \PY{n}{y\PYZus{}test}
    \PY{k}{def} \PY{n+nf}{train\PYZus{}test\PYZus{}split\PYZus{}actname}\PY{p}{(}\PY{n+nb+bp}{self}\PY{p}{)}\PY{p}{:}
        \PY{c+c1}{\PYZsh{} Function to give input ouput matrix for all 12 activies}

        \PY{n}{le} \PY{o}{=} \PY{n}{preprocessing}\PY{o}{.}\PY{n}{LabelEncoder}\PY{p}{(}\PY{p}{)}
        \PY{n}{train}\PY{p}{,} \PY{n}{val}\PY{p}{,} \PY{n}{test}\PY{p}{,} \PY{n}{x\PYZus{}train}\PY{p}{,} \PY{n}{x\PYZus{}val}\PY{p}{,} \PY{n}{x\PYZus{}test} \PY{o}{=} \PY{n+nb+bp}{self}\PY{o}{.}\PY{n}{split\PYZus{}input\PYZus{}data}\PY{p}{(}\PY{p}{)}
        \PY{n}{y\PYZus{}train} \PY{o}{=} \PY{n}{le}\PY{o}{.}\PY{n}{fit\PYZus{}transform}\PY{p}{(}\PY{n}{train}\PY{o}{.}\PY{n}{activity\PYZus{}name}\PY{p}{)}
        \PY{n}{y\PYZus{}val} \PY{o}{=} \PY{n}{le}\PY{o}{.}\PY{n}{fit\PYZus{}transform}\PY{p}{(}\PY{n}{val}\PY{o}{.}\PY{n}{activity\PYZus{}name}\PY{p}{)}
        \PY{n}{y\PYZus{}test} \PY{o}{=} \PY{n}{le}\PY{o}{.}\PY{n}{fit\PYZus{}transform}\PY{p}{(}\PY{n}{test}\PY{o}{.}\PY{n}{activity\PYZus{}name}\PY{p}{)}
        \PY{k}{return} \PY{n}{x\PYZus{}train}\PY{p}{,} \PY{n}{x\PYZus{}val}\PY{p}{,} \PY{n}{x\PYZus{}test}\PY{p}{,} \PY{n}{y\PYZus{}train}\PY{p}{,} \PY{n}{y\PYZus{}val}\PY{p}{,} \PY{n}{y\PYZus{}test}\PY{p}{,} \PY{n}{le}
\end{Verbatim}
\end{tcolorbox}

    \begin{tcolorbox}[breakable, size=fbox, boxrule=1pt, pad at break*=1mm,colback=cellbackground, colframe=cellborder]
\prompt{In}{incolor}{151}{\boxspacing}
\begin{Verbatim}[commandchars=\\\{\}]
\PY{k}{def} \PY{n+nf}{final\PYZus{}sliding\PYZus{}window}\PY{p}{(}\PY{n}{clean\PYZus{}data}\PY{p}{)}\PY{p}{:}
    \PY{c+c1}{\PYZsh{} Function for generating sliding window}
    \PY{n}{feats} \PY{o}{=} \PY{p}{[}\PY{n}{i} \PY{k}{for} \PY{n}{i} \PY{o+ow}{in} \PY{n}{clean\PYZus{}data}\PY{o}{.}\PY{n}{columns} \PY{k}{if} \PY{n}{i} \PY{o+ow}{not} \PY{o+ow}{in} \PY{n}{discard}\PY{p}{]}
    \PY{n}{final} \PY{o}{=} \PY{p}{[}\PY{p}{]}
    \PY{k}{for} \PY{n}{i} \PY{o+ow}{in} \PY{n}{clean\PYZus{}data}\PY{o}{.}\PY{n}{id}\PY{o}{.}\PY{n}{unique}\PY{p}{(}\PY{p}{)}\PY{p}{:}
        \PY{n}{temp} \PY{o}{=} \PY{n}{clean\PYZus{}data}\PY{p}{[}\PY{n}{clean\PYZus{}data}\PY{o}{.}\PY{n}{id} \PY{o}{==} \PY{n}{i}\PY{p}{]} \PY{c+c1}{\PYZsh{} }
        \PY{n}{temp} \PY{o}{=} \PY{n}{sliding\PYZus{}window\PYZus{}feats}\PY{p}{(}\PY{n}{temp}\PY{p}{,} \PY{n}{feats}\PY{p}{,} \PY{l+m+mi}{256}\PY{p}{,} \PY{l+m+mi}{100}\PY{p}{)}
        \PY{n}{temp}\PY{p}{[}\PY{l+s+s2}{\PYZdq{}}\PY{l+s+s2}{id}\PY{l+s+s2}{\PYZdq{}}\PY{p}{]} \PY{o}{=} \PY{p}{[}\PY{n}{i}\PY{p}{]} \PY{o}{*} \PY{n+nb}{len}\PY{p}{(}\PY{n}{temp}\PY{p}{)}
        \PY{n}{final}\PY{o}{.}\PY{n}{append}\PY{p}{(}\PY{n}{temp}\PY{p}{)}
    \PY{n}{clean\PYZus{}data\PYZus{}feats} \PY{o}{=} \PY{n}{pd}\PY{o}{.}\PY{n}{concat}\PY{p}{(}\PY{n}{final}\PY{p}{)}
    \PY{n}{clean\PYZus{}data\PYZus{}feats}\PY{o}{.}\PY{n}{to\PYZus{}pickle}\PY{p}{(}\PY{l+s+s2}{\PYZdq{}}\PY{l+s+s2}{activity\PYZus{}short\PYZus{}data.pkl}\PY{l+s+s2}{\PYZdq{}}\PY{p}{)}
    \PY{k}{return} \PY{n}{clean\PYZus{}data\PYZus{}feats}
\end{Verbatim}
\end{tcolorbox}

    \textbf{Warning}: This cell takes a very long time to run.It is advised
to use a debugger to run it line by line to check it.

    \begin{tcolorbox}[breakable, size=fbox, boxrule=1pt, pad at break*=1mm,colback=cellbackground, colframe=cellborder]
\prompt{In}{incolor}{ }{\boxspacing}
\begin{Verbatim}[commandchars=\\\{\}]
\PY{n}{final\PYZus{}sliding\PYZus{}window}\PY{p}{(}\PY{n}{clean\PYZus{}data}\PY{p}{)}
\end{Verbatim}
\end{tcolorbox}

    \begin{tcolorbox}[breakable, size=fbox, boxrule=1pt, pad at break*=1mm,colback=cellbackground, colframe=cellborder]
\prompt{In}{incolor}{17}{\boxspacing}
\begin{Verbatim}[commandchars=\\\{\}]
\PY{n}{clean\PYZus{}data\PYZus{}feats} \PY{o}{=} \PY{n}{pd}\PY{o}{.}\PY{n}{read\PYZus{}pickle}\PY{p}{(}\PY{l+s+s2}{\PYZdq{}}\PY{l+s+s2}{activity\PYZus{}short\PYZus{}data.pkl}\PY{l+s+s2}{\PYZdq{}}\PY{p}{)} \PY{c+c1}{\PYZsh{} Load saved clean data}
\PY{n}{features} \PY{o}{=} \PY{p}{[}\PY{n}{i} \PY{k}{for} \PY{n}{i} \PY{o+ow}{in} \PY{n}{clean\PYZus{}data\PYZus{}feats}\PY{o}{.}\PY{n}{columns} \PY{k}{if} \PY{n}{i} \PY{o+ow}{not} \PY{o+ow}{in} \PY{n}{discard}\PY{p}{]}
\PY{n}{model} \PY{o}{=} \PY{n}{modelling}\PY{p}{(}\PY{n}{clean\PYZus{}data\PYZus{}feats}\PY{p}{,} \PY{n}{features}\PY{p}{)}
\end{Verbatim}
\end{tcolorbox}

    \begin{tcolorbox}[breakable, size=fbox, boxrule=1pt, pad at break*=1mm,colback=cellbackground, colframe=cellborder]
\prompt{In}{incolor}{18}{\boxspacing}
\begin{Verbatim}[commandchars=\\\{\}]
\PY{p}{(}
    \PY{n}{x\PYZus{}train}\PY{p}{,}
    \PY{n}{x\PYZus{}val}\PY{p}{,}
    \PY{n}{x\PYZus{}test}\PY{p}{,}
    \PY{n}{y\PYZus{}train}\PY{p}{,}
    \PY{n}{y\PYZus{}val}\PY{p}{,}
    \PY{n}{y\PYZus{}test}\PY{p}{,}
    \PY{n}{le}\PY{p}{,}
\PY{p}{)} \PY{o}{=} \PY{n}{model}\PY{o}{.}\PY{n}{train\PYZus{}test\PYZus{}split\PYZus{}actname}\PY{p}{(}\PY{p}{)} \PY{c+c1}{\PYZsh{} generate input ouput matrix}
\end{Verbatim}
\end{tcolorbox}

    \begin{tcolorbox}[breakable, size=fbox, boxrule=1pt, pad at break*=1mm,colback=cellbackground, colframe=cellborder]
\prompt{In}{incolor}{20}{\boxspacing}
\begin{Verbatim}[commandchars=\\\{\}]
\PY{n}{x\PYZus{}train\PYZus{}labels} \PY{o}{=} \PY{n}{pd}\PY{o}{.}\PY{n}{DataFrame}\PY{p}{(}\PY{p}{)}
\PY{n}{x\PYZus{}train\PYZus{}labels}\PY{p}{[}\PY{l+s+s2}{\PYZdq{}}\PY{l+s+s2}{activity\PYZus{}name}\PY{l+s+s2}{\PYZdq{}}\PY{p}{]} \PY{o}{=} \PY{n}{le}\PY{o}{.}\PY{n}{inverse\PYZus{}transform}\PY{p}{(}\PY{n}{y\PYZus{}train}\PY{p}{)}
\end{Verbatim}
\end{tcolorbox}

    \hypertarget{feature-selection-process}{%
\subsubsection{Feature Selection
Process}\label{feature-selection-process}}

\begin{enumerate}
\def\labelenumi{\arabic{enumi}.}
\item
  First, Clustering is performed for each feature.V-measure is used to
  determine how good the cluster is.If ncluster is the number of
  clusters chooses for clustering, then clustering is performed using
  different values of nclusters and all of them are evaluated based on
  v-measure.Minimum size choosen is 12 as that is the number of
  activities and ideally every activity should be associated with one
  cluster. The 12 activity names are considered as the class labels.
\item
  Same cluster size is used for clustering each feature.
\item
  The average of v-measure score for all the feature is taken to
  determine the final v-measure score.
\item
  The v-measure score for different cluster sizes give almost the same
  score, hence the cluster size of 100 is choosen for each feature.
\item
  The probability of an activity i given a cluster j is computed as
  \[p_{ij} = \frac{n(i \cap j)}{n(j)}\]
\end{enumerate}

,where \(n(i \cap j)\) is the count of occurence of activity i and j
together and \(n(j)\) is the count of occurence of cluster j

\begin{enumerate}
\def\labelenumi{\arabic{enumi}.}
\setcounter{enumi}{5}
\item
  For each feature a precision score is calculated activty i such that

  \[ P_{ik} = \frac{\sum_{j=1}^{N}p_{ij}C_{ij}}{\sum_{j=1}^{N}C_{ij}} \]

  , where \(C_{ij}\) is the number of rows of cluster j present in
  activity i. A higher value of precision for an activity implies that
  there are many clusters that have samples mostly for this activity.
\item
  The precision score gives us a good idea of how good a feature will be
  in predicting a particular activity.
\end{enumerate}

    \begin{tcolorbox}[breakable, size=fbox, boxrule=1pt, pad at break*=1mm,colback=cellbackground, colframe=cellborder]
\prompt{In}{incolor}{10}{\boxspacing}
\begin{Verbatim}[commandchars=\\\{\}]
\PY{k}{def} \PY{n+nf}{precision}\PY{p}{(}\PY{n}{df}\PY{p}{)}\PY{p}{:}
    \PY{c+c1}{\PYZsh{} Function for computing precision}
    \PY{n}{df}\PY{o}{.}\PY{n}{columns} \PY{o}{=} \PY{p}{[}\PY{l+s+s2}{\PYZdq{}}\PY{l+s+s2}{activity}\PY{l+s+s2}{\PYZdq{}}\PY{p}{,} \PY{l+s+s2}{\PYZdq{}}\PY{l+s+s2}{labels}\PY{l+s+s2}{\PYZdq{}}\PY{p}{]}
    \PY{n}{act\PYZus{}precision} \PY{o}{=} \PY{n+nb}{dict}\PY{p}{(}\PY{p}{)}
    \PY{k}{for} \PY{n}{act} \PY{o+ow}{in} \PY{n}{df}\PY{o}{.}\PY{n}{activity}\PY{o}{.}\PY{n}{unique}\PY{p}{(}\PY{p}{)}\PY{p}{:}
        \PY{n}{num} \PY{o}{=} \PY{l+m+mi}{0}
        \PY{n}{denom} \PY{o}{=} \PY{l+m+mi}{0}
        \PY{n}{df\PYZus{}act} \PY{o}{=} \PY{n}{df}\PY{p}{[}\PY{n}{df}\PY{o}{.}\PY{n}{activity} \PY{o}{==} \PY{n}{act}\PY{p}{]}
        \PY{n}{c\PYZus{}lab} \PY{o}{=} \PY{n}{df\PYZus{}act}\PY{o}{.}\PY{n}{labels}\PY{o}{.}\PY{n}{value\PYZus{}counts}\PY{p}{(}\PY{p}{)}
        \PY{k}{for} \PY{n}{lab} \PY{o+ow}{in} \PY{n}{df\PYZus{}act}\PY{o}{.}\PY{n}{labels}\PY{o}{.}\PY{n}{unique}\PY{p}{(}\PY{p}{)}\PY{p}{:}
            \PY{n}{clust\PYZus{}prob} \PY{o}{=} \PY{n+nb}{len}\PY{p}{(}\PY{n}{df}\PY{p}{[}\PY{p}{(}\PY{n}{df}\PY{o}{.}\PY{n}{activity} \PY{o}{==} \PY{n}{act}\PY{p}{)} \PY{o}{\PYZam{}} \PY{p}{(}\PY{n}{df}\PY{o}{.}\PY{n}{labels} \PY{o}{==} \PY{n}{lab}\PY{p}{)}\PY{p}{]}\PY{p}{)} \PY{o}{/} \PY{n+nb}{len}\PY{p}{(}
                \PY{n}{df}\PY{p}{[}\PY{n}{df}\PY{o}{.}\PY{n}{labels} \PY{o}{==} \PY{n}{lab}\PY{p}{]}
            \PY{p}{)}
            \PY{n}{num} \PY{o}{=} \PY{n}{num} \PY{o}{+} \PY{n}{clust\PYZus{}prob} \PY{o}{*} \PY{n}{c\PYZus{}lab}\PY{p}{[}\PY{n}{lab}\PY{p}{]}
            \PY{n}{denom} \PY{o}{=} \PY{n}{denom} \PY{o}{+} \PY{n}{c\PYZus{}lab}\PY{p}{[}\PY{n}{lab}\PY{p}{]}
        \PY{n}{act\PYZus{}precision}\PY{p}{[}\PY{n}{act}\PY{p}{]} \PY{o}{=} \PY{n}{num} \PY{o}{/} \PY{n}{denom}
    \PY{k}{return} \PY{n}{act\PYZus{}precision}
\end{Verbatim}
\end{tcolorbox}

    \begin{tcolorbox}[breakable, size=fbox, boxrule=1pt, pad at break*=1mm,colback=cellbackground, colframe=cellborder]
\prompt{In}{incolor}{11}{\boxspacing}
\begin{Verbatim}[commandchars=\\\{\}]
\PY{k}{def} \PY{n+nf}{best\PYZus{}cluster}\PY{p}{(}\PY{p}{)}\PY{p}{:}
    \PY{c+c1}{\PYZsh{} Function for determining  best cluster}
    \PY{n}{v\PYZus{}measure} \PY{o}{=} \PY{n+nb}{dict}\PY{p}{(}\PY{p}{)}
    \PY{k}{for} \PY{n}{nclust} \PY{o+ow}{in} \PY{n+nb}{range}\PY{p}{(}\PY{l+m+mi}{12}\PY{p}{,} \PY{l+m+mi}{112}\PY{p}{,} \PY{l+m+mi}{5}\PY{p}{)}\PY{p}{:}
        \PY{n}{clust\PYZus{}vmeasure} \PY{o}{=} \PY{p}{[}\PY{p}{]}
        \PY{k}{for} \PY{n}{col} \PY{o+ow}{in} \PY{n}{x\PYZus{}train}\PY{o}{.}\PY{n}{columns}\PY{p}{:}
            \PY{n}{clust} \PY{o}{=} \PY{n}{cluster}\PY{o}{.}\PY{n}{KMeans}\PY{p}{(}\PY{n}{init}\PY{o}{=}\PY{l+s+s2}{\PYZdq{}}\PY{l+s+s2}{random}\PY{l+s+s2}{\PYZdq{}}\PY{p}{,} \PY{n}{random\PYZus{}state}\PY{o}{=}\PY{l+m+mi}{0}\PY{p}{,} \PY{n}{n\PYZus{}clusters}\PY{o}{=}\PY{n}{nclust}\PY{p}{)}
            \PY{n}{clust}\PY{o}{.}\PY{n}{fit}\PY{p}{(}\PY{n}{x\PYZus{}train}\PY{p}{[}\PY{p}{[}\PY{n}{col}\PY{p}{]}\PY{p}{]}\PY{p}{)}
            \PY{n}{x\PYZus{}train\PYZus{}labels}\PY{p}{[}\PY{l+s+sa}{f}\PY{l+s+s2}{\PYZdq{}}\PY{l+s+si}{\PYZob{}}\PY{n}{col}\PY{l+s+si}{\PYZcb{}}\PY{l+s+s2}{\PYZus{}label}\PY{l+s+s2}{\PYZdq{}}\PY{p}{]} \PY{o}{=} \PY{n}{clust}\PY{o}{.}\PY{n}{predict}\PY{p}{(}\PY{n}{x\PYZus{}train}\PY{p}{[}\PY{p}{[}\PY{n}{col}\PY{p}{]}\PY{p}{]}\PY{p}{)}
            \PY{n}{clust\PYZus{}vmeasure}\PY{o}{.}\PY{n}{append}\PY{p}{(}
                \PY{n}{v\PYZus{}measure\PYZus{}score}\PY{p}{(}\PY{n}{y\PYZus{}train}\PY{p}{,} \PY{n}{x\PYZus{}train\PYZus{}labels}\PY{p}{[}\PY{l+s+sa}{f}\PY{l+s+s2}{\PYZdq{}}\PY{l+s+si}{\PYZob{}}\PY{n}{col}\PY{l+s+si}{\PYZcb{}}\PY{l+s+s2}{\PYZus{}label}\PY{l+s+s2}{\PYZdq{}}\PY{p}{]}\PY{p}{)}
            \PY{p}{)}
        \PY{n}{v\PYZus{}measure}\PY{p}{[}\PY{n}{nclust}\PY{p}{]} \PY{o}{=} \PY{p}{[}\PY{n}{np}\PY{o}{.}\PY{n}{array}\PY{p}{(}\PY{n}{clust\PYZus{}vmeasure}\PY{p}{)}\PY{o}{.}\PY{n}{mean}\PY{p}{(}\PY{p}{)}\PY{p}{]}
    \PY{n}{nclust\PYZus{}max} \PY{o}{=} \PY{n+nb}{max}\PY{p}{(}\PY{n}{v\PYZus{}measure}\PY{p}{,} \PY{n}{key}\PY{o}{=}\PY{n}{v\PYZus{}measure}\PY{o}{.}\PY{n}{get}\PY{p}{)}
    \PY{n+nb}{print}\PY{p}{(}\PY{l+s+sa}{f}\PY{l+s+s2}{\PYZdq{}}\PY{l+s+s2}{best cluster size : }\PY{l+s+si}{\PYZob{}}\PY{n}{nclust\PYZus{}max}\PY{l+s+si}{\PYZcb{}}\PY{l+s+s2}{\PYZdq{}}\PY{p}{)}
    \PY{k}{return} \PY{n}{v\PYZus{}measure}
\end{Verbatim}
\end{tcolorbox}

    \textbf{Warning}: The cell below takes a very long time to run. A
debugger can be used to check it by executing the function line by line.

    \begin{tcolorbox}[breakable, size=fbox, boxrule=1pt, pad at break*=1mm,colback=cellbackground, colframe=cellborder]
\prompt{In}{incolor}{ }{\boxspacing}
\begin{Verbatim}[commandchars=\\\{\}]
\PY{n}{vm} \PY{o}{=} \PY{n}{pd}\PY{o}{.}\PY{n}{DataFrame}\PY{p}{(}\PY{n}{best\PYZus{}cluster}\PY{p}{(}\PY{p}{)}\PY{p}{)}
\PY{n}{vm}\PY{o}{.}\PY{n}{to\PYZus{}pickle}\PY{p}{(}\PY{l+s+s2}{\PYZdq{}}\PY{l+s+s2}{v\PYZus{}measure.pkl}\PY{l+s+s2}{\PYZdq{}}\PY{p}{)}
\end{Verbatim}
\end{tcolorbox}

    V measure of different cluster size

    \begin{tcolorbox}[breakable, size=fbox, boxrule=1pt, pad at break*=1mm,colback=cellbackground, colframe=cellborder]
\prompt{In}{incolor}{105}{\boxspacing}
\begin{Verbatim}[commandchars=\\\{\}]
\PY{n}{vm} \PY{o}{=} \PY{n}{pd}\PY{o}{.}\PY{n}{read\PYZus{}pickle}\PY{p}{(}\PY{l+s+s2}{\PYZdq{}}\PY{l+s+s2}{v\PYZus{}measure.pkl}\PY{l+s+s2}{\PYZdq{}}\PY{p}{)}
\PY{n+nb}{print}\PY{p}{(}\PY{l+s+s2}{\PYZdq{}}\PY{l+s+s2}{A condensed view of average v\PYZhy{}measure of cluster of different sizes}\PY{l+s+s2}{\PYZdq{}}\PY{p}{)}
\PY{n}{display}\PY{p}{(}\PY{n}{vm}\PY{p}{[}\PY{n}{vm}\PY{o}{.}\PY{n}{columns}\PY{p}{[}\PY{l+m+mi}{0}\PY{p}{:}\PY{l+m+mi}{9}\PY{p}{]}\PY{p}{]}\PY{p}{)}
\PY{c+c1}{\PYZsh{} Not much difference found so using 100 clusters}
\end{Verbatim}
\end{tcolorbox}

    \begin{Verbatim}[commandchars=\\\{\}]
A condensed view of average v-measure of cluster of different sizes
    \end{Verbatim}

    
    \begin{verbatim}
         12        17        22        27        32       37        42  \
0  0.217367  0.218729  0.217423  0.215188  0.212477  0.20998  0.207591   

         47        52  
0  0.205433  0.203584  
    \end{verbatim}

    
    \begin{tcolorbox}[breakable, size=fbox, boxrule=1pt, pad at break*=1mm,colback=cellbackground, colframe=cellborder]
\prompt{In}{incolor}{14}{\boxspacing}
\begin{Verbatim}[commandchars=\\\{\}]
\PY{k}{def} \PY{n+nf}{activity\PYZus{}precision}\PY{p}{(}\PY{p}{)}\PY{p}{:}
    \PY{c+c1}{\PYZsh{} Function for computing precision of each activity}
    \PY{n}{label\PYZus{}act\PYZus{}precision} \PY{o}{=} \PY{n+nb}{dict}\PY{p}{(}\PY{p}{)}
    \PY{k}{for} \PY{n}{i} \PY{o+ow}{in} \PY{n}{x\PYZus{}train}\PY{o}{.}\PY{n}{columns}\PY{p}{:}
        \PY{n}{clust} \PY{o}{=} \PY{n}{cluster}\PY{o}{.}\PY{n}{KMeans}\PY{p}{(}
            \PY{n}{init}\PY{o}{=}\PY{l+s+s2}{\PYZdq{}}\PY{l+s+s2}{random}\PY{l+s+s2}{\PYZdq{}}\PY{p}{,} \PY{n}{random\PYZus{}state}\PY{o}{=}\PY{l+m+mi}{0}\PY{p}{,} \PY{n}{n\PYZus{}clusters}\PY{o}{=}\PY{l+m+mi}{1000}
        \PY{p}{)}
        \PY{n}{clust}\PY{o}{.}\PY{n}{fit}\PY{p}{(}\PY{n}{x\PYZus{}train}\PY{p}{[}\PY{p}{[}\PY{n}{i}\PY{p}{]}\PY{p}{]}\PY{p}{)}
        \PY{n}{x\PYZus{}train\PYZus{}labels}\PY{p}{[}\PY{l+s+sa}{f}\PY{l+s+s2}{\PYZdq{}}\PY{l+s+si}{\PYZob{}}\PY{n}{i}\PY{l+s+si}{\PYZcb{}}\PY{l+s+s2}{\PYZus{}label}\PY{l+s+s2}{\PYZdq{}}\PY{p}{]} \PY{o}{=} \PY{n}{clust}\PY{o}{.}\PY{n}{predict}\PY{p}{(}\PY{n}{x\PYZus{}train}\PY{p}{[}\PY{p}{[}\PY{n}{i}\PY{p}{]}\PY{p}{]}\PY{p}{)}
        \PY{n}{label\PYZus{}act\PYZus{}precision}\PY{p}{[}\PY{n}{i}\PY{p}{]} \PY{o}{=} \PY{n}{precision}\PY{p}{(}
            \PY{n}{x\PYZus{}train\PYZus{}labels}\PY{p}{[}\PY{p}{[}\PY{l+s+s2}{\PYZdq{}}\PY{l+s+s2}{activity\PYZus{}name}\PY{l+s+s2}{\PYZdq{}}\PY{p}{,} \PY{l+s+sa}{f}\PY{l+s+s2}{\PYZdq{}}\PY{l+s+si}{\PYZob{}}\PY{n}{i}\PY{l+s+si}{\PYZcb{}}\PY{l+s+s2}{\PYZus{}label}\PY{l+s+s2}{\PYZdq{}}\PY{p}{]}\PY{p}{]}
        \PY{p}{)}
    \PY{k}{return} \PY{n}{label\PYZus{}act\PYZus{}precision}
\end{Verbatim}
\end{tcolorbox}

    \textbf{Warning}: The cell below takes a very long time to run. A
debugger can be used to check it by executing the function line by line.

    \begin{tcolorbox}[breakable, size=fbox, boxrule=1pt, pad at break*=1mm,colback=cellbackground, colframe=cellborder]
\prompt{In}{incolor}{21}{\boxspacing}
\begin{Verbatim}[commandchars=\\\{\}]
\PY{n}{lab\PYZus{}score} \PY{o}{=} \PY{n}{pd}\PY{o}{.}\PY{n}{DataFrame}\PY{p}{(}\PY{n}{activity\PYZus{}precision}\PY{p}{(}\PY{p}{)}\PY{p}{)}
\PY{n}{lab\PYZus{}score}\PY{o}{.}\PY{n}{to\PYZus{}pickle}\PY{p}{(}\PY{l+s+s2}{\PYZdq{}}\PY{l+s+s2}{precision\PYZus{}score1.pkl}\PY{l+s+s2}{\PYZdq{}}\PY{p}{)}
\end{Verbatim}
\end{tcolorbox}

    \begin{tcolorbox}[breakable, size=fbox, boxrule=1pt, pad at break*=1mm,colback=cellbackground, colframe=cellborder]
\prompt{In}{incolor}{22}{\boxspacing}
\begin{Verbatim}[commandchars=\\\{\}]
\PY{n}{lab\PYZus{}score} \PY{o}{=} \PY{n}{pd}\PY{o}{.}\PY{n}{read\PYZus{}pickle}\PY{p}{(}\PY{l+s+s2}{\PYZdq{}}\PY{l+s+s2}{precision\PYZus{}score.pkl}\PY{l+s+s2}{\PYZdq{}}\PY{p}{)}
\end{Verbatim}
\end{tcolorbox}

    \begin{tcolorbox}[breakable, size=fbox, boxrule=1pt, pad at break*=1mm,colback=cellbackground, colframe=cellborder]
\prompt{In}{incolor}{135}{\boxspacing}
\begin{Verbatim}[commandchars=\\\{\}]
\PY{n}{acts} \PY{o}{=} \PY{n+nb}{list}\PY{p}{(}\PY{n}{lab\PYZus{}score}\PY{o}{.}\PY{n}{T}\PY{o}{.}\PY{n}{columns}\PY{p}{)}
\PY{k}{for} \PY{n}{act} \PY{o+ow}{in} \PY{n}{acts}\PY{p}{:}
  \PY{n}{temp} \PY{o}{=} \PY{n}{lab\PYZus{}score}\PY{p}{[}\PY{n}{lab\PYZus{}score}\PY{o}{.}\PY{n}{index}\PY{o}{==}\PY{n}{act}\PY{p}{]}
  \PY{n+nb}{print}\PY{p}{(}\PY{n}{act}\PY{p}{)}
  \PY{n+nb}{print}\PY{p}{(}\PY{l+s+sa}{f}\PY{l+s+s2}{\PYZdq{}}\PY{l+s+s2}{Maximum Precision Score: }\PY{l+s+si}{\PYZob{}}\PY{n}{temp}\PY{o}{.}\PY{n}{max}\PY{p}{(}\PY{n}{axis}\PY{o}{=}\PY{l+m+mi}{1}\PY{p}{)}\PY{l+s+si}{\PYZcb{}}\PY{l+s+s2}{\PYZdq{}}\PY{p}{)}
\end{Verbatim}
\end{tcolorbox}

    \begin{Verbatim}[commandchars=\\\{\}]
lying
Maximum Precision Score: lying    0.883875
dtype: float64
sitting
Maximum Precision Score: sitting    0.472997
dtype: float64
standing
Maximum Precision Score: standing    0.412574
dtype: float64
ironing
Maximum Precision Score: ironing    0.464035
dtype: float64
vacuum\_cleaning
Maximum Precision Score: vacuum\_cleaning    0.497042
dtype: float64
ascending\_stairs
Maximum Precision Score: ascending\_stairs    0.431175
dtype: float64
descending\_stairs
Maximum Precision Score: descending\_stairs    0.384129
dtype: float64
walking
Maximum Precision Score: walking    0.570877
dtype: float64
Nordic\_walking
Maximum Precision Score: Nordic\_walking    0.482387
dtype: float64
cycling
Maximum Precision Score: cycling    0.528448
dtype: float64
running
Maximum Precision Score: running    0.883534
dtype: float64
rope\_jumping
Maximum Precision Score: rope\_jumping    0.810501
dtype: float64
    \end{Verbatim}

    From above analysis, it is observed that lying has the maximum precision
score, implying it is easier to predict. The analysis is taken forward
using this information.

    \begin{tcolorbox}[breakable, size=fbox, boxrule=1pt, pad at break*=1mm,colback=cellbackground, colframe=cellborder]
\prompt{In}{incolor}{136}{\boxspacing}
\begin{Verbatim}[commandchars=\\\{\}]
\PY{k}{def} \PY{n+nf}{log\PYZus{}reg}\PY{p}{(}\PY{n}{model}\PY{p}{,} \PY{n}{split\PYZus{}type}\PY{p}{,} \PY{n}{activity\PYZus{}type}\PY{p}{)}\PY{p}{:}
    \PY{c+c1}{\PYZsh{} Function for acrrying out the Logistic Regression modelling}
    \PY{k}{if} \PY{n}{split\PYZus{}type} \PY{o}{==} \PY{l+s+s2}{\PYZdq{}}\PY{l+s+s2}{one\PYZus{}act}\PY{l+s+s2}{\PYZdq{}}\PY{p}{:}
        \PY{n}{x\PYZus{}train}\PY{p}{,} \PY{n}{x\PYZus{}val}\PY{p}{,} \PY{n}{x\PYZus{}test}\PY{p}{,} \PY{n}{y\PYZus{}train}\PY{p}{,} \PY{n}{y\PYZus{}val}\PY{p}{,} \PY{n}{y\PYZus{}test} \PY{o}{=} \PY{n}{model}\PY{o}{.}\PY{n}{split\PYZus{}one\PYZus{}act}\PY{p}{(}
            \PY{n}{activity\PYZus{}type}
        \PY{p}{)}
    \PY{k}{else}\PY{p}{:}
        \PY{p}{(}
            \PY{n}{x\PYZus{}train}\PY{p}{,}
            \PY{n}{x\PYZus{}val}\PY{p}{,}
            \PY{n}{x\PYZus{}test}\PY{p}{,}
            \PY{n}{y\PYZus{}train}\PY{p}{,}
            \PY{n}{y\PYZus{}val}\PY{p}{,}
            \PY{n}{y\PYZus{}test}\PY{p}{,}
            \PY{n}{le}\PY{p}{,}
        \PY{p}{)} \PY{o}{=} \PY{n}{model}\PY{o}{.}\PY{n}{train\PYZus{}test\PYZus{}split\PYZus{}actname}\PY{p}{(}\PY{p}{)}
    \PY{n}{pca} \PY{o}{=} \PY{n}{PCA}\PY{p}{(}\PY{n}{n\PYZus{}components}\PY{o}{=}\PY{l+m+mf}{0.99}\PY{p}{)}
    \PY{n}{x\PYZus{}train} \PY{o}{=} \PY{n}{pca}\PY{o}{.}\PY{n}{fit\PYZus{}transform}\PY{p}{(}\PY{n}{x\PYZus{}train}\PY{p}{)}
    \PY{n}{x\PYZus{}val} \PY{o}{=} \PY{n}{pca}\PY{o}{.}\PY{n}{transform}\PY{p}{(}\PY{n}{x\PYZus{}val}\PY{p}{)}
    \PY{n}{x\PYZus{}test} \PY{o}{=} \PY{n}{pca}\PY{o}{.}\PY{n}{transform}\PY{p}{(}\PY{n}{x\PYZus{}test}\PY{p}{)}
    \PY{n}{f1} \PY{o}{=} \PY{p}{[}\PY{p}{]}
    \PY{n}{acc} \PY{o}{=} \PY{p}{[}\PY{p}{]}
    \PY{n+nb}{print}\PY{p}{(}\PY{l+s+sa}{f}\PY{l+s+s2}{\PYZdq{}}\PY{l+s+s2}{Principal Component Feature size: }\PY{l+s+si}{\PYZob{}}\PY{n}{x\PYZus{}train}\PY{o}{.}\PY{n}{shape}\PY{p}{[}\PY{l+m+mi}{1}\PY{p}{]}\PY{l+s+si}{\PYZcb{}}\PY{l+s+s2}{\PYZdq{}}\PY{p}{)}
    \PY{k}{for} \PY{n}{lam} \PY{o+ow}{in} \PY{n}{np}\PY{o}{.}\PY{n}{arange}\PY{p}{(}\PY{l+m+mf}{0.1}\PY{p}{,} \PY{l+m+mi}{2}\PY{p}{,} \PY{l+m+mf}{0.1}\PY{p}{)}\PY{p}{:}
        \PY{n}{lr} \PY{o}{=} \PY{n}{LogisticRegression}\PY{p}{(}\PY{n}{solver}\PY{o}{=}\PY{l+s+s2}{\PYZdq{}}\PY{l+s+s2}{saga}\PY{l+s+s2}{\PYZdq{}}\PY{p}{,} \PY{n}{random\PYZus{}state}\PY{o}{=}\PY{l+m+mi}{30}\PY{p}{,} \PY{n}{C}\PY{o}{=}\PY{l+m+mi}{1} \PY{o}{/} \PY{n}{lam}\PY{p}{)}
        \PY{n}{lr}\PY{o}{.}\PY{n}{fit}\PY{p}{(}\PY{n}{x\PYZus{}train}\PY{p}{,} \PY{n}{y\PYZus{}train}\PY{p}{)}
        \PY{n}{f1}\PY{o}{.}\PY{n}{append}\PY{p}{(}\PY{n}{f1\PYZus{}score}\PY{p}{(}\PY{n}{y\PYZus{}val}\PY{p}{,} \PY{n}{lr}\PY{o}{.}\PY{n}{predict}\PY{p}{(}\PY{n}{x\PYZus{}val}\PY{p}{)}\PY{p}{,} \PY{n}{average}\PY{o}{=}\PY{l+s+s2}{\PYZdq{}}\PY{l+s+s2}{macro}\PY{l+s+s2}{\PYZdq{}}\PY{p}{)}\PY{p}{)}
        \PY{n}{acc}\PY{o}{.}\PY{n}{append}\PY{p}{(}\PY{n}{accuracy\PYZus{}score}\PY{p}{(}\PY{n}{y\PYZus{}val}\PY{p}{,} \PY{n}{lr}\PY{o}{.}\PY{n}{predict}\PY{p}{(}\PY{n}{x\PYZus{}val}\PY{p}{)}\PY{p}{)}\PY{p}{)}
    \PY{n}{df\PYZus{}lr} \PY{o}{=} \PY{n}{pd}\PY{o}{.}\PY{n}{DataFrame}\PY{p}{(}\PY{p}{)}
    \PY{n}{df\PYZus{}lr}\PY{p}{[}\PY{l+s+s2}{\PYZdq{}}\PY{l+s+s2}{validation\PYZus{}accuracy}\PY{l+s+s2}{\PYZdq{}}\PY{p}{]} \PY{o}{=} \PY{n}{acc}
    \PY{n}{df\PYZus{}lr}\PY{p}{[}\PY{l+s+s2}{\PYZdq{}}\PY{l+s+s2}{f1}\PY{l+s+s2}{\PYZdq{}}\PY{p}{]} \PY{o}{=} \PY{n}{f1}
    \PY{n}{df\PYZus{}lr}\PY{p}{[}\PY{l+s+s2}{\PYZdq{}}\PY{l+s+s2}{lambda}\PY{l+s+s2}{\PYZdq{}}\PY{p}{]} \PY{o}{=} \PY{n}{np}\PY{o}{.}\PY{n}{arange}\PY{p}{(}\PY{l+m+mf}{0.1}\PY{p}{,} \PY{l+m+mi}{2}\PY{p}{,} \PY{l+m+mf}{0.1}\PY{p}{)}
    \PY{k}{return} \PY{n}{df\PYZus{}lr}
\end{Verbatim}
\end{tcolorbox}

    A prediction model is made which determines if the subject is lying or
not at some time t. Logistic Regression is used to make the prediction
and a set of 19 different regularization parameters(\(\lambda\)) are
tested on the validation set to determine the best one.The features are
sorted in descending order of precision score and the top 4 features
with maximum precision score for lying are selected.Before feeding the
features into the Logistic Regression model, PCA transformation is
performed such that the resulting principal components explain 99\%
variance of the original data. PCA is important to perform before
Logistic Regression because Logistic Regression works on the assumption
that the input features are not correlated with each other.PCA
transforms orignal input data into non-correlated principal components
which can be used as inputs into the model.It is important to note that
the eigenvectors and the eigenvalues are only computed on the training
set. The Principle Components of the test set and validation set are
computes using the eigenvectors already computed from the training set.
If these eigenvalues were computed over the entire dataset, that would
mean that the model being trained on the training set is incorporating
information from the validation and testing set as well which would
jeopardize the prupose of splitting tha data into different sets.

    \hypertarget{metrics-for-measuring-performance}{%
\subsubsection{Metrics for Measuring
Performance}\label{metrics-for-measuring-performance}}

For determining how good this model is ,we use something called
F1-Score. In prediction tasks such as the one which concerns us, where
there will be a huge class imbalance. Accuracy is not a good measure for
such tasks as simply predicting the class with most number of occurences
will give us a very high accuracy score. Therefore we need 2 additional
measure for determininbg how good the classification is.

Preicison is computed as \[\frac{TP}{TP+FP}\],

where TP is the number of True Positives and FP is the number of False
Positives.

Recall is computed as \[\frac{TP}{TP+FN}\],

where FN is the number of false negatives.

Finally the F score is the harmonic mean of Precision and recall. F1
Score close to one implies that the classification model has a good
balance of precision and recall.

For the classification the best 4 features, i.e., the features with the
highest precision for the `lying' and the worst 4 features with lowest
precision for the same activity are considered and prediction using both
these types of features is performed.

    \begin{tcolorbox}[breakable, size=fbox, boxrule=1pt, pad at break*=1mm,colback=cellbackground, colframe=cellborder]
\prompt{In}{incolor}{25}{\boxspacing}
\begin{Verbatim}[commandchars=\\\{\}]
\PY{k}{def} \PY{n+nf}{one\PYZus{}act\PYZus{}model}\PY{p}{(}\PY{n}{act}\PY{p}{,} \PY{n}{low\PYZus{}index}\PY{p}{,} \PY{n}{up\PYZus{}index}\PY{p}{,} \PY{n}{lab\PYZus{}score}\PY{p}{)}\PY{p}{:}
    \PY{c+c1}{\PYZsh{} Return results for one activity classification modelling}
    \PY{n}{lab\PYZus{}score} \PY{o}{=} \PY{n}{lab\PYZus{}score}\PY{o}{.}\PY{n}{T}
    \PY{n}{best\PYZus{}feats} \PY{o}{=} \PY{n+nb}{list}\PY{p}{(}
        \PY{n}{lab\PYZus{}score}\PY{p}{[}\PY{n}{act}\PY{p}{]}\PY{o}{.}\PY{n}{sort\PYZus{}values}\PY{p}{(}\PY{n}{ascending}\PY{o}{=}\PY{k+kc}{False}\PY{p}{)}\PY{o}{.}\PY{n}{index}\PY{p}{[}\PY{n}{low\PYZus{}index}\PY{p}{:}\PY{n}{up\PYZus{}index}\PY{p}{]}
    \PY{p}{)}
    \PY{n}{model} \PY{o}{=} \PY{n}{modelling}\PY{p}{(}\PY{n}{clean\PYZus{}data\PYZus{}feats}\PY{p}{,} \PY{n}{best\PYZus{}feats}\PY{p}{)}
    \PY{n}{df\PYZus{}lr} \PY{o}{=} \PY{n}{log\PYZus{}reg}\PY{p}{(}\PY{n}{model}\PY{p}{,} \PY{l+s+s2}{\PYZdq{}}\PY{l+s+s2}{one\PYZus{}act}\PY{l+s+s2}{\PYZdq{}}\PY{p}{,} \PY{l+s+s2}{\PYZdq{}}\PY{l+s+s2}{lying}\PY{l+s+s2}{\PYZdq{}}\PY{p}{)}
    \PY{k}{return} \PY{n}{df\PYZus{}lr}\PY{p}{,} \PY{n}{model}\PY{p}{,} \PY{n}{best\PYZus{}feats}
\end{Verbatim}
\end{tcolorbox}

    \begin{tcolorbox}[breakable, size=fbox, boxrule=1pt, pad at break*=1mm,colback=cellbackground, colframe=cellborder]
\prompt{In}{incolor}{121}{\boxspacing}
\begin{Verbatim}[commandchars=\\\{\}]
\PY{n}{df\PYZus{}lr\PYZus{}best\PYZus{}feat}\PY{p}{,} \PY{n}{best\PYZus{}model}\PY{p}{,} \PY{n}{best\PYZus{}feats} \PY{o}{=} \PY{n}{one\PYZus{}act\PYZus{}model}\PY{p}{(}\PY{l+s+s2}{\PYZdq{}}\PY{l+s+s2}{lying}\PY{l+s+s2}{\PYZdq{}}\PY{p}{,} \PY{l+m+mi}{0}\PY{p}{,} \PY{l+m+mi}{4}\PY{p}{,} \PY{n}{lab\PYZus{}score}\PY{p}{)}
\PY{n}{df\PYZus{}lr\PYZus{}worst\PYZus{}feat}\PY{p}{,} \PY{n}{worst\PYZus{}model}\PY{p}{,} \PY{n}{worst\PYZus{}feats} \PY{o}{=} \PY{n}{one\PYZus{}act\PYZus{}model}\PY{p}{(}\PY{l+s+s2}{\PYZdq{}}\PY{l+s+s2}{lying}\PY{l+s+s2}{\PYZdq{}}\PY{p}{,} \PY{o}{\PYZhy{}}\PY{l+m+mi}{4}\PY{p}{,} \PY{o}{\PYZhy{}}\PY{l+m+mi}{1}\PY{p}{,} \PY{n}{lab\PYZus{}score}\PY{p}{)}
\end{Verbatim}
\end{tcolorbox}

    \begin{Verbatim}[commandchars=\\\{\}]
Feature size: 3
Feature size: 1
    \end{Verbatim}

    \begin{tcolorbox}[breakable, size=fbox, boxrule=1pt, pad at break*=1mm,colback=cellbackground, colframe=cellborder]
\prompt{In}{incolor}{70}{\boxspacing}
\begin{Verbatim}[commandchars=\\\{\}]
\PY{n}{copy}\PY{p}{(}\PY{n}{df\PYZus{}lr\PYZus{}best\PYZus{}feat}\PY{p}{)}
\end{Verbatim}
\end{tcolorbox}

    Best Features Classification Performance

\begin{longtable}[]{@{}rrrr@{}}
\toprule
& validation\_accuracy & f1 & lambda\tabularnewline
\midrule
\endhead
0 & 0.994152 & 0.981657 & 0.1\tabularnewline
1 & 0.994152 & 0.981657 & 0.2\tabularnewline
2 & 0.994152 & 0.981657 & 0.3\tabularnewline
3 & 0.994152 & 0.981657 & 0.4\tabularnewline
4 & 0.994152 & 0.981657 & 0.5\tabularnewline
5 & 0.994152 & 0.981657 & 0.6\tabularnewline
6 & 0.994152 & 0.981657 & 0.7\tabularnewline
7 & 0.994152 & 0.981657 & 0.8\tabularnewline
8 & 0.994152 & 0.981657 & 0.9\tabularnewline
9 & 0.994152 & 0.981657 & 1\tabularnewline
10 & 0.994152 & 0.981657 & 1.1\tabularnewline
11 & 0.994152 & 0.981657 & 1.2\tabularnewline
12 & 0.994152 & 0.981657 & 1.3\tabularnewline
13 & 0.994152 & 0.981657 & 1.4\tabularnewline
14 & 0.994152 & 0.981657 & 1.5\tabularnewline
15 & 0.994152 & 0.981657 & 1.6\tabularnewline
16 & 0.994152 & 0.981657 & 1.7\tabularnewline
17 & 0.994152 & 0.981657 & 1.8\tabularnewline
18 & 0.994152 & 0.981657 & 1.9\tabularnewline
\bottomrule
\end{longtable}

    \begin{tcolorbox}[breakable, size=fbox, boxrule=1pt, pad at break*=1mm,colback=cellbackground, colframe=cellborder]
\prompt{In}{incolor}{71}{\boxspacing}
\begin{Verbatim}[commandchars=\\\{\}]
\PY{n}{copy}\PY{p}{(}\PY{n}{df\PYZus{}lr\PYZus{}worst\PYZus{}feat}\PY{p}{)}
\end{Verbatim}
\end{tcolorbox}

    Worst Feature Classification Performance

\begin{longtable}[]{@{}rrrr@{}}
\toprule
& validation\_accuracy & f1 & lambda\tabularnewline
\midrule
\endhead
0 & 0.909747 & 0.47637 & 0.1\tabularnewline
1 & 0.909747 & 0.47637 & 0.2\tabularnewline
2 & 0.909747 & 0.47637 & 0.3\tabularnewline
3 & 0.909747 & 0.47637 & 0.4\tabularnewline
4 & 0.909747 & 0.47637 & 0.5\tabularnewline
5 & 0.909747 & 0.47637 & 0.6\tabularnewline
6 & 0.909747 & 0.47637 & 0.7\tabularnewline
7 & 0.909747 & 0.47637 & 0.8\tabularnewline
8 & 0.909747 & 0.47637 & 0.9\tabularnewline
9 & 0.909747 & 0.47637 & 1\tabularnewline
10 & 0.909747 & 0.47637 & 1.1\tabularnewline
11 & 0.909747 & 0.47637 & 1.2\tabularnewline
12 & 0.909747 & 0.47637 & 1.3\tabularnewline
13 & 0.909747 & 0.47637 & 1.4\tabularnewline
14 & 0.909747 & 0.47637 & 1.5\tabularnewline
15 & 0.909747 & 0.47637 & 1.6\tabularnewline
16 & 0.909747 & 0.47637 & 1.7\tabularnewline
17 & 0.909747 & 0.47637 & 1.8\tabularnewline
18 & 0.909747 & 0.47637 & 1.9\tabularnewline
\bottomrule
\end{longtable}

    

    The F1-Scores and accuracy clearly tells us that the features which has
the highest precision for lying outperform the ones with the lowest
precision for lying. This shows us that the precision metric is a really
good way of determining which features to use for classification
problems. Since all \(\lambda\) values give the same results we use just
\(\lambda = 0.9\) and test this final model on test set.

    \begin{tcolorbox}[breakable, size=fbox, boxrule=1pt, pad at break*=1mm,colback=cellbackground, colframe=cellborder]
\prompt{In}{incolor}{50}{\boxspacing}
\begin{Verbatim}[commandchars=\\\{\}]
\PY{n}{lam} \PY{o}{=} \PY{l+m+mf}{0.9}
\PY{n}{x\PYZus{}train}\PY{p}{,} \PY{n}{x\PYZus{}val}\PY{p}{,} \PY{n}{x\PYZus{}test}\PY{p}{,} \PY{n}{y\PYZus{}train}\PY{p}{,} \PY{n}{y\PYZus{}val}\PY{p}{,} \PY{n}{y\PYZus{}test} \PY{o}{=} \PY{n}{best\PYZus{}model}\PY{o}{.}\PY{n}{split\PYZus{}one\PYZus{}act}\PY{p}{(}\PY{l+s+s2}{\PYZdq{}}\PY{l+s+s2}{lying}\PY{l+s+s2}{\PYZdq{}}\PY{p}{)}
\PY{n}{lr} \PY{o}{=} \PY{n}{LogisticRegression}\PY{p}{(}\PY{n}{solver}\PY{o}{=}\PY{l+s+s2}{\PYZdq{}}\PY{l+s+s2}{saga}\PY{l+s+s2}{\PYZdq{}}\PY{p}{,} \PY{n}{random\PYZus{}state}\PY{o}{=}\PY{l+m+mi}{30}\PY{p}{,} \PY{n}{C}\PY{o}{=}\PY{l+m+mi}{1} \PY{o}{/} \PY{n}{lam}\PY{p}{)}
\PY{n}{lr}\PY{o}{.}\PY{n}{fit}\PY{p}{(}\PY{n}{x\PYZus{}train}\PY{p}{,} \PY{n}{y\PYZus{}train}\PY{p}{)}
\PY{n}{y\PYZus{}pred} \PY{o}{=} \PY{n}{lr}\PY{o}{.}\PY{n}{predict}\PY{p}{(}\PY{n}{x\PYZus{}test}\PY{p}{)}
\PY{n+nb}{print}\PY{p}{(}\PY{l+s+s2}{\PYZdq{}}\PY{l+s+s2}{Test Set Results}\PY{l+s+s2}{\PYZdq{}}\PY{p}{)}
\PY{n+nb}{print}\PY{p}{(}\PY{n}{classification\PYZus{}report}\PY{p}{(}\PY{n}{y\PYZus{}test}\PY{p}{,} \PY{n}{y\PYZus{}pred}\PY{p}{)}\PY{p}{)}
\PY{n+nb}{print}\PY{p}{(}\PY{l+s+sa}{f}\PY{l+s+s2}{\PYZdq{}}\PY{l+s+s2}{Time spent lying (predicted): }\PY{l+s+si}{\PYZob{}}\PY{n+nb}{list}\PY{p}{(}\PY{n}{y\PYZus{}pred}\PY{p}{)}\PY{o}{.}\PY{n}{count}\PY{p}{(}\PY{l+m+mi}{1}\PY{p}{)}\PY{l+s+si}{\PYZcb{}}\PY{l+s+s2}{ seconds}\PY{l+s+s2}{\PYZdq{}}\PY{p}{)}
\PY{n+nb}{print}\PY{p}{(}\PY{l+s+sa}{f}\PY{l+s+s2}{\PYZdq{}}\PY{l+s+s2}{Time spent lying (actual): }\PY{l+s+si}{\PYZob{}}\PY{n+nb}{list}\PY{p}{(}\PY{n}{y\PYZus{}test}\PY{p}{)}\PY{o}{.}\PY{n}{count}\PY{p}{(}\PY{l+m+mi}{1}\PY{p}{)}\PY{l+s+si}{\PYZcb{}}\PY{l+s+s2}{ seconds}\PY{l+s+s2}{\PYZdq{}}\PY{p}{)}
\end{Verbatim}
\end{tcolorbox}

    \begin{Verbatim}[commandchars=\\\{\}]
Test Set Results
              precision    recall  f1-score   support

           0       0.99      1.00      1.00      4451
           1       1.00      0.95      0.97       494

    accuracy                           0.99      4945
   macro avg       1.00      0.97      0.98      4945
weighted avg       0.99      0.99      0.99      4945

Time spent lying (predicted): 467 seconds
Time spent lying (actual): 494 seconds
    \end{Verbatim}

    \begin{tcolorbox}[breakable, size=fbox, boxrule=1pt, pad at break*=1mm,colback=cellbackground, colframe=cellborder]
\prompt{In}{incolor}{173}{\boxspacing}
\begin{Verbatim}[commandchars=\\\{\}]
\PY{n+nb}{print}\PY{p}{(}\PY{l+s+sa}{f}\PY{l+s+s2}{\PYZdq{}}\PY{l+s+s2}{Best Features for lying: }\PY{l+s+si}{\PYZob{}}\PY{n}{best\PYZus{}feats}\PY{l+s+si}{\PYZcb{}}\PY{l+s+s2}{\PYZdq{}}\PY{p}{)}
\PY{n+nb}{print}\PY{p}{(}\PY{l+s+sa}{f}\PY{l+s+s2}{\PYZdq{}}\PY{l+s+s2}{Worst Features for lying: }\PY{l+s+si}{\PYZob{}}\PY{n}{worst\PYZus{}feats}\PY{l+s+si}{\PYZcb{}}\PY{l+s+s2}{\PYZdq{}}\PY{p}{)}
\end{Verbatim}
\end{tcolorbox}

    \begin{Verbatim}[commandchars=\\\{\}]
Best Features for lying: ['ankle\_3D\_acceleration\_16\_y\_roll\_median',
'ankle\_3D\_acceleration\_16\_x\_roll\_mean',
'ankle\_3D\_acceleration\_16\_x\_roll\_median',
'chest\_3D\_acceleration\_16\_z\_roll\_median']
Worst Features for lying: ['heart\_rate\_roll\_var', 'ankle\_temperature\_roll\_var',
'chest\_temperature\_roll\_var']
    \end{Verbatim}

    Now an attempt is made to make a model to predict all 12 activities. To
select features for this, the top 4 features with the highest precision
for each activity is selected.

    \begin{tcolorbox}[breakable, size=fbox, boxrule=1pt, pad at break*=1mm,colback=cellbackground, colframe=cellborder]
\prompt{In}{incolor}{54}{\boxspacing}
\begin{Verbatim}[commandchars=\\\{\}]
\PY{n}{feat\PYZus{}score} \PY{o}{=} \PY{n}{lab\PYZus{}score}\PY{o}{.}\PY{n}{T}
\PY{n}{best\PYZus{}feats} \PY{o}{=} \PY{n}{np}\PY{o}{.}\PY{n}{concatenate}\PY{p}{(}
    \PY{p}{[}
        \PY{n+nb}{list}\PY{p}{(}\PY{n}{feat\PYZus{}score}\PY{p}{[}\PY{n}{act}\PY{p}{]}\PY{o}{.}\PY{n}{sort\PYZus{}values}\PY{p}{(}\PY{n}{ascending}\PY{o}{=}\PY{k+kc}{False}\PY{p}{)}\PY{o}{.}\PY{n}{index}\PY{p}{[}\PY{l+m+mi}{0}\PY{p}{:}\PY{l+m+mi}{4}\PY{p}{]}\PY{p}{)}
        \PY{k}{for} \PY{n}{act} \PY{o+ow}{in} \PY{n}{feat\PYZus{}score}\PY{o}{.}\PY{n}{columns}
    \PY{p}{]}
\PY{p}{)}
\PY{n}{best\PYZus{}feats} \PY{o}{=} \PY{n+nb}{list}\PY{p}{(}\PY{n+nb}{set}\PY{p}{(}\PY{n}{best\PYZus{}feats}\PY{p}{)}\PY{p}{)}    \PY{c+c1}{\PYZsh{} Removed duplicates}
\end{Verbatim}
\end{tcolorbox}

    \begin{tcolorbox}[breakable, size=fbox, boxrule=1pt, pad at break*=1mm,colback=cellbackground, colframe=cellborder]
\prompt{In}{incolor}{55}{\boxspacing}
\begin{Verbatim}[commandchars=\\\{\}]
\PY{n}{cluster\PYZus{}pred} \PY{o}{=} \PY{n}{modelling}\PY{p}{(}\PY{n}{clean\PYZus{}data\PYZus{}feats}\PY{p}{,} \PY{n}{best\PYZus{}feats}\PY{p}{)}
\PY{p}{(}
    \PY{n}{x\PYZus{}train}\PY{p}{,}
    \PY{n}{x\PYZus{}val}\PY{p}{,}
    \PY{n}{x\PYZus{}test}\PY{p}{,}
    \PY{n}{y\PYZus{}train}\PY{p}{,}
    \PY{n}{y\PYZus{}val}\PY{p}{,}
    \PY{n}{y\PYZus{}test}\PY{p}{,}
    \PY{n}{le}\PY{p}{,}
\PY{p}{)} \PY{o}{=} \PY{n}{cluster\PYZus{}pred}\PY{o}{.}\PY{n}{train\PYZus{}test\PYZus{}split\PYZus{}actname}\PY{p}{(}\PY{p}{)}
\end{Verbatim}
\end{tcolorbox}

    \begin{tcolorbox}[breakable, size=fbox, boxrule=1pt, pad at break*=1mm,colback=cellbackground, colframe=cellborder]
\prompt{In}{incolor}{56}{\boxspacing}
\begin{Verbatim}[commandchars=\\\{\}]
\PY{k}{def} \PY{n+nf}{determine\PYZus{}ncluster}\PY{p}{(}\PY{n}{x\PYZus{}train}\PY{p}{,} \PY{n}{y\PYZus{}train}\PY{p}{)}\PY{p}{:}
    \PY{c+c1}{\PYZsh{} Compute v measure for all cluster sizes for dinal clustering task}
    \PY{n}{v\PYZus{}measure\PYZus{}feats} \PY{o}{=} \PY{n}{defaultdict}\PY{p}{(}\PY{n+nb}{list}\PY{p}{)}
    \PY{k}{for} \PY{n}{ncluster} \PY{o+ow}{in} \PY{n+nb}{range}\PY{p}{(}\PY{l+m+mi}{12}\PY{p}{,} \PY{l+m+mi}{100}\PY{p}{,} \PY{l+m+mi}{5}\PY{p}{)}\PY{p}{:}
        \PY{n}{clust} \PY{o}{=} \PY{n}{cluster}\PY{o}{.}\PY{n}{KMeans}\PY{p}{(}\PY{n}{init}\PY{o}{=}\PY{l+s+s2}{\PYZdq{}}\PY{l+s+s2}{random}\PY{l+s+s2}{\PYZdq{}}\PY{p}{,} \PY{n}{random\PYZus{}state}\PY{o}{=}\PY{l+m+mi}{0}\PY{p}{,} \PY{n}{n\PYZus{}clusters}\PY{o}{=}\PY{n}{ncluster}\PY{p}{)}
        \PY{n}{clust}\PY{o}{.}\PY{n}{fit}\PY{p}{(}\PY{n}{x\PYZus{}train}\PY{p}{)}
        \PY{n}{y\PYZus{}lab} \PY{o}{=} \PY{n}{clust}\PY{o}{.}\PY{n}{predict}\PY{p}{(}\PY{n}{x\PYZus{}train}\PY{p}{)}
        \PY{n}{v\PYZus{}measure\PYZus{}feats}\PY{p}{[}\PY{n}{ncluster}\PY{p}{]}\PY{o}{.}\PY{n}{append}\PY{p}{(}\PY{n}{v\PYZus{}measure\PYZus{}score}\PY{p}{(}\PY{n}{y\PYZus{}train}\PY{p}{,} \PY{n}{y\PYZus{}lab}\PY{p}{)}\PY{p}{)}
        \PY{n+nb}{print}\PY{p}{(}\PY{l+s+sa}{f}\PY{l+s+s2}{\PYZdq{}}\PY{l+s+s2}{cluster }\PY{l+s+si}{\PYZob{}}\PY{n}{ncluster}\PY{l+s+si}{\PYZcb{}}\PY{l+s+s2}{ done}\PY{l+s+s2}{\PYZdq{}}\PY{p}{)}
    \PY{k}{return} \PY{n}{v\PYZus{}measure\PYZus{}feats}
\end{Verbatim}
\end{tcolorbox}

    \textbf{Warning}: The cell below takes a very long time to run. A
debugger can be used to check it by executing the function line by line.

    \begin{tcolorbox}[breakable, size=fbox, boxrule=1pt, pad at break*=1mm,colback=cellbackground, colframe=cellborder]
\prompt{In}{incolor}{58}{\boxspacing}
\begin{Verbatim}[commandchars=\\\{\}]
\PY{n}{v\PYZus{}measure\PYZus{}feats} \PY{o}{=} \PY{n}{determine\PYZus{}ncluster}\PY{p}{(}\PY{n}{x\PYZus{}train}\PY{p}{,} \PY{n}{y\PYZus{}train}\PY{p}{)}
\PY{n}{pd}\PY{o}{.}\PY{n}{DataFrame}\PY{p}{(}\PY{n}{v\PYZus{}measure\PYZus{}feats}\PY{p}{)}\PY{o}{.}\PY{n}{to\PYZus{}pickle}\PY{p}{(}\PY{l+s+s2}{\PYZdq{}}\PY{l+s+s2}{multiple\PYZus{}feature\PYZus{}vmeasure.pkl}\PY{l+s+s2}{\PYZdq{}}\PY{p}{)}
\end{Verbatim}
\end{tcolorbox}

    \begin{Verbatim}[commandchars=\\\{\}]
cluster 12 done
cluster 17 done
cluster 22 done
cluster 27 done
cluster 32 done
cluster 37 done
cluster 42 done
cluster 47 done
cluster 52 done
cluster 57 done
cluster 62 done
cluster 67 done
cluster 72 done
cluster 77 done
cluster 82 done
cluster 87 done
cluster 92 done
cluster 97 done
    \end{Verbatim}

    K-Means clustering is performed over the n dimensional data obtained
from selecting high precision features and these clusters are used to
compute the probability of the activities given a particular cluster.

Yet again, to determine the optimal number of clusters, V-measure score
is used and 57 is determined to be the optimal number of clusters

The probability is computed using the following formula
\[p_{ij} = \frac{n(i \cap j)}{n(j)}\]

,where \(n(i \cap j)\) is the count of occurence of activity i and j
together and \(n(j)\) is the count of occurence of cluster j. \(p_{ij}\)
is the probability of activity i given cluster j.

    \begin{tcolorbox}[breakable, size=fbox, boxrule=1pt, pad at break*=1mm,colback=cellbackground, colframe=cellborder]
\prompt{In}{incolor}{62}{\boxspacing}
\begin{Verbatim}[commandchars=\\\{\}]
\PY{n}{v\PYZus{}measure} \PY{o}{=} \PY{n}{pd}\PY{o}{.}\PY{n}{read\PYZus{}pickle}\PY{p}{(}\PY{l+s+s2}{\PYZdq{}}\PY{l+s+s2}{multiple\PYZus{}feature\PYZus{}vmeasure.pkl}\PY{l+s+s2}{\PYZdq{}}\PY{p}{)}
\PY{n}{ncluster} \PY{o}{=} \PY{n}{v\PYZus{}measure}\PY{o}{.}\PY{n}{idxmax}\PY{p}{(}\PY{n}{axis}\PY{o}{=}\PY{l+m+mi}{1}\PY{p}{)}\PY{o}{.}\PY{n}{values}\PY{p}{[}\PY{l+m+mi}{0}\PY{p}{]}
\PY{n+nb}{print}\PY{p}{(}\PY{l+s+sa}{f}\PY{l+s+s2}{\PYZdq{}}\PY{l+s+s2}{Optimal No. of Clusters:}\PY{l+s+si}{\PYZob{}}\PY{n}{ncluster}\PY{l+s+si}{\PYZcb{}}\PY{l+s+s2}{\PYZdq{}}\PY{p}{)}
\PY{n}{clf} \PY{o}{=} \PY{n}{cluster}\PY{o}{.}\PY{n}{KMeans}\PY{p}{(}\PY{n}{init}\PY{o}{=}\PY{l+s+s2}{\PYZdq{}}\PY{l+s+s2}{random}\PY{l+s+s2}{\PYZdq{}}\PY{p}{,} \PY{n}{n\PYZus{}clusters}\PY{o}{=}\PY{n}{ncluster}\PY{p}{,} \PY{n}{random\PYZus{}state}\PY{o}{=}\PY{l+m+mi}{0}\PY{p}{)}
\PY{n}{clf}\PY{o}{.}\PY{n}{fit}\PY{p}{(}\PY{n}{x\PYZus{}train}\PY{p}{,} \PY{n}{y\PYZus{}train}\PY{p}{)}
\PY{n}{x\PYZus{}train\PYZus{}labels} \PY{o}{=} \PY{n}{x\PYZus{}train}\PY{o}{.}\PY{n}{copy}\PY{p}{(}\PY{p}{)}
\PY{n}{x\PYZus{}train\PYZus{}labels}\PY{p}{[}\PY{l+s+s2}{\PYZdq{}}\PY{l+s+s2}{labels}\PY{l+s+s2}{\PYZdq{}}\PY{p}{]} \PY{o}{=} \PY{n}{clf}\PY{o}{.}\PY{n}{predict}\PY{p}{(}\PY{n}{x\PYZus{}train}\PY{p}{)}
\PY{n}{x\PYZus{}train\PYZus{}labels}\PY{p}{[}\PY{l+s+s2}{\PYZdq{}}\PY{l+s+s2}{activity\PYZus{}name}\PY{l+s+s2}{\PYZdq{}}\PY{p}{]} \PY{o}{=} \PY{n}{le}\PY{o}{.}\PY{n}{inverse\PYZus{}transform}\PY{p}{(}\PY{n}{y\PYZus{}train}\PY{p}{)}
\PY{n}{xc} \PY{o}{=} \PY{n}{pd}\PY{o}{.}\PY{n}{DataFrame}\PY{p}{(}
    \PY{n}{index}\PY{o}{=}\PY{n}{x\PYZus{}train\PYZus{}labels}\PY{o}{.}\PY{n}{activity\PYZus{}name}\PY{o}{.}\PY{n}{unique}\PY{p}{(}\PY{p}{)}\PY{p}{,}
    \PY{n}{columns}\PY{o}{=}\PY{n}{x\PYZus{}train\PYZus{}labels}\PY{o}{.}\PY{n}{labels}\PY{o}{.}\PY{n}{unique}\PY{p}{(}\PY{p}{)}\PY{p}{,}
\PY{p}{)}
\PY{k}{for} \PY{n}{i} \PY{o+ow}{in} \PY{n+nb}{range}\PY{p}{(}\PY{n}{ncluster}\PY{p}{)}\PY{p}{:}
    \PY{n}{temp} \PY{o}{=} \PY{n}{x\PYZus{}train\PYZus{}labels}\PY{p}{[}\PY{n}{x\PYZus{}train\PYZus{}labels}\PY{o}{.}\PY{n}{labels} \PY{o}{==} \PY{n}{i}\PY{p}{]}
    \PY{k}{for} \PY{n}{j} \PY{o+ow}{in} \PY{n}{x\PYZus{}train\PYZus{}labels}\PY{o}{.}\PY{n}{activity\PYZus{}name}\PY{o}{.}\PY{n}{unique}\PY{p}{(}\PY{p}{)}\PY{p}{:}
        \PY{n}{clust\PYZus{}prob} \PY{o}{=} \PY{n+nb}{len}\PY{p}{(}\PY{n}{temp}\PY{p}{[}\PY{n}{temp}\PY{o}{.}\PY{n}{activity\PYZus{}name} \PY{o}{==} \PY{n}{j}\PY{p}{]}\PY{p}{)} \PY{o}{/} \PY{n+nb}{len}\PY{p}{(}\PY{n}{temp}\PY{p}{)}
        \PY{n}{xc}\PY{o}{.}\PY{n}{loc}\PY{p}{[}\PY{n}{j}\PY{p}{,} \PY{n}{i}\PY{p}{]} \PY{o}{=} \PY{n}{clust\PYZus{}prob} \PY{c+c1}{\PYZsh{} Computing probability of activity given a cluster}
\PY{n+nb}{print}\PY{p}{(}\PY{l+s+s2}{\PYZdq{}}\PY{l+s+s2}{Probability of activity given a cluster label:}\PY{l+s+s2}{\PYZdq{}}\PY{p}{)}
\PY{n}{xc} \PY{o}{=} \PY{n}{xc}\PY{o}{.}\PY{n}{astype}\PY{p}{(}\PY{l+s+s2}{\PYZdq{}}\PY{l+s+s2}{float}\PY{l+s+s2}{\PYZdq{}}\PY{p}{)}
\end{Verbatim}
\end{tcolorbox}

    \begin{Verbatim}[commandchars=\\\{\}]
Optimal No. of Clusters:52
Probability of activity given a cluster label:
    \end{Verbatim}

    \begin{tcolorbox}[breakable, size=fbox, boxrule=1pt, pad at break*=1mm,colback=cellbackground, colframe=cellborder]
\prompt{In}{incolor}{96}{\boxspacing}
\begin{Verbatim}[commandchars=\\\{\}]
\PY{n}{xc}\PY{o}{.}\PY{n}{index}\PY{o}{.}\PY{n}{name} \PY{o}{=} \PY{l+s+s1}{\PYZsq{}}\PY{l+s+s1}{Activity Type}\PY{l+s+s1}{\PYZsq{}}
\PY{n}{xc}\PY{o}{.}\PY{n}{columns}\PY{o}{.}\PY{n}{name} \PY{o}{=} \PY{l+s+s1}{\PYZsq{}}\PY{l+s+s1}{Cluster Label}\PY{l+s+s1}{\PYZsq{}}
\PY{n+nb}{print}\PY{p}{(}\PY{l+s+s2}{\PYZdq{}}\PY{l+s+s2}{A condesed view of the probability table }\PY{l+s+se}{\PYZbs{}n}\PY{l+s+s2}{\PYZdq{}}\PY{p}{)}
\PY{n+nb}{print}\PY{p}{(}\PY{l+s+s2}{\PYZdq{}}\PY{l+s+s2}{(Probability of a  ceratin activty given a cluster label)}\PY{l+s+s2}{\PYZdq{}}\PY{p}{)}
\PY{n}{xc}\PY{p}{[}\PY{n}{xc}\PY{o}{.}\PY{n}{columns}\PY{p}{[}\PY{l+m+mi}{0}\PY{p}{:}\PY{l+m+mi}{4}\PY{p}{]}\PY{p}{]}
\end{Verbatim}
\end{tcolorbox}

    \begin{Verbatim}[commandchars=\\\{\}]
A condesed view of the probability table

(Probability of a  ceratin activty given a cluster label)
    \end{Verbatim}

            \begin{tcolorbox}[breakable, size=fbox, boxrule=.5pt, pad at break*=1mm, opacityfill=0]
\prompt{Out}{outcolor}{96}{\boxspacing}
\begin{Verbatim}[commandchars=\\\{\}]
Cluster Label            37      18        50        51
Activity Type
lying              0.193548  0.0625  0.119266  0.042857
sitting            0.161290  0.6250  0.009174  0.000000
standing           0.564516  0.0000  0.000000  0.000000
ironing            0.000000  0.0000  0.009174  0.000000
vacuum\_cleaning    0.000000  0.3125  0.477064  0.728571
ascending\_stairs   0.032258  0.0000  0.000000  0.085714
descending\_stairs  0.000000  0.0000  0.091743  0.128571
walking            0.048387  0.0000  0.110092  0.000000
Nordic\_walking     0.000000  0.0000  0.110092  0.000000
cycling            0.000000  0.0000  0.055046  0.014286
running            0.000000  0.0000  0.000000  0.000000
rope\_jumping       0.000000  0.0000  0.018349  0.000000
\end{Verbatim}
\end{tcolorbox}
        
    \begin{tcolorbox}[breakable, size=fbox, boxrule=1pt, pad at break*=1mm,colback=cellbackground, colframe=cellborder]
\prompt{In}{incolor}{93}{\boxspacing}
\begin{Verbatim}[commandchars=\\\{\}]
\PY{n}{xc}
\end{Verbatim}
\end{tcolorbox}

            \begin{tcolorbox}[breakable, size=fbox, boxrule=.5pt, pad at break*=1mm, opacityfill=0]
\prompt{Out}{outcolor}{93}{\boxspacing}
\begin{Verbatim}[commandchars=\\\{\}]
Cluster Label            37      18        50        51   0         47  \textbackslash{}
Activity Type
lying              0.193548  0.0625  0.119266  0.042857  1.0  0.013378
sitting            0.161290  0.6250  0.009174  0.000000  0.0  0.973244
standing           0.564516  0.0000  0.000000  0.000000  0.0  0.003344
ironing            0.000000  0.0000  0.009174  0.000000  0.0  0.000000
vacuum\_cleaning    0.000000  0.3125  0.477064  0.728571  0.0  0.010033
ascending\_stairs   0.032258  0.0000  0.000000  0.085714  0.0  0.000000
descending\_stairs  0.000000  0.0000  0.091743  0.128571  0.0  0.000000
walking            0.048387  0.0000  0.110092  0.000000  0.0  0.000000
Nordic\_walking     0.000000  0.0000  0.110092  0.000000  0.0  0.000000
cycling            0.000000  0.0000  0.055046  0.014286  0.0  0.000000
running            0.000000  0.0000  0.000000  0.000000  0.0  0.000000
rope\_jumping       0.000000  0.0000  0.018349  0.000000  0.0  0.000000

Cluster Label            32   7         40        36        14        21  \textbackslash{}
Activity Type
lying              0.789474  1.0  0.000000  0.000000  0.012308  0.000000
sitting            0.000000  0.0  0.589520  0.012448  0.024615  0.846154
standing           0.000000  0.0  0.384279  0.000000  0.030769  0.000000
ironing            0.000000  0.0  0.000000  0.966805  0.916923  0.000000
vacuum\_cleaning    0.210526  0.0  0.000000  0.000000  0.015385  0.000000
ascending\_stairs   0.000000  0.0  0.000000  0.000000  0.000000  0.076923
descending\_stairs  0.000000  0.0  0.000000  0.020747  0.000000  0.061538
walking            0.000000  0.0  0.004367  0.000000  0.000000  0.015385
Nordic\_walking     0.000000  0.0  0.000000  0.000000  0.000000  0.000000
cycling            0.000000  0.0  0.000000  0.000000  0.000000  0.000000
running            0.000000  0.0  0.021834  0.000000  0.000000  0.000000
rope\_jumping       0.000000  0.0  0.000000  0.000000  0.000000  0.000000

Cluster Label            11        8         34        23        3         27  \textbackslash{}
Activity Type
lying              0.000000  0.000000  0.006645  0.127753  0.000000  0.000000
sitting            0.000000  0.000000  0.000000  0.030837  0.070111  0.077982
standing           0.979167  0.000000  0.063123  0.044053  0.380074  0.600917
ironing            0.000000  0.007752  0.853821  0.198238  0.007380  0.036697
vacuum\_cleaning    0.000000  0.038760  0.073090  0.594714  0.298893  0.275229
ascending\_stairs   0.000000  0.054264  0.000000  0.000000  0.007380  0.009174
descending\_stairs  0.020833  0.085271  0.000000  0.004405  0.029520  0.000000
walking            0.000000  0.000000  0.000000  0.000000  0.092251  0.000000
Nordic\_walking     0.000000  0.077519  0.003322  0.000000  0.007380  0.000000
cycling            0.000000  0.713178  0.000000  0.000000  0.088561  0.000000
running            0.000000  0.000000  0.000000  0.000000  0.018450  0.000000
rope\_jumping       0.000000  0.023256  0.000000  0.000000  0.000000  0.000000

Cluster Label            45        12        48        15        4         41  \textbackslash{}
Activity Type
lying              0.000000  0.000000  0.000000  0.005714  0.000000  0.000000
sitting            0.000000  0.000000  0.000000  0.000000  0.000000  0.000000
standing           0.000000  0.000000  0.000000  0.000000  0.000000  0.000000
ironing            0.000000  0.000000  0.000000  0.000000  0.000000  0.000000
vacuum\_cleaning    0.872727  0.902778  0.312883  0.148571  0.087302  0.483871
ascending\_stairs   0.000000  0.069444  0.000000  0.051429  0.214286  0.306452
descending\_stairs  0.009091  0.013889  0.000000  0.005714  0.309524  0.145161
walking            0.009091  0.000000  0.042945  0.017143  0.015873  0.000000
Nordic\_walking     0.000000  0.000000  0.049080  0.000000  0.063492  0.000000
cycling            0.109091  0.013889  0.595092  0.765714  0.238095  0.064516
running            0.000000  0.000000  0.000000  0.000000  0.063492  0.000000
rope\_jumping       0.000000  0.000000  0.000000  0.005714  0.007937  0.000000

Cluster Label            13        33        6         10        16        30  \textbackslash{}
Activity Type
lying              0.000000  0.000000  0.000000  0.000000  0.000000  0.000000
sitting            0.007782  0.000000  0.000000  0.000000  0.000000  0.000000
standing           0.704280  0.000000  0.000000  0.000000  0.000000  0.000000
ironing            0.000000  0.000000  0.882353  0.000000  0.000000  0.000000
vacuum\_cleaning    0.151751  0.073529  0.066176  0.005587  0.006536  0.000000
ascending\_stairs   0.116732  0.040441  0.003676  0.251397  0.562092  0.521368
descending\_stairs  0.019455  0.058824  0.018382  0.078212  0.294118  0.410256
walking            0.000000  0.018382  0.000000  0.491620  0.117647  0.068376
Nordic\_walking     0.000000  0.000000  0.018382  0.145251  0.000000  0.000000
cycling            0.000000  0.790441  0.011029  0.027933  0.019608  0.000000
running            0.000000  0.000000  0.000000  0.000000  0.000000  0.000000
rope\_jumping       0.000000  0.018382  0.000000  0.000000  0.000000  0.000000

Cluster Label            31        26        49        9         46        44  \textbackslash{}
Activity Type
lying              0.000000  0.000000  0.000000  0.000000  0.000000  0.000000
sitting            0.000000  0.000000  0.000000  0.000000  0.000000  0.000000
standing           0.000000  0.000000  0.000000  0.000000  0.000000  0.000000
ironing            0.000000  0.000000  0.000000  0.000000  0.000000  0.000000
vacuum\_cleaning    0.000000  0.000000  0.000000  0.000000  0.000000  0.000000
ascending\_stairs   0.052960  0.236181  0.042254  0.216216  0.150407  0.468085
descending\_stairs  0.060748  0.040201  0.323944  0.729730  0.077236  0.489362
walking            0.878505  0.185930  0.626761  0.027027  0.012195  0.010638
Nordic\_walking     0.007788  0.361809  0.007042  0.009009  0.613821  0.021277
cycling            0.000000  0.080402  0.000000  0.000000  0.146341  0.000000
running            0.000000  0.045226  0.000000  0.000000  0.000000  0.000000
rope\_jumping       0.000000  0.050251  0.000000  0.018018  0.000000  0.010638

Cluster Label            2         39    5         42   25   38   43  \textbackslash{}
Activity Type
lying              0.000000  0.000000  0.00  0.000000  0.0  0.0  0.0
sitting            0.000000  0.008621  0.00  0.018519  0.0  0.0  0.0
standing           0.000000  0.000000  0.00  0.000000  0.0  0.0  0.0
ironing            0.000000  0.000000  0.00  0.012346  0.0  0.0  0.0
vacuum\_cleaning    0.000000  0.000000  0.00  0.000000  0.0  0.0  0.0
ascending\_stairs   0.068127  0.000000  0.00  0.000000  0.0  0.0  0.0
descending\_stairs  0.051095  0.000000  0.00  0.000000  0.0  0.0  0.0
walking            0.671533  0.000000  0.00  0.024691  0.0  0.0  0.0
Nordic\_walking     0.189781  0.982759  0.94  0.932099  0.0  0.0  0.0
cycling            0.019465  0.000000  0.04  0.000000  0.0  0.0  0.0
running            0.000000  0.008621  0.01  0.006173  1.0  1.0  1.0
rope\_jumping       0.000000  0.000000  0.01  0.006173  0.0  0.0  0.0

Cluster Label            22   24   29   28        1         20        19   17  \textbackslash{}
Activity Type
lying              0.000000  0.0  0.0  1.0  0.000000  0.000000  0.000000  1.0
sitting            0.000000  0.0  0.0  0.0  0.497984  0.000000  0.973214  0.0
standing           0.000000  0.0  0.0  0.0  0.439516  0.000000  0.000000  0.0
ironing            0.000000  0.0  0.0  0.0  0.016129  0.000000  0.000000  0.0
vacuum\_cleaning    0.000000  0.0  0.0  0.0  0.046371  0.000000  0.026786  0.0
ascending\_stairs   0.000000  0.0  0.0  0.0  0.000000  0.243478  0.000000  0.0
descending\_stairs  0.000000  0.0  0.0  0.0  0.000000  0.608696  0.000000  0.0
walking            0.000000  0.0  0.0  0.0  0.000000  0.000000  0.000000  0.0
Nordic\_walking     0.000000  0.0  0.0  0.0  0.000000  0.000000  0.000000  0.0
cycling            0.000000  0.0  0.0  0.0  0.000000  0.147826  0.000000  0.0
running            0.242424  0.0  0.0  0.0  0.000000  0.000000  0.000000  0.0
rope\_jumping       0.757576  1.0  1.0  0.0  0.000000  0.000000  0.000000  0.0

Cluster Label       35
Activity Type
lying              0.0
sitting            0.0
standing           0.0
ironing            0.0
vacuum\_cleaning    0.0
ascending\_stairs   0.0
descending\_stairs  0.0
walking            0.0
Nordic\_walking     0.0
cycling            0.0
running            1.0
rope\_jumping       0.0
\end{Verbatim}
\end{tcolorbox}
        
    \begin{tcolorbox}[breakable, size=fbox, boxrule=1pt, pad at break*=1mm,colback=cellbackground, colframe=cellborder]
\prompt{In}{incolor}{78}{\boxspacing}
\begin{Verbatim}[commandchars=\\\{\}]
\PY{k}{def} \PY{n+nf}{accuracy}\PY{p}{(}\PY{n}{x}\PY{p}{,} \PY{n}{y}\PY{p}{)}\PY{p}{:}
    \PY{n}{x\PYZus{}labels} \PY{o}{=} \PY{n}{pd}\PY{o}{.}\PY{n}{DataFrame}\PY{p}{(}\PY{n}{x}\PY{p}{)}\PY{o}{.}\PY{n}{copy}\PY{p}{(}\PY{p}{)}
    \PY{n}{x\PYZus{}labels}\PY{p}{[}\PY{l+s+s2}{\PYZdq{}}\PY{l+s+s2}{activity\PYZus{}name}\PY{l+s+s2}{\PYZdq{}}\PY{p}{]} \PY{o}{=} \PY{n}{le}\PY{o}{.}\PY{n}{inverse\PYZus{}transform}\PY{p}{(}\PY{n}{y}\PY{p}{)}
    \PY{n}{x\PYZus{}labels}\PY{p}{[}\PY{l+s+s2}{\PYZdq{}}\PY{l+s+s2}{labels}\PY{l+s+s2}{\PYZdq{}}\PY{p}{]} \PY{o}{=} \PY{n}{clf}\PY{o}{.}\PY{n}{predict}\PY{p}{(}\PY{n}{x}\PY{p}{)}
    \PY{n}{x\PYZus{}labels}\PY{p}{[}\PY{l+s+s2}{\PYZdq{}}\PY{l+s+s2}{predicted\PYZus{}activity}\PY{l+s+s2}{\PYZdq{}}\PY{p}{]} \PY{o}{=} \PY{n}{x\PYZus{}labels}\PY{o}{.}\PY{n}{labels}\PY{o}{.}\PY{n}{apply}\PY{p}{(}
        \PY{k}{lambda} \PY{n}{x}\PY{p}{:} \PY{n}{xc}\PY{p}{[}\PY{p}{[}\PY{n}{x}\PY{p}{]}\PY{p}{]}\PY{o}{.}\PY{n}{idxmax}\PY{p}{(}\PY{p}{)}\PY{o}{.}\PY{n}{values}\PY{p}{[}\PY{l+m+mi}{0}\PY{p}{]}
    \PY{p}{)}
    \PY{n+nb}{print}\PY{p}{(}
        \PY{n+nb}{len}\PY{p}{(}\PY{n}{x\PYZus{}labels}\PY{p}{[}\PY{n}{x\PYZus{}labels}\PY{o}{.}\PY{n}{activity\PYZus{}name} \PY{o}{==} \PY{n}{x\PYZus{}labels}\PY{o}{.}\PY{n}{predicted\PYZus{}activity}\PY{p}{]}\PY{p}{)}
        \PY{o}{/} \PY{n+nb}{len}\PY{p}{(}\PY{n}{x\PYZus{}labels}\PY{p}{)}
    \PY{p}{)}
    \PY{k}{return} \PY{n}{x\PYZus{}labels}
\end{Verbatim}
\end{tcolorbox}

    \begin{tcolorbox}[breakable, size=fbox, boxrule=1pt, pad at break*=1mm,colback=cellbackground, colframe=cellborder]
\prompt{In}{incolor}{76}{\boxspacing}
\begin{Verbatim}[commandchars=\\\{\}]
\PY{n+nb}{print}\PY{p}{(}\PY{l+s+s2}{\PYZdq{}}\PY{l+s+s2}{Validation accuracy for Clustering:}\PY{l+s+s2}{\PYZdq{}}\PY{p}{)}
\PY{n}{xval\PYZus{}lab}\PY{o}{=}\PY{n}{accuracy}\PY{p}{(}\PY{n}{x\PYZus{}val}\PY{p}{,}\PY{n}{y\PYZus{}val}\PY{p}{)}
\end{Verbatim}
\end{tcolorbox}

    \begin{Verbatim}[commandchars=\\\{\}]
Validation accuracy for Clustering:
0.7189083820662768
    \end{Verbatim}

    Logistic Regressionmodel is aldo trained using the 48 features to see if
this performs better.

    \begin{tcolorbox}[breakable, size=fbox, boxrule=1pt, pad at break*=1mm,colback=cellbackground, colframe=cellborder]
\prompt{In}{incolor}{74}{\boxspacing}
\begin{Verbatim}[commandchars=\\\{\}]
\PY{n}{df\PYZus{}lr} \PY{o}{=} \PY{n}{log\PYZus{}reg}\PY{p}{(}\PY{n}{cluster\PYZus{}pred}\PY{p}{,} \PY{l+s+s2}{\PYZdq{}}\PY{l+s+s2}{normal}\PY{l+s+s2}{\PYZdq{}}\PY{p}{,} \PY{l+s+s2}{\PYZdq{}}\PY{l+s+s2}{\PYZdq{}}\PY{p}{)}
\PY{n}{copy}\PY{p}{(}\PY{n}{df\PYZus{}lr}\PY{p}{)}
\end{Verbatim}
\end{tcolorbox}

    \begin{Verbatim}[commandchars=\\\{\}]
Feature size: 9
Validation accuracy for LR:
    \end{Verbatim}

    Validation Accuracy For Logistic Regression Multiple Activity
Classification

\begin{longtable}[]{@{}rrrr@{}}
\toprule
& validation\_accuracy & f1 & lambda\tabularnewline
\midrule
\endhead
0 & 0.654386 & 0.603246 & 0.1\tabularnewline
1 & 0.654386 & 0.603246 & 0.2\tabularnewline
2 & 0.654386 & 0.603246 & 0.3\tabularnewline
3 & 0.654386 & 0.603246 & 0.4\tabularnewline
4 & 0.654386 & 0.603246 & 0.5\tabularnewline
5 & 0.654386 & 0.603246 & 0.6\tabularnewline
6 & 0.654386 & 0.603246 & 0.7\tabularnewline
7 & 0.654386 & 0.603246 & 0.8\tabularnewline
8 & 0.654386 & 0.603246 & 0.9\tabularnewline
9 & 0.654386 & 0.603246 & 1\tabularnewline
10 & 0.654386 & 0.603246 & 1.1\tabularnewline
11 & 0.654386 & 0.603246 & 1.2\tabularnewline
12 & 0.654386 & 0.603246 & 1.3\tabularnewline
13 & 0.654386 & 0.603246 & 1.4\tabularnewline
14 & 0.654386 & 0.603246 & 1.5\tabularnewline
15 & 0.654386 & 0.603246 & 1.6\tabularnewline
16 & 0.654386 & 0.603246 & 1.7\tabularnewline
17 & 0.654386 & 0.603246 & 1.8\tabularnewline
18 & 0.654386 & 0.603246 & 1.9\tabularnewline
\bottomrule
\end{longtable}

    Since the validation accuracy of our Logistic Regression model is lesser
than that of the clustering model, Clustering is choosen as the final
model which will be evaluated on the test set.

    The results of the final model on the testing set are printed below

    \begin{tcolorbox}[breakable, size=fbox, boxrule=1pt, pad at break*=1mm,colback=cellbackground, colframe=cellborder]
\prompt{In}{incolor}{79}{\boxspacing}
\begin{Verbatim}[commandchars=\\\{\}]
\PY{n+nb}{print}\PY{p}{(}\PY{l+s+s2}{\PYZdq{}}\PY{l+s+s2}{Testing Accuracy}\PY{l+s+s2}{\PYZdq{}}\PY{p}{)}
\PY{n}{clust\PYZus{}test} \PY{o}{=} \PY{n}{accuracy}\PY{p}{(}\PY{n}{x\PYZus{}test}\PY{p}{,} \PY{n}{y\PYZus{}test}\PY{p}{)}
\PY{n}{clust\PYZus{}test}\PY{p}{[}\PY{l+s+s2}{\PYZdq{}}\PY{l+s+s2}{id}\PY{l+s+s2}{\PYZdq{}}\PY{p}{]} \PY{o}{=} \PY{n}{clean\PYZus{}data\PYZus{}feats}\PY{p}{[}\PY{n}{clean\PYZus{}data\PYZus{}feats}\PY{o}{.}\PY{n}{id}\PY{o}{.}\PY{n}{isin}\PY{p}{(}\PY{p}{[}\PY{l+m+mi}{107}\PY{p}{,} \PY{l+m+mi}{108}\PY{p}{]}\PY{p}{)}\PY{p}{]}\PY{o}{.}\PY{n}{id}
\PY{k}{for} \PY{n}{subj} \PY{o+ow}{in} \PY{p}{[}\PY{l+m+mi}{107}\PY{p}{,} \PY{l+m+mi}{108}\PY{p}{]}\PY{p}{:}
    \PY{n}{subj\PYZus{}df} \PY{o}{=} \PY{n}{clust\PYZus{}test}\PY{p}{[}\PY{n}{clust\PYZus{}test}\PY{o}{.}\PY{n}{id} \PY{o}{==} \PY{n}{subj}\PY{p}{]}
    \PY{n}{act\PYZus{}freq\PYZus{}predicted} \PY{o}{=} \PY{n}{subj\PYZus{}df}\PY{o}{.}\PY{n}{predicted\PYZus{}activity}\PY{o}{.}\PY{n}{value\PYZus{}counts}\PY{p}{(}\PY{p}{)}
    \PY{n}{act\PYZus{}freq\PYZus{}actual} \PY{o}{=} \PY{n}{subj\PYZus{}df}\PY{o}{.}\PY{n}{activity\PYZus{}name}\PY{o}{.}\PY{n}{value\PYZus{}counts}\PY{p}{(}\PY{p}{)}
    \PY{n+nb}{print}\PY{p}{(}\PY{l+s+sa}{f}\PY{l+s+s2}{\PYZdq{}}\PY{l+s+s2}{For subject }\PY{l+s+si}{\PYZob{}}\PY{n}{subj}\PY{l+s+si}{\PYZcb{}}\PY{l+s+s2}{\PYZdq{}}\PY{p}{)}
    \PY{k}{for} \PY{n}{i} \PY{o+ow}{in} \PY{n}{subj\PYZus{}df}\PY{o}{.}\PY{n}{activity\PYZus{}name}\PY{o}{.}\PY{n}{unique}\PY{p}{(}\PY{p}{)}\PY{p}{:}
        \PY{n+nb}{print}\PY{p}{(}\PY{l+s+sa}{f}\PY{l+s+s2}{\PYZdq{}}\PY{l+s+s2}{Time spent }\PY{l+s+si}{\PYZob{}}\PY{n}{i}\PY{l+s+si}{\PYZcb{}}\PY{l+s+s2}{ (predicted) : }\PY{l+s+si}{\PYZob{}}\PY{n}{act\PYZus{}freq\PYZus{}predicted}\PY{p}{[}\PY{n}{i}\PY{p}{]}\PY{l+s+si}{\PYZcb{}}\PY{l+s+s2}{ seconds}\PY{l+s+s2}{\PYZdq{}}\PY{p}{)}
        \PY{n+nb}{print}\PY{p}{(}\PY{l+s+sa}{f}\PY{l+s+s2}{\PYZdq{}}\PY{l+s+s2}{Time spent }\PY{l+s+si}{\PYZob{}}\PY{n}{i}\PY{l+s+si}{\PYZcb{}}\PY{l+s+s2}{ (actual) : }\PY{l+s+si}{\PYZob{}}\PY{n}{act\PYZus{}freq\PYZus{}actual}\PY{p}{[}\PY{n}{i}\PY{p}{]}\PY{l+s+si}{\PYZcb{}}\PY{l+s+s2}{ seconds}\PY{l+s+s2}{\PYZdq{}}\PY{p}{)}
\end{Verbatim}
\end{tcolorbox}

    \begin{Verbatim}[commandchars=\\\{\}]
Testing Accuracy
0.6151668351870576
For subject 107
Time spent lying (predicted) : 245 seconds
Time spent lying (actual) : 254 seconds
Time spent sitting (predicted) : 385 seconds
Time spent sitting (actual) : 123 seconds
Time spent standing (predicted) : 129 seconds
Time spent standing (actual) : 257 seconds
Time spent ironing (predicted) : 334 seconds
Time spent ironing (actual) : 295 seconds
Time spent vacuum\_cleaning (predicted) : 128 seconds
Time spent vacuum\_cleaning (actual) : 216 seconds
Time spent ascending\_stairs (predicted) : 142 seconds
Time spent ascending\_stairs (actual) : 176 seconds
Time spent descending\_stairs (predicted) : 121 seconds
Time spent descending\_stairs (actual) : 117 seconds
Time spent walking (predicted) : 361 seconds
Time spent walking (actual) : 337 seconds
Time spent Nordic\_walking (predicted) : 312 seconds
Time spent Nordic\_walking (actual) : 287 seconds
Time spent cycling (predicted) : 137 seconds
Time spent cycling (actual) : 227 seconds
Time spent running (predicted) : 31 seconds
Time spent running (actual) : 37 seconds
For subject 108
Time spent lying (predicted) : 235 seconds
Time spent lying (actual) : 240 seconds
Time spent sitting (predicted) : 259 seconds
Time spent sitting (actual) : 229 seconds
Time spent standing (predicted) : 259 seconds
Time spent standing (actual) : 251 seconds
Time spent ironing (predicted) : 21 seconds
Time spent ironing (actual) : 330 seconds
Time spent vacuum\_cleaning (predicted) : 546 seconds
Time spent vacuum\_cleaning (actual) : 243 seconds
Time spent ascending\_stairs (predicted) : 59 seconds
Time spent ascending\_stairs (actual) : 117 seconds
Time spent descending\_stairs (predicted) : 178 seconds
Time spent descending\_stairs (actual) : 97 seconds
Time spent walking (predicted) : 377 seconds
Time spent walking (actual) : 315 seconds
Time spent cycling (predicted) : 170 seconds
Time spent cycling (actual) : 255 seconds
Time spent Nordic\_walking (predicted) : 302 seconds
Time spent Nordic\_walking (actual) : 289 seconds
Time spent running (predicted) : 145 seconds
Time spent running (actual) : 165 seconds
Time spent rope\_jumping (predicted) : 68 seconds
Time spent rope\_jumping (actual) : 88 seconds
    \end{Verbatim}

    \begin{tcolorbox}[breakable, size=fbox, boxrule=1pt, pad at break*=1mm,colback=cellbackground, colframe=cellborder]
\prompt{In}{incolor}{80}{\boxspacing}
\begin{Verbatim}[commandchars=\\\{\}]
\PY{n+nb}{print}\PY{p}{(}\PY{l+s+s2}{\PYZdq{}}\PY{l+s+s2}{Features used: }\PY{l+s+s2}{\PYZdq{}}\PY{p}{)}
\PY{n+nb}{print}\PY{p}{(}\PY{n}{best\PYZus{}feats}\PY{p}{)}
\end{Verbatim}
\end{tcolorbox}

    \begin{Verbatim}[commandchars=\\\{\}]
Features used:
['chest\_3D\_magnetometer\_y\_roll\_median', 'chest\_temperature',
'chest\_3D\_acceleration\_16\_z\_roll\_median',
'ankle\_3D\_magnetometer\_z\_spectral\_centroid',
'chest\_3D\_acceleration\_16\_y\_roll\_var', 'ankle\_3D\_magnetometer\_x\_roll\_mean',
'hand\_3D\_acceleration\_16\_y\_roll\_var', 'chest\_3D\_gyroscope\_y\_roll\_mean',
'chest\_3D\_acceleration\_16\_z\_roll\_var', 'ankle\_3D\_acceleration\_16\_x\_roll\_mean',
'ankle\_3D\_gyroscope\_y\_roll\_mean', 'ankle\_3D\_gyroscope\_z\_roll\_var',
'ankle\_3D\_acceleration\_16\_y\_roll\_median',
'chest\_3D\_acceleration\_16\_z\_roll\_mean',
'hand\_3D\_acceleration\_16\_x\_spectral\_centroid',
'ankle\_3D\_magnetometer\_z\_roll\_mean', 'hand\_temperature',
'hand\_temperature\_roll\_mean', 'heart\_rate\_roll\_median',
'ankle\_3D\_magnetometer\_x', 'ankle\_3D\_magnetometer\_z',
'ankle\_3D\_acceleration\_16\_x\_roll\_median', 'chest\_3D\_gyroscope\_x\_roll\_var',
'hand\_3D\_acceleration\_16\_x\_roll\_mean', 'chest\_3D\_magnetometer\_y\_roll\_mean',
'ankle\_3D\_magnetometer\_z\_roll\_median', 'ankle\_3D\_magnetometer\_x\_roll\_median',
'ankle\_3D\_acceleration\_16\_x\_roll\_var', 'chest\_3D\_gyroscope\_y\_roll\_median',
'ankle\_3D\_gyroscope\_x\_roll\_mean', 'hand\_3D\_acceleration\_16\_z\_spectral\_centroid']
    \end{Verbatim}

    \hypertarget{conclusion}{%
\section{Conclusion}\label{conclusion}}

\begin{enumerate}
\def\labelenumi{\arabic{enumi}.}
\item
  It is relatively easy to predict if a subject is lying or not.
\item
  Time domain features like mean,median and variance and frequency
  domain features like spectral centroid computed over a sliding window
  are useful features for performing classification.
\item
  Metrics such as precision score can be very useful to select features
  to feed into classifier for the task of activity prediction.
\end{enumerate}

    \hypertarget{references}{%
\section{References}\label{references}}

\begin{enumerate}
\def\labelenumi{\arabic{enumi}.}
\item
  Parkka, J., Ermes, M., Korpipaa, P., Mantyjarvi, J., Peltola, J. and
  Korhonen, I., 2006. Activity classification using realistic data from
  wearable sensors. IEEE Transactions on information technology in bio
\item
  Ermes, M., Parkka, J. and Cluitmans, L., 2008, August. Advancing from
  offline to online activity recognition with wearable sensors. In 2008
  30th annual international conference of the ieee engineering in
  medicine and biology society (pp.~4451-4454). IEEE.
\item
  Huynh, T. and Schiele, B., 2005, October. Analyzing features for
  activity recognition. In Proceedings of the 2005 joint conference on
  Smart objects and ambient intelligence: innovative context-aware
  services: usages and technologies (pp.~159-163).
\end{enumerate}

    \begin{tcolorbox}[breakable, size=fbox, boxrule=1pt, pad at break*=1mm,colback=cellbackground, colframe=cellborder]
\prompt{In}{incolor}{ }{\boxspacing}
\begin{Verbatim}[commandchars=\\\{\}]

\end{Verbatim}
\end{tcolorbox}


    % Add a bibliography block to the postdoc
    
    
    
\end{document}
